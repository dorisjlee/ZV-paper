%!TEX root = main.tex
\section{Evaluation Study Findings\label{sec:eval_findings}}
% \begin{figure*}[t!]
% \minipage{0.6\textwidth}
%   \includegraphics[width=\linewidth]{figures/evalstudytimeline.pdf}
%   \caption{Timeline of event code and component usage, with every timepoint as an event on the x axis. For clarity, we hide most of the event coding labels other than the insight labels. Black vertical tick indicates a session break, signaling the beginning of a new line of inquiry.}\label{fig:evalstudytimeline}
% \endminipage\hfill
% \minipage{0.4\textwidth}
%   \includegraphics[width=0.8\linewidth]{figures/PENcoding.pdf}
%   \caption{Heatmap of features categorized as practical usage (P), envisioned usage (E), and not useful (N).  \techreport{We find that participants preferred to query using bottom-up methods such as drag-and-drop over top-down approaches such as sketching or input equations. Participants found that data faceting via filter constraints and dynamic class creation were powerful ways to compare between subgroups or filtered subsets. The columns are arranged in the order of subject areas and the features are arranged in the order of the three foraging acts.}}\label{fig:feature_heatmap}
% \endminipage
% \end{figure*}
We recorded audio, video screen captures, and click-stream logs of the participants' actions during the evaluation study. We analyzed transcriptions of these recordings through open-coding and categorized every event in the user study as either a feature usage or via event coding labels. The event codes included insights or provoked actions related to science or data, occasions when participants were confused or wanted features that was unaddressed by the system, and the use of external tools.
% \begin{denselist}
%     \item Insight (Science): Insight that connected back to the science (e.g. ``This cluster resembles a repressed gene.'')
%     \item Insight (Data): Data-related insights (e.g. ``A bug in my data cleaning code generated this peak artifact.'')
%     \item Provoke (Science): Interactions or observations made while using the VQS that provoked a scientific hypothesis to be generated.
%     \item Provoke (Data): Interactions or observations made while using the VQS that provoked further data actions to continue the investigation.
%     \item Confusion: Participants were confused during this part of the analysis.
%     \item Want: Additional features that participant wants, which is not currently available on the system.
%     \item External Tools: The use of external tools outside of \zv to complement the analysis process.
% \end{denselist}
To characterize the usefulness of each feature, we further categorized the features into one of the three usage types based on how each feature was used during the study:
\begin{denselist}
    \item Practical: Features used in a sensible and meaningful way.
    \item Envisioned: Features which could be used practically if the envisioned data was available or if they conducted downstream analysis, but was not performed due to the limited time during the study.
    \item Not useful: Features that are not useful or do not make sense for the participant's research question and dataset.
\end{denselist}
\begin{figure}[h!]
  \includegraphics[width=0.8\linewidth]{figures/PENcoding.pdf}
  \caption{Heatmap of features categorized as practical usage, envisioned usage, and not useful. \techreport{We find that participants preferred to query using bottom-up methods such as drag-and-drop over top-down approaches such as sketching or input equations. Participants found that data faceting via filter constraints and dynamic class creation were powerful ways to compare between subgroups or filtered subsets. The columns are arranged in the order of subject areas and the features are arranged in the order of the three foraging acts.}}
  \label{fig:feature_heatmap}
\end{figure}
We derived these labels from the study transcription to circumvent self-reporting bias, which can often artificially inflate the usefulness of the feature or tool under examination.

For the remaining paper, we will discuss the results of the thematic analysis to understand the usage of these features and sensemaking paradigms in real-world analytic tasks.
%focus on understanding the design space of VQSs and highlight the takeaways of our study.%developing a process model and design guideline for insight formation in VQSs and divert our thematic analysis of how VQSs fit into the context of an analysis workflow to our technical report.% These observation inform our ----- search-browse paradigm
% \subsubsection{Discovery of Real-world insights}
% \par Our participants' original workflow often required them to compare between many visualizations manually through separate analysis and visualization steps. Three of the participants cited that this segmented analyze-then-visualize workflow was one of their chief bottlenecks. The cognitive overhead from the segmented workflow made them more hesitant to visualize the results of different parameters and data operations, as A2 noted:
% \begin{quote}
% The quick visualization is something that I could not do on my current framework. I could not query as fast as you do; I need to wait for it, plot, and then compare. Every time I plot, I need to define subplots for 12 visualizations, then its slower. That's the reason why I sometimes plot less, and I rely more on the statistics from the likelihood tests. Sometimes I plot less than I really should be doing.
% \end{quote}
% The ability to rapidly experiment with large numbers of hypotheses in real time is a crucial step in the agile creative process in helping analysts discover actionable insights~\cite{Shneiderman2007a}. Five out of nine participants discussed how the dynamic, interactive update of the visualization in \zv was the main advantage for using VQSs over their original workflow.
\subsection{DP1: The Inefficiency of Sketch}
% \subsection{DC3: Closing the loop in VQS sense-making cycle with bottom-up data-driven inquries}
\par Our interactions with the scientists showed that different modalities for inputting a query can be useful for different problem contexts. To our surprise, despite the prevalence of sketch-to-query systems in literature,
Figure \ref{fig:feature_heatmap} shows that only two out of our nine users had a practical usage for querying by sketching. %Overall, bottom-up querying via drag-and-drop was more intuitive and more commonly used than top-down querying methods, such as sketching or input equations.
\par The main reason why participants did not find sketching useful was that they often do not start their analysis with a pattern in mind. Later, their intuition about what to query is derived from other visualizations that they see in the VQS, in which case it made more sense to query using those visualizations as examples directly. In addition, even if a user has a query pattern in mind, sketch queries can be ambiguous or even impossible to draw by sketching (e.g. A2 looked for a highly-varying signal enveloped by a sinusoidal pattern indicating planetary rotation \includegraphics[width=1.7\baselineskip,keepaspectratio]{figures/impossible_sketch.png}).
\par The latter case is also supported by the unexpected use cases where sketching was simply used as a mechanism to modify dragged-and-dropped queries. As shown in Figure \ref{query_modification} (top), M2 first sketched a pattern to find solvent classes with anticorrelated properties. However, the sketched query did not return visualizations of interest. So, he instead dragged and dropped one of the peripheral visualizations that was close enough to his desired visualization to the sketchpad and then smoothed out the noise due to outlier datapoints by tracing a sketch over the visualization. M2 repeated this workflow twice in separate occurrences during the study and was able to derive insights from the results. Likewise, A3 was interested in pulsating stars characterized by dramatic changes in the amplitudes of the light curves. During the search, hotspots on stellar surfaces often show up as false positives as they also result in dramatic amplitude fluctuations, but happen at a regular intervals. In the VQS, A3 looked for patterns that exhibits amplitude variations, but also some irregularities. As shown in Figure \ref{query_modification} (bottom), she first picked out a regular pattern (suspected star spot), then modified it slightly so that the pattern looks more irregular.
\begin{figure}[ht!]
    \centering
    \includegraphics[width=\columnwidth]{figures/QueryModificationBySketch.pdf}
    \caption{\tvcg{Examples of query modification by M2 (top) and A3 (bottom) performed during the study The inital drag-and-dropped query is shown in blue and the sketch-modified queries in red.}
    \label{query_modification}}
    \vspace{-10pt}
\end{figure}
\par The lack of practical use of top-down pattern specification is also reflected in the fact that querying by equation is unpopular. Both querying mechanism adopt a problematic assumption that analysts start with a known and easy-to-specify search pattern in mind. In both astronomy and genetics use cases, the visualization patterns result from complex physical processes that could not be written down as an equation analytically. Even in the case of material science when analytical relationships do exist, it is challenging to formulate functional forms in an prescriptive, ad-hoc manner.
% Despite functional fitting being common in scientific data analysis, Figure \ref{feature_heatmap} shows that
% . However,
\par Both of these findings suggest that while sketching is an useful analogy for people to express their queries, \emph{the existing ad-hoc, sketch-only model for visualization querying is insufficient without data examples that can help analysts jumpstart their exploration}. Table~\ref{table:relatedwork} show that most past work focus on optimizing the components in the top-down paradigm, missing out largely on the key components in the other two paradigms, indicated by the absence of green features on the right hand side of the table. We suspect that the limited coverage in addressing different types of analytics use cases may be why existing sketch-to-query systems are not commonly adopted in practice. %This result points to a need for ----- in future VQSs. %This, however, points to an exciting direction for sketching interface in VQSs for developing advanced drawing and modification tools that enable more precise visualization query specification.}
%For instance, material science discovered a known inverse relationship during exploration
%Which is really interesting. Which is something that we observed experimentally also. That is an interesting insight right htere. This seems to suggest that there is a fundamental issue in if you want to try to get better on this axis, and get as low as possible, you lose out on the other axis.
%once they see it they know it but they don't know beforehand

\subsection{DP2: Practical Use of Bottom-up approaches}
\par Our results indicate that \emph{bottom-up data-driven inquiries are more common than top-down pattern specification when the users have no desired patterns in mind}, which is commonly the case for exploratory data analysis. Examples of practical uses of result querying includes inspecting the top-most similar visualizations that lie in a cluster and finding visualizations that are similar to an object of interest that exhibits a desired pattern.
\par Likewise, many participants envisioned use cases for pattern loading. The ability to load in data patterns as a query would enable users to compare visualizations between different experiments, species, or surveys, query with known patterns from an external reference catalog (e.g. important genes of interest, objects labeled as supernovae), or verify the results of a simulation or downstream analysis by finding similar patterns in their existing dataset. Users can also specify a more precise query that captures the essential shape features of a desired pattern (e.g. amplitude, width of peak), that cannot be precisely sketched. For example, the width of a supernovae light curve is characteristic to the radioactive decay rate of its chemical signature~\cite{Nugent1997}, so querying with an exact pattern template would be helpful for distinguishing the patterns of interest from noise.
\par The prevalence of bottom-up approaches not only point to the need for supporting result querying in VQSs, but also to the need for providing recommendation for users who may not have a desired pattern in mind. We found that geneticists often gain their intuition about the data from the recommended representative trends. One example of rapid insight discovery comes from G2 and G3, who identified that the three representative patterns shown in \zv---induced genes (profiles with expression levels staying up), repressed genes (started high but went down), and transients (go up and then come down at different time points)---corresponded to the same three groups of genes discussed in a recent publication\cite{Gloss2017}. The clusters provoked G2 to generate a hypothesis regarding the properties of transients: \textit{``Is that because all the transient groups get clustered together, can I get sharp patterns that rise and ebb at different time points?''} To verify this hypothesis, G2 increased the parameter controlling the number of clusters and noticed that the cluster no longer exhibited the clean, intuitive patterns he had seen earlier. G3 expressed a similar sentiment and proceeded by inspecting the visualizations in the cluster via drag-and-drop. He found a group of genes that all transitioned at the same timestep, while others transitioned at different timesteps. G3 described the process of using VQSs as doing ``detective work'' that provoked him to generate further scientific hypotheses as well as data actions.
\par By browsing through the ranked list of result, representative, and outlier in \zv, participants were also able to gain a peripheral overview of the data and spot anomalies during exploration. For example, A1 spotted time series that were too faint to look like stars after applying a filter constraint of CLASS\_STAR=1. After a series of query results browsing and consultation with an external database, he concluded that the dataset had been incorrectly labelled with all the stars with CLASS\_STAR=0 as 1 during data cleaning. These examples show that both the browsing-act through recommendations and performing search via these results are essential for `closing the loop' between the sensemaking acts in VQSs.
\subsection{DP3: Enriching Search with Context}
\par Past studies in taxonomies of visualization tasks have shown that it is important to design features that enable users to select relevant subsets of data in visual analytics\cite{Amar2005,Heer2012}. %We designed two dynamic faceting features coupled with coordinated views that enabled users to specify subsets of data they are querying on and see immediate changes updated in the query, representative, and outlier results.
We found that all participants either envisioned a use case or utilized components of the context creation paradigm offered in \zv to explore and compare subsets of their data.
\par A1 expressed that even though the filtering step could be easily done programmatically on the dataset and reloaded into \zv, filtering on-the-fly was a powerful way to dynamically test his hypothesis. Interactive filtering lowers the barrier between the iterative hypothesize-then-compare cycle, thereby enabling participants to test conditions and tune values that they would not have otherwise modified.
% echoing our previous finding that segmented workflow prevents extensive exploration.
During the study, participants used filtering to address questions such as: \textit{Are there more genes similar to a known activator when we subselect only the differentially expressed genes?} \texttt{DIFFEXP=1} (G2) or \textit{Can I find more supernovae candidates if I query only on objects that are bright and classified as a star?} \texttt{flux\textgreater10 AND CLASS\_STAR=1} (A1). Three participants had also used filtering as a way to pick out individual objects of interest to query with. For example, G2 set the filter as gene=9687 and explained that since ``this gene is regulated by the estrogen receptor, when we search for other genes that resemble this gene, we can find other genes that are potentially affected by the same factors.''
\par While filtering enabled users to narrow down to a selected data subset, dynamic class creation enabled users to compare relationships between multiple attributes and between subgroups of data. For example, M2 divided solvents in the database to eight different categories based on voltage properties, state of matter, and viscosity levels, by dynamically setting the cutoff values on the quantitative variables to create these classes. By exploring these custom classes, M2 learned that the relationship between viscosity and lithium solvation energy is independent of whether a solvent belongs to the class of high voltage or low voltage solvents and cited that dynamic class creation was central to learning about this previously-unknown attribute properties:
\begin{quote}
All this is really possible because of dynamic class creation, so this allows you to bucket your intuition and put that together. [...] I can now bucket things as high voltage stable, liquid stable, viscous, or not viscous and start doing this classification quickly and start to explore trends. [...] And look how quickly we can do it! Quite good!
\end{quote}
Context creation enables users to change the lens in which they look through when preforming visual querying, thereby creating more opportunities to see the queried data from different perspectives.
%Context creation is a useful ---- despite the --- pattern instance. Filtering still useful
%\par Participants employed \emph{a mix of bottom-up and top-down approaches when faceting through data in VQS}, including narrowing the search space based on some intuition about a phenomena, selecting individual visualizations, or specifying high-level groupings to compare and query with.
LIMITATIONS
