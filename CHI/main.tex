%\documentclass[sigchi, review]{acmart}
\PassOptionsToPackage{table}{xcolor}
\documentclass[sigchi]{acmart}
\settopmatter{printacmref=false} % Removes citation information below abstract
\renewcommand\footnotetextcopyrightpermission[1]{} % removes footnote with conference information in first column
\pagestyle{plain} % removes running headers
% Load basic packages
\usepackage{balance}  % to better equalize the last page
\usepackage{graphics} % for EPS, load graphicx instead
\usepackage[T1]{fontenc}
\usepackage{txfonts}
\usepackage{mathptmx}
\usepackage[htt]{hyphenat}
% \usepackage[pdftex]{hyperref}

% \usepackage{booktabs}
% \usepackage{textcomp}
\usepackage{xspace}
\usepackage{setspace}
\usepackage{graphicx}
\usepackage{caption}
\usepackage[textsize=tiny]{todonotes}
% Some optional stuff you might like/need.
\usepackage{microtype} % Improved Tracking and Kerning
% \usepackage[all]{hypcap}  % Fixes bug in hyperref caption linking
\usepackage{ccicons}  % Cite your images correctly!
% \usepackage[utf8]{inputenc} % for a UTF8 editor only
\usepackage{verbatim}
\usepackage{relsize}
\usepackage{etoolbox}
\usepackage{lipsum}   % for filler text
\usepackage{setspace} % for \onehalfspacing and \singlespacing macros
\usepackage[normalem]{ulem}
\usepackage{enumitem}
\usepackage{relsize,etoolbox}% http://ctan.org/pkg/{relsize,etoolbox}
\usepackage{makecell}
\renewcommand\theadalign{bc}
\renewcommand\theadgape{\Gape[4pt]}
\renewcommand\cellgape{\Gape[4pt]}
\AtBeginEnvironment{quote}{\small}% Step font down one size relative to current font.
\setcopyright{none}

% DOI
\acmDOI{10.475/123_4}

% ISBN
\acmISBN{123-4567-24-567/08/06}

%Conference
\acmConference[CHI]{ACM CHI}{2019}{}
\acmYear{2019}
\copyrightyear{2019}

\acmPrice{15.00}

\def\plaintitle{Accelerating Scientific Data Exploration \\ via Visual Query Systems}
\def\emptyauthor{}
\def\plainkeywords{Data visualization, exploratory data analysis, visual query, scientific data.}
\def\plaingeneralterms{Documentation, Standardization}

% llt: Define a global style for URLs, rather that the default one
\makeatletter
\def\url@leostyle{%
  \@ifundefined{selectfont}{
    \def\UrlFont{\sf}
  }{
    \def\UrlFont{\small\bf\ttfamily}
  }}
\makeatother

\newenvironment{denselist}{
    \begin{list}{\small{$\bullet$}}%
    {\setlength{\itemsep}{0ex} \setlength{\topsep}{0ex}
    \setlength{\parsep}{0pt} \setlength{\itemindent}{0pt}
    \setlength{\leftmargin}{1.5em}
    \setlength{\partopsep}{0pt}}}%
    {\end{list}}

\newcommand{\squishlist}{
   \begin{list}{$\bullet$}
    { \setlength{\itemsep}{0pt}
      \setlength{\parsep}{2pt}
      \setlength{\topsep}{0pt}
      \setlength{\partopsep}{0pt}
      \leftmargin=25pt
\rightmargin=0pt
\labelsep=5pt
\labelwidth=10pt
\itemindent=0pt
\listparindent=0pt
\itemsep=\parsep
    }
}
\newcommand*{\img}[1]{%
    \raisebox{-.3\baselineskip}{%
        \includegraphics[
        height=\baselineskip,
        width=\baselineskip,
        keepaspectratio,
        ]{#1}%
    }%
}
\newcommand{\squishend}{\end{list}}
\newcommand{\npar}{\par\noindent}
% use extensively to toggle between paper and TR
\newcommand{\eat}[1]{}
% \newcommand{\papertext}[1]{{\leavevmode\color{blue}{#1}}}
% \newcommand{\techreport}[1]{{\leavevmode\color{red}{#1}}}
\newcommand{\papertext}[1]{#1}
\newcommand{\techreport}[1]{}
\newcommand{\nonannon}[1]{#1}
\newcommand{\annon}[1]{}
\newcommand{\boldpara}[1]{\par\noindent\textbf{#1}}
% de-facto paragraph format
\newcommand{\stitle}[1]{\noindent\textbf{#1}}
\newcommand{\tvcg}[1]{#1}
\newcommand{\cut}[1]{{\leavevmode\color{lightgray}{#1}}}
\newcommand{\ccut}[1]{} %confirmed cut

\urlstyle{leo}

% To make various LaTeX processors do the right thing with page size.
\def\pprw{8.5in}
\def\pprh{11in}
\special{papersize=\pprw,\pprh}
\setlength{\paperwidth}{\pprw}
\setlength{\paperheight}{\pprh}
\setlength{\pdfpagewidth}{\pprw}
\setlength{\pdfpageheight}{\pprh}

% create a shortcut to typeset table headings
% \newcommand\tabhead[1]{\small\textbf{#1}}
\newcommand{\zv}{\textit{zenvisage}\xspace}
\newcommand{\astro}{\textit{astro}\xspace}
\newcommand{\bio}{\textit{genetics}\xspace}
\newcommand{\matsci}{\textit{matsci}\xspace}

\newcommand{\agp}[1]{\textcolor{teal}{Aditya: #1}}
\newcommand{\kk}[1]{\textcolor{red}{Karrie: #1}}
\newcommand{\dor}[1]{\textcolor{violet}{Doris: #1}}

\renewenvironment{quote}{%
   \list{}{%
     \leftmargin0.15cm   
     \rightmargin\leftmargin
   }
   \item\relax
}
{\endlist}
%%%%%%%%%%%%%%%%%%%%%%%%%%%%%%%%%%%%%%%%%%%%%%%%%%%%%%%%%%%%%%%%
%%%%%%%%%%%%%%%%%%%%%% START OF THE PAPER %%%%%%%%%%%%%%%%%%%%%%
%%%%%%%%%%%%%%%%%%%%%%%%%%%%%%%%%%%%%%%%%%%%%%%%%%%%%%%%%%%%%%%%%

\begin{document}
% \title{Understanding usage patterns for query specification in sketch-based visual query systems: A Case Study with Zenvisage}
\title{How are visual query systems used in practice: A Design Study with Zenvisage}

\begin{abstract}
The increasing availability of rich and complex data in a variety of scientific domains poses a pressing need for tools to enable scientists to rapidly make sense of and gather insights from data. One proposed solution is to design visual query systems (VQSs) that allow scientists to interactively search for desired patterns in their datasets. While many existing VQSs promise to accelerate exploratory data analysis by facilitating this search, they are not widely used in practice. Through a year-long collaboration with scientists in three distinct domains---astronomy, genetics, and material science---we study the impact of various features within VQSs that can aid rapid visual data analysis, and how VQSs fit into scientists' analysis workflow. Our findings offer design guidelines for improving the usability and adoption of next-generation VQSs, paving the way for VQSs to be applied to a variety of scientific domains.
\end{abstract}
\keywords{Visual analytics, visualization, exploratory data analysis, visual query, scientific data.}
% Note that keywords are not normally used for peerreview papers.
% \begin{teaserfigure}
% % \begin{figure}[h!]
%     \centering
%     \includegraphics[width=0.6\linewidth]{figures/search-browse-model.png}
%     \vspace{-6pt}\caption{Search Browse Model}
%     \label{sbmodel}
%     \vspace{-5pt}
% % \end{figure}
% \end{teaserfigure}
\maketitle

%!TEX root=main.tex
 \vspace{-5pt}
 \section{Introduction\label{sec:intro}}
 % one for each key finding: a) many features deemed to be of importance to VQSs by domain experts, not all supported by present-day VQSs b) sketch is inefficient, perhaps explaining why present-day VQSs are not popular c) identify 3 typical workflows involving various sensemaking modalities in different proportions, depending on the application
 %To discover patterns of interest, analysts construct line chart visualizations \cut{using toolkits like \texttt{ggplot} or \texttt{matplotlib}, or visualization construction interfaces like Excel or Tableau,} \change{by} specifying {\em exactly} what they want to visualize. For example, when trying to find celestial objects corresponding to supernovae, which have a specific pattern of brightness over time, astronomers individually inspect the corresponding line chart for each object---numbering in the hundreds---until they find ones that match the pattern.
 Line charts are commonly employed during data exploration---the intuitive connected patterns often illustrate complex underlying processes
 and yield interpretable and visually compelling data-driven narratives~\cite{Few2012}. 
%To discover patterns of interest, analysts often have to construct and inspect thousands of line chart visualizations manually to find ones that match their desired pattern.
 \achange{However, discovering line charts that display certain meaningful patterns, trends, or characteristics of interest is often  an overwhelming and error-prone process, consisting of manual examination of large numbers of line charts. For example, when trying to find supernovae, which exhibits a unique pattern of brightness over time (an initial peak followed by a long-tail decay), astronomers often have to construct and inspect thousands of line chart visualizations manually to find ones that match their desired pattern.}
 %\techreport{For example, when trying to find celestial objects corresponding to supernovae, which have a specific pattern of brightness over time, astronomers individually inspect the corresponding line chart for each object---numbering in the hundreds---until they find ones that match the pattern.}\ccut{Similarly, when trying to infer relationships between two physical properties for different subsets of battery electrolytes, scientists need to individually visualize these properties for each subset (out of an unbounded number of such subsets) until they identify relationships that make sense to them.} 
 %This process of manual exploration of large numbers of line charts \change{for pattern identification} is not only error-prone, but also overwhelming for analysts. 
 To address this \change{exploration} challenge, there \change{has} been a large number of papers dedicated to building \emph{Visual Query Systems} (VQSs)\change{---the term coined by Ryall et al.~\cite{ryall2005querylines} to describe} systems that allow users to specify and search for desired \change{line chart patterns via visual} interfaces~\cite{mohebbi2011google,Hochheiser2004,wattenberg2001sketching,Siddiqui2017VLDB,ryall2005querylines,correll2016semantics,Mannino2018,Eichmann2015,Holz2009}. %visual patterns via interactive 
 These interfaces typically include a sketching canvas where users can draw a visual pattern of interest, with the system automatically traversing all potential visualization candidates to find those that match the specification. 
 % \par While this intuitive specification interface appears to be a promising solution
 \par While these intuitive specification interfaces were proposed as a promising solution to the problem of painful manual exploration of visualizations \change{for time-series analysis}~\cite{ryall2005querylines,wattenberg2001sketching}, to the best of our knowledge, VQSs have not lived up to these expectations and are not very commonly used in practice. \achange{One likely reason for the lack of VQS adoption may be attributed to how prior work have focused almost solely on optimizing for better pattern-matching algorithms and interactions, with few that invested in understanding actual user needs and how VQSs can be used for solving real-world problems.} {\em Our paper seeks to understand how VQSs can actually be used in practice, as a first step towards the broad adoption of VQSs in data analysis}. Unlike prior work on VQSs, we set out to not only evaluate VQSs in-situ on real problem domains, but also involve participants from these domains in the VQS design. We present findings from a series of interviews, contextual inquiry, participatory design, and user studies with scientists from three different domains---{\em astronomy, genetics,} and {\em material science}---over the course of
 a year-long collaboration. \change{The amount of time we invested in each of these three diverse domains surpasses the norm in this field and is key to uncovering the insights presented in this paper.} As illustrated in Figure~\ref{science_goal}, these domains were selected to capture a diverse set of goals and datasets wherein VQSs can help address important scientific questions, such as: How does a treatment affect the expression of a gene in a breast cancer cell-line? Which battery components have sustainable levels of energy-efficiency and are safe and
 cheap to manufacture in production?
 \begin{figure}[ht!]
 	\centering
 	\includegraphics[width=\linewidth]{figures/science_goal.pdf}
 	\caption{Desired insights, problem and dataset challenges for each of the three application domains in our study.}
 	\label{science_goal}
 	\vspace*{-15pt}
 \end{figure}
 \par Via contextual inquiry and interviews, we first identified challenges in existing data analysis workflows in these domains
 that could be potentially addressed by a VQS. Building on top of an existing open-source VQS, \zv~\cite{Siddiqui2017,Siddiqui2017VLDB}, we engaged participants in a process of participatory design (PD)~\cite{Muller1993,BodkerGronbaek,HoltzblattJones} to enable them to better compose data exploration workflows that lead to insight discovery, over the course of a year. \change{Rather than targeting a domain-specific solution, we chose to perform participatory design across multiple domains (an uncommon practice in visualization design studies) to observe differences and commonalities across domains to synthesize \change{high-level} insights regarding the use of VQSs.} \achange{While designing and performing this multi-phased, mixed-methods research agenda across three different use cases was an ambitious feat, this endeavor was necessary for addressing the qualitative, participant-centered research questions investigated in this work.}
 \par We organize our PD findings into a taxonomy of VQS capabilities, involving three sensemaking processes inspired by Pirolli and Card's notional model of analyst sensemaking~\cite{Pirolli}. The sensemaking processes include \emph{top-down pattern search} (translating a pattern ``in-the-head'' into a visual query), \emph{bottom-up data-driven inquiries} (querying or recommending based on data), and \emph{context-creation} (navigating across different collections of visualizations). We find that prior VQSs have focused on enabling top-down processes \change{(via sketching \achange{capabilities})}, \achange{but have largely overlooked the two other processes that we found to be essential in all three domains. These missing \achange{capabilities} partially explains why \achange{prior VQSs} have not been widely adopted in practice}.
% other two processes that we found to be crucial for all three domains.
 %to gather feedback and iterate on VQS feature designs, culminating in a new enhanced VQS, \zvpp.
 \par To study how various VQSs are used in practice, we conducted a final evaluation study with nine participants using our final VQS prototype to address their research questions on their own datasets. During this 1.5-hour study, participants \achange{gained} novel scientific insights,
 such as identifying a star with a transient pattern that was known to harbor a Jupiter-sized planet\achange{, discovering a previously-unknown relationship between solvent properties,} and finding characteristic gene expression profiles confirming the results of a related publication. %\techreport{Participants also gain additional insights about their datasets, including debugging mislabeled features and uncovering erroneous data preprocessing procedure applied to a collaborator's dataset.}%\techreport{, and discovering that the dip in an astronomical light curve is caused by saturated imaging equipment overlooked by the existing error-detection pipeline.}
 %\agp{Explain why these findings are important.}\dor{I think saying that planetary discovery is related to future colonization is a bit too much here and significance of characteristic gene expression profiles. Also we already described the significance of each domain earlier with the `important scientific questions' part.}
 %that goes from a pattern in-the-head to a desired visualization
 \par By analyzing the evaluation study results, we \achange{were somewhat surprised to discover} that sketching a pattern for querying is often ineffective on its own. This is due to the fact that sketching makes the problematic assumption that users know the pattern that they want to sketch and are able to sketch it precisely.\achange{ However, this is not the case in practice. For example, the geneticists from our study often did not have a preconceived knowledge of what to sketch and search for and relied heavily on recommended common patterns and outlying ones provided by the VQS to jumpstart their queries. Likewise, while the material scientists from our study were interested in datapoints that fall within specific value-ranges, they did not have an apriori notion of what these desired patterns would look like. Overall,} participants typically opted to combine sketching with other means of pattern specification---one common mechanism was to drag-and-drop a recommended pattern onto the canvas, and then modify it (e.g., by smoothing it out). %\cut{However, most VQSs do not support these other mechanisms (as we argued earlier, they typically focus only on top-down sensemaking processes, without covering bottom-up and context creation)}\dor{cutting this out since already mentioned 2 paragraphs ago}.
 %Participants were, however, able to apply the two other sensemaking processes to gain novel scientific insights, such as identifying a star with a transient pattern that was known to harbor a Jupiter-sized planet, finding characteristic gene expression profiles confirming the results of a related publication, and discovering mislabelled features from a data preprocessing mistake.
 % Further analysis of how participants transition between different sensemaking processes during analysis using a Markov model illustrated
 \par \change{To further understand how participants engaged with VQSs in their analytical workflows, we used a Markov model to \achange{characterize} how participants transitioned between different sensemaking processes during their analysis. \achange{We found that} participants often constructed a diverse set of \achange{analytical} workflows tailored to their domains by focusing} around a primary sensemaking process, while iteratively interleaving their analysis with the two other processes. This finding points to how all three sensemaking processes, along with seamless transitions between them, are \achange{crucial} for enabling \achange{the effective use and adoption of VQSs for addressing real-world challenges.}%for data exploration.%For example, participants often center on a main sensemaking process, while interleaving variations with other two processes as they iterate on an analytic task.
 %---including the construction of a Markov model---
 \par To the best of our knowledge, our study is the \emph{first to holistically examine how VQSs can be designed to fit the needs of real-world, analysts and how they are actually used in practice}. Working with participants from multiple domains enabled us to compare the differences and commonalities across different domains, thereby identifying general VQS challenges and requirements for supporting common analytical goals. Our contributions include:
 \begin{denselist}
 \item a characterization of the problems addressable by VQSs through design studies with three different domains,
 \item the construction of a taxonomy of essential VQSs capabilities leading to a sensemaking model for VQSs, grounded in participatory design findings, %, as well as an articulation of the problem space that is amenable to VQSs
 \item an integrative VQS, \zvpp, post participatory design, capable of facilitating rapid hypothesis generation and insight discovery,
 \item study findings on how VQSs are used in practice, leading to the development of a novel sensemaking model for VQSs. %including the ineffectiveness of
 %evaluation
 % sketching and the ---- workflow
 \end{denselist}
 Our work not only opens up a new space of opportunities beyond the narrow use cases considered by prior studies, but also advocates common design guidelines and end-user considerations for building next-generation VQSs.
%!TEX root = main.tex
\section{Methods\label{sec:methods}}
\subsection{Background and Motivation}
\par Visual query systems enable users to directly search for visualizations matching certain patterns through an intuitive specification interface. Early work in this space focused on interfaces to search for time series with specific patterns, including TimeSearcher~\cite{Hochheiser2001,Hochheiser2004}, where the query specification mechanism is a rectangular box, filtering out all of the time series that does not pass through it, QuerySketch~\cite{wattenberg2001sketching} and Google Correlate~\cite{mohebbi2011google}, where the query is sketched as a pattern on canvas, filtering out all of the time series that have a different shape. Subsequent work recognized the ambiguity in sketching by studying how humans rank the similarity in patterns~\cite{Eichmann2015,correll2016semantics,Mannino2018} and improving the expressiveness of sketched queries through finer-grained specification interfaces and pattern-matching algorithms~\cite{ryall2005querylines,Holz2009}.
%performed crowdsourced perceptual studies to understand how humans rank similarity in patterns subjectively
% , including the use of soft constraints~\cite{ryall2005querylines} and implicit relaxed selection techniques~\cite{Holz2009}.
% In addition to this ongoing work, recent work have also performed crowdsourced perceptual studies to understand how humans rank similarity in patterns subjectively~\cite{Eichmann2015,correll2016semantics,Mannino2018}.
\par While these systems have been effective in controlled lab studies, they have never been designed and evaluated in-situ on real-world use cases. Even when use cases were involved~\cite{Hochheiser2004,correll2016semantics}, the inclusion of these use cases had a narrow objective and had little influence on the major design decisions of the system. In the context of Munzner's nested model~\cite{munzner2009nested}, this represents the common ``downstream threat'' of jumping prematurely into the deep levels of \textit{encoding, interaction, or algorithm design}, before a proper \textit{domain problem characterization} and \textit{data/operation abstraction design}. \change{In this work, we performed design studies~\cite{lam2012empirical,shneiderman2006strategies,Sedlmair2012} with three different subject areas for \textit{domain problem characterization}. Comparing and contrasting between the diverse set of questions, datasets, and challenges across these three use cases revealed new generalizable insights and enabled us to better understand how VQSs can be extended for novel and unforeseen use cases.} Based on these findings, we develop a feature taxonomy for understanding the sensemaking process in VQSs as part of the \textit{data/operation abstraction design}. Finally, we validated the abstraction design with grounded evaluation~\cite{Plaisant2004,Isenberg2008}, where we invite participants to bring in their own datasets and research problems that they have a vested interest in to test our final deployed system. \change{Next, we will describe these two phases of our study in more detail.}
%  on our final system ----
% , drawing from ---- grounded evaluation
% validated with usage of deployed system, target users , 
% grounded evaluation
% evaluated on 
% participatory design with multiple case studies to ----- target domain, while 
% opted for a qualitative mixed-methods approach drawing from ethnographic methods, participatory design, and grounded evaluation~\cite{Plaisant2004,lam2012empirical,shneiderman2006strategies,Sedlmair2012,Isenberg2008} to more thoroughly characterize the problem design space of VQSs and taxonomy abstraction for understanding the sensemaking process in VQSs. Moreover, we performed design studies with three different subject areas with a diverse set of questions, datasets, and challenges to further generalize our findings.
%Most of these systems have not been evaluated in-situ on real-world use cases. Even when design study was performed~\cite{correll2016semantics,Hochheiser2001}, the focus these  narrow use case in mind, and these --- never adopted in practice,either ---did not --- didn't influence the design decisions made to --- . We found there is a genuine need in the community to more thoroughly understand the design space of VQSs and how various components of VQSs are used in practice. In this work, We make use of multiple case studies ....
%\par While these systems have been shown to be effective for visual querying in controlled lab studies, they have not been evaluated in-situ on real-world use cases. 
%In this work, we adopted a mixed methods research methodology that draws inspiration from ethnographic methods, iterative and participatory design, and controlled studies~\cite{Plaisant2004,lam2012empirical,shneiderman2006strategies,Muller1993} Participatory design has been successfully used in the development of interactive visualization systems in the past~\cite{Aragon2008,Chuang2012}. Sedlmair et al. \cite{Sedlmair2012} advocate that design study methodology is suitable for use cases in which the data is available for prototyping, but the task is only partially known and the information is partially in the user's head. 
%to understand how VQSs can be used in scientific data analysis. %In that regard, our scientific use cases with VQS is well-suited for a design study methodology, as we learn about the scientist's data and analysis requirements and design interactions that helps users translate their ``in-the-head'' specifications into actionable visual queries.
\vspace{-10pt}
\begin{table} 
    \includegraphics[width=0.8\linewidth]{figures/related_works_table.pdf}
    \caption{Table summarizing whether key functionalities of VQSs (columns) are covered by past systems (row), indicated by checked cells. Column header colors blue, orange, green represents three sensemaking process (top-down querying, search with context, and bottom-up querying) described in Section~\ref{sec:pd_findings}. The heavily-used, practical features in our study for context-creation and bottom-up inquiry is largely missing from prior VQSs.}
    \label{table:relatedwork}
    \vspace{-29pt}
\end{table}
\subsection{Phase I: Participatory Design}
\par We recruited participants by reaching out to research groups via email and word of mouth, who have experienced challenges in data exploration. Based on our early conversations with analysts from 12 different potential application areas, we narrowed down to three use cases in astronomy, genetics, and material science for our participatory design study, chosen based on their suitability for VQSs as well as diversity in use cases. Six scientists from three research groups participated in the design of \zv. On average, participants had more than 6 years of research experience working in their respective fields. Via interviews and cognitive walkthroughs with researchers from the three different scientific research groups, we identified the needs and challenges of these use cases. 
\par For the participatory design study, we built on an existing VQS, \zv~\cite{Siddiqui2017,Siddiqui2017VLDB}, that allowed users to sketch a pattern or drag-and-drop an existing visualization as a query, with the system returning visualizations that had the closest Euclidean distance from the queried pattern. We chose to build on top of \zv, since it was open-source, extensible, and encompassed a large selection of features compared to existing systems, which focused largely on features for pattern and match specification (as compared in Table~\ref{table:relatedwork}).
%The details of the system is described in \cite{Siddiqui2017,Siddiqui2017VLDB}, which focused on the system and scalability aspects of the VQSs. 

 %Table~\ref{table:relatedwork} summarizes the list of features offered by these existing systems.
%  that ---
% identified potential opportunities for VQSs 
% and potential opportunities for VQSs
% We initially spoke to analysts from 12 different potential application areas and narrowed down to three use cases in astronomy, genetics, and material science for our participatory design study, based on their suitability for VQS as well as diversity in use cases. Six scientists from three research groups participated in the design of \zv. On average, participants had more than 8 years of research experience working in their respective fields. %\techreport{We list the participants in Table~\ref{participants}, and will refer to them by their anonymized ID as listed in the table throughout the paper.}
% \par Given our early conversations with participants, we built a basic VQS to serve as the functional prototype in the design study. This early VQS prototype allowed users to sketch a pattern or drag-and-drop an existing visualization as a query, then the system would return visualizations that had the closest Euclidean distance from the queried pattern. The details of the system is described in \cite{Siddiqui2017,Siddiqui2017VLDB}, which focused on the system and scalability aspects of the VQSs.
% 	% \begin{figure}[ht!]
% 	% \centering
% 	% \includegraphics[width=\linewidth]{figures/oldZV_nozql.pdf}
% 	% \caption{The \zv prototype allowed users to sketch a pattern in (a), which would then return (b) results that had the closest Euclidean distance from the sketched pattern. The system also displays (c) representative patterns obtained through K-Means clustering and (d) outlier patterns to help the users gain an overview of the dataset.}
% 	% \label{oldZV}
% 	% \end{figure}
% \par The use of functional prototypes is common and effective in participatory design to provide a starting point for the participants, as studied by Ciolfi et al.\cite{Ciolfi2016}. %For example, Ciolfi et al.\cite{Ciolfi2016} studied two different alternatives to co-design (starting with open brief versus functional prototype) in the development of museum guidance systems and found that while both approaches were equally fruitful, functional prototypes can make addressing a specific challenge more immediate and focused.
% Our motivation for providing a functional prototype at the beginning of the participatory design sessions is to showcase capabilities of VQSs. Especially since VQSs are not common in the existing workflows of these scientists, participants may not be able to imagine their use cases without a starting point.
\par During participatory design, we collaborated with each team closely with an average of two meetings per month, where we learned about their datasets, objectives, and how VQSs could help address their research questions. A summary timeline of our collaboration with participants over a year and features inspired by their use cases can be found in Figure \ref{timeline}. 
%Participants provided datasets they were exploring from their domain, whereby they had a vested interest in using a VQS to address their own research questions. 
Through this process, we identified and incorporated more than 20 desired features into \change{the new version of our VQS, \zvpp, described more in Section~\ref{sec:pd_findings}.}
\begin{figure*}[ht!]
	\centering
	\captionsetup{justification=centering,margin=2cm}
	\vspace{-10pt}
	\includegraphics[width=6in]{figures/timeline_anon.pdf}
	\vspace{-6pt}\caption{Timeline for progress in participatory design studies.}
	\label{timeline}
	\vspace{-10pt}
\end{figure*}
\vspace{-10pt}
\subsection{Phase II: Evaluation Study}
% \par Visualization systems are often evaluated using controlled studies that measure the user's performance against an existing visualization baseline~\cite{Plaisant2004}. Techniques such as artificially inserting ``insights'' or setting predefined tasks for example datasets work well for objective tasks, \techreport{such as debugging data errors~\cite{kandel2011wrangler,Patel2010},} but they are unsuitable for trying to learn about the types of real-world queries users may want to pose on VQSs. %Due to the unrealistic nature of controlled studies, many have proposed using a more multi-faceted, ethnographic approach to understand how analysts perform visual data analysis and reasoning~\cite{Plaisant2004,lam2012empirical,shneiderman2006strategies,munzner2009nested,Sedlmair2012}.
At the end of our participatory design study, we performed a qualitative evaluation to study how analysts interact with different VQS components in practice. In order to make the evaluation more realistic, we invited participants to use datasets that they have a vested interest in exploring to address unanswered research questions. \change{As shown in Table~\ref{participants},} the evaluation study participants included the six scientists from participatory design, along with three additional ``blank-slate'' participants who had never encountered \zvpp before. While participatory design subjects actively provided feedback on \zvpp with their data, they only saw us demonstrating their requested features and explaining the system to them, rather than actively using the system on their own. So the evaluation study was the first time that all participants used \zvpp to explore their datasets.
\par \change{Evaluation study participants} were recruited from each of the three aforementioned research groups, as well as domain-specific mailing lists. Prior to the study, we asked potential participants to fill out a pre-study survey to determine eligibility. Eligibility criteria included: being an active researcher in the subject area with more than one year of experience, and having worked on a research project involving data of the same nature used in participatory design. The nine participants brought a total of six different datasets to the study. \techreport{The research questions and objectives of the participants were diverse even among the same subject area. Examples included understanding gene expression profiles of breast cancer cells after a particular treatment and comparing common patterns among stars that exhibit planetary transits versus stars that do not.\techreport{from the Kepler space telescope\footnote{\url{www.nasa.gov/mission_pages/kepler/main/index.html}}.}} 
% Four of the evaluation studies were conducted remotely. Participants had the option of exploring their own dataset or an existing dataset that they provided to us during the participatory design process. All three blank-slate participants opted to explore their own datasets.
 %After loading their dataset, we emailed them a screenshot of a visualization from our tool to verify that we configured the system to meet their needs.
\par At the start, participants were provided with an interactive walk-through explaining the system details and given approximately ten minutes for a guided exploration of \zvpp with a preloaded real-estate example dataset from Zillow \cite{zillow}.\techreport{This dataset contained housing data for various cities, metropolitan areas, and states in the U.S. from 2004-15.} After familiarizing themselves with the tool, we loaded the participant's dataset and encouraged them to talk-aloud during data exploration and use external tools. If the participant was out of ideas\ccut{ for three minutes}, we suggested one of the ten main VQS functionalities \techreport{\footnote{query by sketching, drag-and-drop, pattern loading, input equations, representative and outliers, narrow/ignore x-range options, filtering, data smoothing, creating dynamic classes,  data export}}that they had not yet used. If any of these operations were not applicable to their specific dataset, they were allowed to skip the operation after having considered how it may or may not be applicable to their workflow. The user study ended after they covered all ten main functionalities. On average, data exploration lasted for 63 minutes. After the study, we asked them open-ended questions about their experience.% and suggested an appropriate choice of axis to begin the exploration.
%\par During the exploration phase, participants were informed that they could use other tools as needed.
\begin{table}[h!]
\centering
\vspace{-10pt}
\includegraphics[width=\linewidth]{figures/participant_info.pdf}
\caption{Participant information. The Likert scale used for dataset familiarity ranges from 1 (not at all familiar) to 5 (extremely familiar).}
\label{participants}
\vspace{-15pt}
\end{table}
%!TEX root = main.tex
\section{Participants and Datasets\label{sec:participantdatasets}}
At the start of our design study, we observed participants as they conducted cognitive walkthroughs demonstrating their existing data analysis workflows. Next, we describe our study participants and their preferred analysis workflows.%use cases to highlight behaviors that participants have adopted for conducting certain analysis tasks.
\par\noindent\stitle{Astronomy:} The Dark Energy Survey is a multi-institution project that surveys 300 million galaxies over 525 nights to study dark energy~\cite{Drlica-Wagner2017}. The telescope \change{used to survey these galaxies} also focuses on smaller patches of the sky on a weekly interval to discover astrophysical transients (objects whose brightness changes dramatically as a function of time), such as supernovae or quasars. Their dataset consists of a large collection of brightness observations over time\change{, one} associated with each astrophysical object, called a {\em light curve}, and plotted as a time series. For over five months, we worked closely with A1, an astronomer on the project's data management team working at a supercomputing facility. Their scientific goal is to identify potential astrophysical transients in order to study their properties. \techreport{These insights can help further constrain physical models regarding the formation of these objects.}
\par To identify transients, astronomers programmatically generate visualizations of candidate objects with \texttt{matplotlib} and visually examine each light curve. While an experienced astronomer who has examined many transient light curves can often distinguish an interesting transient object from noise by sight, manual searching for transients is time-consuming and error prone\change{, since} the large majority of the objects are false positives. A1 was interested in VQSs as he recognized how specific pattern queries could help astronomers directly search for these rare transients.
\techreport{\par If an object of interest or region is identified through the visual analysis, then the astronomer may be interested in inspecting the image of the region for cross-checking that the significant change in brightness of the object is not due to an imaging artifact. This could be done using a custom built web-interface that facilitates the access of cutout images for a queried region of the sky.}
\par\noindent\stitle{Genetics:} Gene expression is a common measurement in genetics obtained via microarray experiments~\cite{Gloss2017}. \techreport{In these experiments, a grid containing thousands of DNA fragments are exposed to stimuli and measurements for the level at which a gene is expressed are recorded as a function of time.} We worked with a graduate student (G1) and professor (G3) at a research university who were using gene expression data to understand how genes are related to phenotypes expressed during early development\techreport{\cite{Peng2016,Gloss2017}}. Their data consisted of a collection of gene expression profiles over time for mouse stem cells, aggregated over multiple experiments.\techreport{, downloaded from an online database\footnote{\url{ncbi.nlm.nih.gov/geo/}}.} %They were interested in using \zv to cluster gene expression data before conducting analysis with a downstream machine learning workflow.
\par Their typical workflow is as follows: G1 first loads the preprocessed gene expression data into a custom desktop application for visualizing and clustering it\techreport{\footnote{\url{www.cs.cmu.edu/~jernst/stem/}}}. After setting several system parameters and executing the clustering algorithm, the overlaid time series for each cluster is displayed on the interface. G1 visually inspects that the patterns in each cluster looks ``clean'' and checks that the number of outlier genes (i.e., those that do not fall into any of the clusters) is low.  If the number of outliers is high or the clustered visualizations look ``unclean'', she reruns the analysis by increasing the number of clusters. When the visualized clusters look ``good enough'', G1 exports the cluster patterns to her downstream regression tasks.
\par Prior to the study, G1 and G3 spent over a month attempting to determine the best number of clusters based on a series of static visualizations and statistics computed after clustering. While regenerating their results took no more than 15 minutes every time they made a change, the multi-step, segmented workflow meant that all changes had to be done offline.\techreport{, so that valuable meeting time was not wasted trying to regenerate results.} The team were interested in VQSs as they saw how interactively querying time series with clustering results could dramatically speed up their collaborative analysis process.
%that can improve battery performance and stability
\par\noindent\stitle{Material Science:} We collaborated with material scientists at a research university who are working to identify solvents for energy efficient and safe batteries. These scientists work on a large simulation dataset containing chemical properties for more than 280,000 solvents. Each row of their dataset represents a unique solvent with 25 different chemical attributes. We worked closely with a a postdoctoral researcher (M1), professor (M2), and graduate student (M3) for over a year to design a sensible way of exploring their data. They wanted to use VQSs to identify solvents that not only have similar properties to known solvents but are also more favorable (e.g., cheaper or safer to manufacture). To search for these desired solvents, they need to understand how changes in certain chemical attributes affect other properties under specific conditions.
\par M1 typically starts his data exploration process by iteratively applying filters to a list of potential battery solvents using SQL queries. When the remaining list of the solvents is sufficiently small, he examines each solvent in more detail to factor in the cost and availability to determine experimental feasibility. The scientists were interested in VQSs as it was impossible for them to uncover hidden relationships between different attributes across large number of solvents manually.%(such as how changing one attribute affects another attribute)
%!TEX root = main.tex
\section{Participatory Design Findings}
From participatory design, we learned about the characteristic problems and challenges present to VQSs. We first describe the features that we have developed to addresses these challenges, thematically organized by components. %Based on feature requests and discussion with our participants, we incorporated key features missing in our original VQS.
%From these discussion and analysis of past VQSs, we identify nine components of VQSs, described below. T
Along with analysis of past literature, we develop a taxonomy of key functionalities in VQSs. These components are then organized into three paradigms of sensemaking in VQSs that span across different areas in the design space.
% novel contribution on  ---
% contribute to holistic understanding on how sensemaking --- in VQS.
% study on how users
% Implication ---
% •	What types of questions/ dataset/ problem challenges are asked to VQS or can be addressed by VQS? (S3)
% •	What kind of features needs to be designed to address these challenges (S4 PD)
%We employed participatory design with our scientists to incorporate key features missing in our original VQS, and unaddressed in their existing workflows. From these discussion and analysis of past VQSs, we identify nine components of VQSs, described below.
\subsection{Themes Emerging from Participatory Design\label{sec:pd_findings}}
\begin{figure*}[t!]
\centering
\vspace{-15pt}
\includegraphics[width=\linewidth]{figures/system.pdf} %5.5
\vspace{-5pt}\caption{Our VQS after participatory design, which includes: the ability to query via (a) a sketch,(b) input equations, (i) drag and drop, or (j) uploaded patterns; (c) preprocessing via data smoothing; query specification mechanisms including (d) x-range selection and filtering, (e) x-range invariance, (g) Filtering, and (h) Dynamic class creation; recommendation of (k) representative and (l) outlier trends. Prior to the participatory design, \zv only included a single sketch input with no additional options.}
\label{zvOverview}
\vspace{-14pt}
\end{figure*}

 %We discovered three central themes encapsulating these features that are important to facilitate rapid hypothesis generation and insight discovery, but are missing in prior VQSs. While some of our findings echo prior work on system-level taxonomies of visualization tasks \cite{Amar2005,Heer2012}, we highlight how specific analytic tasks and interaction features could be used to enhance VQSs in particular. \techreport{In particular, we learned that \textit{participants wanted more control over the internals of the systems and an integrated workflow that helped streamline their analysis when using VQSs.}}
\boldpara{Exact Shape Specification} interfaces allow users to submit an exact description of a pattern query, then the VQS returns a list of most similar matches. Almost all VQS supports freehand sketching \techreport{on a virtual canvas through mouse or pen as a intuitive mechanism }for specifying desired patterns (Figure \ref{zvOverview}a). In addition to sketching, \zv also allows users to specify a functional form (e.g. y=$x^2$) for a pattern (Figure \ref{zvOverview}b). This feature was requested by material scientists who were interested in finding solvents with known analytical models describing relationships between their chemical properties.
\boldpara{Approximate Shape Specification:} While exact shape specification is an intuitive mechanism for constructing a visual query, as pointed out by past works~\cite{correll2016semantics,Holz2009}, pattern queries can be extremely imprecise. Many interfaces have developed constrained sketching mechanism to allow users to partially specify certain characteristics, such as angular slope queries\techreport{for specifying the slope of a trend line}~\cite{Hochheiser2004} or piecewise trend querylines\techreport{over a specified data range}~\cite{ryall2005querylines}. Both Qetch and \zv supports data smoothing to allow users to interactively change the degree of shape approximation they would like to apply to all visualizations (and consequently for pattern matching). Motivated by the dense and noisy observational data in the astronomy and material science use cases, we developed an interface for users to interactively adjust data smoothing algorithm and parameters on-the-fly to update the resulting visualizations accordingly (Figure \ref{zvOverview}c).
\techreport{\par During participatory design, both material science and astronomy participants noted the difficulty of shape matching on their dense and noisy observational data and the challenge of picking the appropriate smoothing parameters during offline preprocessing. We found that tight integration between smoothing and visual search additionally tradeoff between the smoothness of the curve and the degree of approximation for shape-matching in VQSs. An over-smoothed visualization would return shape matches that only loosely resemble the query pattern. However, without smoothing, the noise may dominate the overall trend, which could lead to bad pattern matches.}
%While the interactions in our original prototype enabled simple visual queries, many scientists were interested in extending their querying capabilities, either through different querying modalities or through more flexible query specification methods.

% While \zv does not attempt to solve all of the pre-processing issues that we faced during participatory design, we identified data smoothing as a common data cleaning procedure that could benefit from a tight integration between pre-processing and visual analysis. Data smoothing is a denoising procedure that generates a smoothed pattern approximating key features of the visualized trend with less noise.
\boldpara{Range Selection:} Often in time series analysis there are specific ranges of time and measure values with special domain specific significance that may be of interest to users. To find such patterns, users can limit the pattern query to be matched only in specific x or y ranges, specified through textboxes~\cite{wattenberg2001sketching,Mannino2018}, min/max line boundaries~\cite{ryall2005querylines}, or brushing interactions~\cite{Hochheiser2001}. \zv employs the brushing mechanism to select desirable x-ranges to perform shape matching (Figure \ref{zvOverview}d). Additionally, y axis range selection could be performed through entering a filter constraint on the measure variable.
\par We chose to support only brushing for x, since it was more common to focus the context based on the independent variable in our use cases, such as zooming into particular sharp dips when looking for planetary transits or anomalous peaks indicative of erroneous experimental measurements. \cut{In contrast, y-range selection tends to be more global and enforced across multiple interaction sequences, such as looking for only signals above a certain threshold.} The TimeSearcher and Queryline approach is most flexible as they allow composition of multiple range selections to formulate complex piecewise queries, such as finding gene expression profiles rising from x=1-5 then declining from x=5-10.
\boldpara{Flexible Matching:} Studies have shown that \techreport{to facilitate subjectively meaningful pattern matches,} VQSs need to support mechanisms for clarifying sketch interpretation and flexible shape matching algorithms~\cite{correll2016semantics,Mannino2018,Eichmann2015}. In \zv, users have the option to change similarity metrics that perform flexible matching (Figure \ref{zvOverview}e). Similar to temporal invariants in SketchQuery, \zv also supports an option to ignore the x-range in shape matching (Figure \ref{zvOverview}f). For finding supernovae, A1 primarily cared about the existence of a peak\techreport{above a certain amplitude with an appropriate width of the curve}, rather than the exact time that the event occurred. G1 also expressed that she does not care about when the `trigger point' occurs as long as the profile is rising.
% This latter features is akin to the temporal invariants in SketchQuery.

\boldpara{Filter Selection:} Users with large datasets often need to first use domain knowledge to narrow down their search to a subset of data. This increases their chances of finding an interesting pattern for a given pattern query. To filter data on-the-fly in \zv, users could compose one or more conditions as filter constraints in a text field (Figure \ref{zvOverview}g). \cut{The filtering can be done on data columns associated with each pattern that is not visualized or on the visualized attributes. This feature is unique to \zv as most existing VQSs do not allow users to interact with data in the non-visualized columns.}
%We designed two dynamic faceting features coupled with coordinated views that enabled users to specify subsets of data they are querying on and see immediate changes updated in the query, representative, and outlier results.
\boldpara{Group Comparison} addresses a common analytical task where users want to bucket data points into customized classes based on existing properties, then compare between these classes. For example, M1 wanted to create classes of solvents with ionization potential under -10 kJ/mol, over -8 kJ/mol, and ones between that range. Then, he wanted to see how visualizations involving lithium solvation energy varied across the three classes. To this end, we implemented dynamic class creation, a feature that allows users to use multiple properties to create custom classes on-the-fly, effectively slicing-and-dicing the data based on their needs (Figure \ref{zvOverview}h). Information regarding the created classes is displayed in a table and as a tooltip over aggregate visualizations.
% , as shown in Figure~\ref{dcc}.
% \begin{figure}[h!]
% \centering
% \includegraphics[width=\linewidth]{figures/dcc_example.pdf}
% \vspace{-6pt}
% \caption{Example of dynamic classes. (a) Four different classes with different Lithium solvation energies (li) and boiling point (bp) attributes based on user-defined data ranges. (b) Users can hover over the visualizations for each dynamic class to see the corresponding attribute ranges for each class. The visualizations of dynamic classes are aggregate across all the visualizations that lie in that class based on the user-selected aggregation method.}
% \label{dcc}
% \vspace{-10pt}
% \end{figure}
%While input equations are useful when simple analytical models exist, this may not be true for other domains. In these scenarios, users can upload a query pattern of a sequence of points
\boldpara{Concept querying} enable users to upload a pattern associated with a concept as a query. As supported in \zv and Google Correlate, users can upload a sequence of points as the query pattern (Figure \ref{zvOverview}i). This is useful for patterns generated from computational models or prelabelled data from an external reference database. For example, A1 wanted to query based on synthetic light curves generated from simulations and known supernovae that have been discovered in the past. %Similarly, Google Correlate allows users to upload their own time series or enter search keywords that corresponds to a time series.
%, usually as part of the downstream analysis of the exploratory workflow. %For example, the genetics team are trying to develop a time series prediction algorithm using machine learning based on some biological parameters \cite{Peng2016}.
\boldpara{Result querying} allows users to submit a query based on the results, essentially asking for patterns that are similar to the selected data pattern. TimeSearcher enable users to instantiate queries via drag-and-drop, whereas QuerySketch does so through double clicking. Similarly in \zv, users can drag and drop a visualization in either the results pane or the representative and outliers to the query canvas (Figure \ref{zvOverview}j). \cut{The distinction between concept and result querying is that concept querying loads in data external to the dataset, whereas result querying initiates the query using visualizations created from the queried data source.}
\boldpara{Recommendation} displays visualizations that may be of interest to the users based on the data context. \zv provides visualizations of representative trends based on clustering and highlights outlier instances\techreport{ that look different from the rest of the visualizations} (Figure \ref{zvOverview}k,l).%The recommendation feature is unique to \zv, which provides visualizations of representative trends based on clustering and highlights outlier instances that looks different from the rest of the visualizations (Figure \ref{zvOverview}k,l).
% In this section, we first describe a model to help characterize the design space for VQS based on the analytical workload and usage patterns from different use cases. Then, we present design challenges related to each of the process.
\subsection{Characterizing Design Space for VQSs}
Based on example use cases and feature components from participatory design, we further characterize the design space of VQSs. Visual querying often consists of searching for a desired visualization instance (Z) across a visualization collection that consists of some attributes (X,Y). We introduce two axes depicting the amount of information known about the visualized attribute and pattern instance, as shown in Figure~\ref{2dmodel}.
\par Along the \textbf{pattern instance} axis, the visualization that contain the desired pattern may already be \texttt{known} to the analyst, exist as a pattern \texttt{in-the-head} of the analyst, or completely \texttt{unknown} to the analyst. In the \texttt{known} pattern instance region (Figure~\ref{2dmodel} grey), a user only be interested in patterns related to a specific gene. Such use cases would be more suited for a visualization-at-a-time system, where analyst manually create and examine each visualization one at a time, rather than a VQS, since analysts can directly work with the selected instance without the need for visual querying. Inspired by Pirolli and Card's information foraging framework~\cite{Pirolli}, which distinguishes between information processing tasks that are \textit{top-down} (from theory to data) and \textit{bottom-up} (from data to theory), we define \textit{top-down pattern specification} as the search-oriented paradigm where analysts query based on their in-the-head pattern (Figure~\ref{2dmodel} blue). On the other hand, in the realm of \textit{bottom-up data-driven inquiry} (Figure~\ref{2dmodel} red), the pattern of interest is unbeknownst and external to the user and must be driven by recommendations or queries that originate from the data (or equivalently, the visualization). As we will discuss latter, this process is a crucial but understudied topic in past works on VQSs.
%analysts often do not start with a known pattern instance. T
\par The second axis, \textbf{visualized attributes}, depicts how much the analyst knows about which X and Y axes she is interested in visualizing. In both the astronomy and genetics use cases, as well as past work in this space, data was in the form of time series with \texttt{known} visualized attributes. In the case of our material science participants, they wanted to explore relationships between different X and Y variables. In the realm of \texttt{unknown} attributes, context creation (Figure~\ref{2dmodel} green) is essential for allowing users to pivot across different visualization subspaces. %Most past VQSs assume that the analyst has a desired pattern in-the-head that could be conveyed through visual specification, such as a sketch.
\begin{figure}[h!]
  \centering
  \includegraphics[width=\linewidth]{figures/2dmodel.pdf}
  \caption{The design space of VQSs is characterized by how much the analyst knows about the visualized attributes and pattern instance. Colored area highlights the three different paradigms of VQSs. While prior work has focused soley on use cases in the blue region, we envision opportunities for VQSs beyond this to a larger space of use cases coverage in the red and green regions.}
  \label{2dmodel}
\end{figure}
\subsection{Design Goals and Challenges for VQS Paradigms}
We further explore the design objectives and challenges of each paradigm by developing a taxonomy for organizing how the aforementioned components fits into the paradigms of sensemaking in VQSs, as shown in Figure~\ref{fig:taxonomy}.
\begin{figure*}[ht!]
  \centering
  \includegraphics[width=0.9\linewidth]{figures/full_taxonomy.pdf}
  \caption{Taxonomy of functionalities in VQSs. From top, each of the three paradigm is broken down into key components in the system, which is instantiated as features in \zv. The bottom-most layer connects the use cases features that have practical or envisioned usage based on the evaluation study.}
  \label{fig:taxonomy}
\end{figure*}
% \par Drawing from our participatory design experience, evaluation study, and literature review in this space, we design a taxonomy for understanding the key functionalities in VQSs. In Figure~\ref{fig:taxonomy}, we show how each use cases makes use of the different features in \zv, then we organize the features into key components for VQSs, which belongs to one of the three paradigms in the VQS design space.
In particular, we will describe the main form of inquiry addressed by each paradigm\cut{(\textit{what, where, which})}, its characteristic use case, and design challenges in supporting these paradigms.
\boldpara{Top-down Pattern Specification} begins with user's intuition about how their desired patterns should look like based on `theory', including visualizations from past experiences or an abstract conceptions based on external knowledge. The goal of top-down pattern specification is to address the \textit{which} questions in visual sensemaking (\textit{which pattern instance exhibits this pattern?}), effectively moving rightwards to the gray area in Figure~\ref{2dmodel}, where the pattern instance is known. Based on this preconceived notion of what to search for, the design challenge is to translate the query in the analyst's head to a query executable by the VQS. In the Figure~\ref{fig:taxonomy} taxonomy, this includes both components for specifying the pattern, as well as controls governing the underlying algorithm of how shape-matching is performed. For example, A1 knows intuitively what a supernovae pattern looks like and the detailed constraints on the shape, such as the width and height of the peak or the level of error tolerance for defining a match. He can search for the transient pattern through sketching, select the option to ignore differences on the x axis, and changes the similarity metric for flexible matching.  %The design challenge of top-down pattern specification is to ----- enable users to How to translate the in-the-head query to visual query and how matching is done.
\boldpara{Bottom-up data-driven inquiry} is a browse-oriented sensemaking process enables users to go from data to theory to addresses the \textit{what} questions in the sensemaking process.
% While the usage of each querying feature may vary from one participant to the next, generally, result querying and pattern upload are considered bottom-up approaches that go from data to theory by enabling users to query via examples of known visualizations. Bottom-up data-driven inquiries
 For example, genetics participants do not have a preconceived knowledge of what to search for in the dataset. They were mostly interested in \textit{what types of patterns exist in the dataset} through representative trends. They queried mainly through these recommended results to jumpstart further queries. The goal of data-driven inquiry is to move towards the blue area in Figure~\ref{2dmodel} to help analysts gain more information about patterns of interest in-the-head.
% notion of what the pattern looks like
The design challenge include developing the right set of `stimuli' that could provoke further data-driven inquiries, as well as low-effort mechanisms to search via these results.
\boldpara{Context Creation} addresses the \textit{where} question of sensemaking by enabling analyst to navigate across different parts of the visualization collection. %to pivot across different visualization collections.
 %Analysts often navigate across different parts of the visualization subspace to narrow to a more manageable scope or to explore relationships between different visualization attributes. Context creation
The goal is to learn about \textit{where the patterns of interest lies}, effectively moving upwards in Figure~\ref{2dmodel} towards the known attributes region. For example, material scientists often do not start with a pattern in-the-head, but recognize salient trends such as inverse correlation or linear correlation. They switch between different visualized attributes or create different dynamic classes to study their data from different perspectives. The design challenge of context creation is to develop features that act as a `lens': navigating users to desired data subsets, visualizing and comparing how the data changes between the different lenses, and ensuring that the context is dynamically reflected across other functionalities in the VQSs.

% \par As illustrated in Figure \ref{fig:sbmodel}, our search-browse paradigm is motivated by the characteristic challenges and foraging acts each use cases pose on existing VQSs observed in our design study.
% \par In the astronomy use case, the participants knew the patterns they are looking for, but the patterns are hard to specify and find. The main challenge for the VQS involves finer specification of sketched patterns, such as amplitude and width of the peak and noise level tolerance for defining a pattern match. Describe more in D1. The main workflow for the astronomers in our user study involves \textit{enriching}, either through finer query specification or via filtering data subsets, to increase the probability that their queries would be more accurately matched with what they are looking for.

% %For example, G2 knew that there was three repeated measurements that was taken for every timestep, in one of the profiles there was a sharp jump whereas other datapoints are relatively flat, he then concludes by inspecting in the scatterplot view that the rise in gene expression is probably due to an experimental error rather than the activation of a gene, because the other two repeated measurements were similar in magnitude. In other words, the scatterplot view offered him density of points as another proxy to consider that was not offered in the line chart perspective.
%  %This is true for both participants with and without a desired pattern in mind. For the participant without a desired pattern (G2), he created groups based on quartile statistics of additional data attributes and recorded the most significant representative pattern.

% - What does the act of browsing and searching mean in the context of VQSs
%   - browse: viewing ranked result and any recommended results on the side, derived from the data and analysis context.
%   - search: act of going from a user's in-the-head concept to an actionable query that could be executed through the VQSs, most work have focussed on sketch, we allow more than this.
%   - The challenge of browsing and searching is well-known in information retrieval~\cite{Olston2003}, browse alone is limited by how much a user can browse and process at once, search alone can be ambiguous without sufficient context from looking at example results.
% \par Pirolli and Card's notional model further characterizes the trade-offs between three central activities in the information foraging process: exploring, enriching, and exploiting~\cite{Pirolli}.  We organize the features that we have developed in \zv into these foraging acts, as shown in Figure~\ref{feature_heatmap}.
% \par We find that participants often create unexpected workflows that chain together multiple analysis steps, including interactions, controls, and queries in order to address a higher-level research question. We find that participants often construct a central workflow, \tvcg{which they then iterate on while adding additional variations.} Their \emph{central workflow often resembles one of the three foraging acts} that aligns with the type of research question and dataset they are interested in. The variations are based on intermixing their central workflow with the other two foraging \tvcg{acts}.
% % We find that participants often have a strong inclination to perform tasks that resembles one of the three foraging act and sparsely intermixed with other activities to support their analysis, depending on the type of research question and dataset they are interested in.
% \par As illustrated in Figure \ref{fig:sbmodel}, our search-browse paradigm is motivated by the characteristic challenges and foraging acts each use cases pose on existing VQSs observed in our design study. For example, the genetics participants do not have a preconceived knowledge of what they want to search for in the dataset. They were mostly interested in \textit{exploring} clusters to gain an overall sense what profiles exist in the dataset \tvcg{through representative trends} and therefore queried mainly through drag-and-drop to jumpstart further queries. Point to need for D3 and D4. The variations to their main workflow include changing cluster sizes and display settings to offer them different perspectives on the dataset (\textit{exploit}) and filtering on data attributes (\textit{enriching}).
% \par In the astronomy use case, the participants knew the patterns they are looking for, but the patterns are hard to specify and find. The main challenge for the VQS involves finer specification of sketched patterns, such as amplitude and width of the peak and noise level tolerance for defining a pattern match. Describe more in D1. The main workflow for the astronomers in our user study involves \textit{enriching}, either through finer query specification or via filtering data subsets, to increase the probability that their queries would be more accurately matched with what they are looking for.
% \par The main workflow for material scientists involves \textit{exploiting}, since they spend the majority of their efforts performing ``close-reading'' of individual visualizations to understand the relationships between physical variables. The participants are able to identify interesting relationships between physical variables when they examine each closely, but they are not sure what patterns to look for to begin with. More in D2.
% %For example, G2 knew that there was three repeated measurements that was taken for every timestep, in one of the profiles there was a sharp jump whereas other datapoints are relatively flat, he then concludes by inspecting in the scatterplot view that the rise in gene expression is probably due to an experimental error rather than the activation of a gene, because the other two repeated measurements were similar in magnitude. In other words, the scatterplot view offered him density of points as another proxy to consider that was not offered in the line chart perspective.
%  %This is true for both participants with and without a desired pattern in mind. For the participant without a desired pattern (G2), he created groups based on quartile statistics of additional data attributes and recorded the most significant representative pattern.
% % [---] out of 9 of our participants had more than one main workflow.

%!TEX root = main.tex
\section{Evaluation Study Findings\label{sec:eval_findings}}
% \begin{figure*}[t!]
% \minipage{0.6\textwidth}
%   \includegraphics[width=\linewidth]{figures/evalstudytimeline.pdf}
%   \caption{Timeline of event code and component usage, with every timepoint as an event on the x axis. For clarity, we hide most of the event coding labels other than the insight labels. Black vertical tick indicates a session break, signaling the beginning of a new line of inquiry.}\label{fig:evalstudytimeline}
% \endminipage\hfill
% \minipage{0.4\textwidth}
%   \includegraphics[width=0.8\linewidth]{figures/PENcoding.pdf}
%   \caption{Heatmap of features categorized as practical usage (P), envisioned usage (E), and not useful (N).  \techreport{We find that participants preferred to query using bottom-up methods such as drag-and-drop over top-down approaches such as sketching or input equations. Participants found that data faceting via filter constraints and dynamic class creation were powerful ways to compare between subgroups or filtered subsets. The columns are arranged in the order of subject areas and the features are arranged in the order of the three foraging acts.}}\label{fig:feature_heatmap}
% \endminipage
% \end{figure*}
We recorded audio, video screen captures, and click-stream logs of the participants' actions during the evaluation study. We analyzed transcriptions of these recordings through open-coding and categorized every event in the user study as either a feature usage or via event coding labels. The event codes included insights or provoked actions related to science or data, occasions when participants were confused or wanted features that was unaddressed by the system, and the use of external tools.
% \begin{denselist}
%     \item Insight (Science): Insight that connected back to the science (e.g. ``This cluster resembles a repressed gene.'')
%     \item Insight (Data): Data-related insights (e.g. ``A bug in my data cleaning code generated this peak artifact.'')
%     \item Provoke (Science): Interactions or observations made while using the VQS that provoked a scientific hypothesis to be generated.
%     \item Provoke (Data): Interactions or observations made while using the VQS that provoked further data actions to continue the investigation.
%     \item Confusion: Participants were confused during this part of the analysis.
%     \item Want: Additional features that participant wants, which is not currently available on the system.
%     \item External Tools: The use of external tools outside of \zv to complement the analysis process.
% \end{denselist}
To characterize the usefulness of each feature, we further categorized the features into one of the three usage types based on how each feature was used during the study:
\begin{denselist}
    \item Practical: Features used in a sensible and meaningful way.
    \item Envisioned: Features which could be used practically if the envisioned data was available or if they conducted downstream analysis, but was not performed due to the limited time during the study.
    \item Not useful: Features that are not useful or do not make sense for the participant's research question and dataset.
\end{denselist}
\begin{figure}[h!]
  \includegraphics[width=0.8\linewidth]{figures/PENcoding.pdf}
  \caption{Heatmap of features categorized as practical usage, envisioned usage, and not useful. \techreport{We find that participants preferred to query using bottom-up methods such as drag-and-drop over top-down approaches such as sketching or input equations. Participants found that data faceting via filter constraints and dynamic class creation were powerful ways to compare between subgroups or filtered subsets. The columns are arranged in the order of subject areas and the features are arranged in the order of the three foraging acts.}}
  \label{fig:feature_heatmap}
\end{figure}
We derived these labels from the study transcription to circumvent self-reporting bias, which can often artificially inflate the usefulness of the feature or tool under examination.

For the remaining paper, we will discuss the results of the thematic analysis to understand the usage of these features and sensemaking paradigms in real-world analytic tasks.
%focus on understanding the design space of VQSs and highlight the takeaways of our study.%developing a process model and design guideline for insight formation in VQSs and divert our thematic analysis of how VQSs fit into the context of an analysis workflow to our technical report.% These observation inform our ----- search-browse paradigm
% \subsubsection{Discovery of Real-world insights}
% \par Our participants' original workflow often required them to compare between many visualizations manually through separate analysis and visualization steps. Three of the participants cited that this segmented analyze-then-visualize workflow was one of their chief bottlenecks. The cognitive overhead from the segmented workflow made them more hesitant to visualize the results of different parameters and data operations, as A2 noted:
% \begin{quote}
% The quick visualization is something that I could not do on my current framework. I could not query as fast as you do; I need to wait for it, plot, and then compare. Every time I plot, I need to define subplots for 12 visualizations, then its slower. That's the reason why I sometimes plot less, and I rely more on the statistics from the likelihood tests. Sometimes I plot less than I really should be doing.
% \end{quote}
% The ability to rapidly experiment with large numbers of hypotheses in real time is a crucial step in the agile creative process in helping analysts discover actionable insights~\cite{Shneiderman2007a}. Five out of nine participants discussed how the dynamic, interactive update of the visualization in \zv was the main advantage for using VQSs over their original workflow.
\subsection{DP1: The Inefficiency of Sketch}
% \subsection{DC3: Closing the loop in VQS sense-making cycle with bottom-up data-driven inquries}
\par Our interactions with the scientists showed that different modalities for inputting a query can be useful for different problem contexts. To our surprise, despite the prevalence of sketch-to-query systems in literature,
Figure \ref{fig:feature_heatmap} shows that only two out of our nine users had a practical usage for querying by sketching. %Overall, bottom-up querying via drag-and-drop was more intuitive and more commonly used than top-down querying methods, such as sketching or input equations.
\par The main reason why participants did not find sketching useful was that they often do not start their analysis with a pattern in mind. Later, their intuition about what to query is derived from other visualizations that they see in the VQS, in which case it made more sense to query using those visualizations as examples directly. In addition, even if a user has a query pattern in mind, sketch queries can be ambiguous or even impossible to draw by sketching (e.g. A2 looked for a highly-varying signal enveloped by a sinusoidal pattern indicating planetary rotation \includegraphics[width=1.7\baselineskip,keepaspectratio]{figures/impossible_sketch.png}).
\par The latter case is also supported by the unexpected use cases where sketching was simply used as a mechanism to modify dragged-and-dropped queries. As shown in Figure \ref{query_modification} (top), M2 first sketched a pattern to find solvent classes with anticorrelated properties. However, the sketched query did not return visualizations of interest. So, he instead dragged and dropped one of the peripheral visualizations that was close enough to his desired visualization to the sketchpad and then smoothed out the noise due to outlier datapoints by tracing a sketch over the visualization. M2 repeated this workflow twice in separate occurrences during the study and was able to derive insights from the results. Likewise, A3 was interested in pulsating stars characterized by dramatic changes in the amplitudes of the light curves. During the search, hotspots on stellar surfaces often show up as false positives as they also result in dramatic amplitude fluctuations, but happen at a regular intervals. In the VQS, A3 looked for patterns that exhibits amplitude variations, but also some irregularities. As shown in Figure \ref{query_modification} (bottom), she first picked out a regular pattern (suspected star spot), then modified it slightly so that the pattern looks more irregular.
\begin{figure}[ht!]
    \centering
    \includegraphics[width=\columnwidth]{figures/QueryModificationBySketch.pdf}
    \caption{\tvcg{Examples of query modification by M2 (top) and A3 (bottom) performed during the study The inital drag-and-dropped query is shown in blue and the sketch-modified queries in red.}
    \label{query_modification}}
    \vspace{-10pt}
\end{figure}
\par The lack of practical use of top-down pattern specification is also reflected in the fact that querying by equation is unpopular. Both querying mechanism adopt a problematic assumption that analysts start with a known and easy-to-specify search pattern in mind. In both astronomy and genetics use cases, the visualization patterns result from complex physical processes that could not be written down as an equation analytically. Even in the case of material science when analytical relationships do exist, it is challenging to formulate functional forms in an prescriptive, ad-hoc manner.
% Despite functional fitting being common in scientific data analysis, Figure \ref{feature_heatmap} shows that
% . However,
\par Both of these findings suggest that while sketching is an useful analogy for people to express their queries, \emph{the existing ad-hoc, sketch-only model for visualization querying is insufficient without data examples that can help analysts jumpstart their exploration}. Table~\ref{table:relatedwork} show that most past work focus on optimizing the components in the top-down paradigm, missing out largely on the key components in the other two paradigms, indicated by the absence of green features on the right hand side of the table. We suspect that the limited coverage in addressing different types of analytics use cases may be why existing sketch-to-query systems are not commonly adopted in practice. %This result points to a need for ----- in future VQSs. %This, however, points to an exciting direction for sketching interface in VQSs for developing advanced drawing and modification tools that enable more precise visualization query specification.}
%For instance, material science discovered a known inverse relationship during exploration
%Which is really interesting. Which is something that we observed experimentally also. That is an interesting insight right htere. This seems to suggest that there is a fundamental issue in if you want to try to get better on this axis, and get as low as possible, you lose out on the other axis.
%once they see it they know it but they don't know beforehand

\subsection{DP2: Practical Use of Bottom-up approaches}
\par Our results indicate that \emph{bottom-up data-driven inquiries are more common than top-down pattern specification when the users have no desired patterns in mind}, which is commonly the case for exploratory data analysis. Examples of practical uses of result querying includes inspecting the top-most similar visualizations that lie in a cluster and finding visualizations that are similar to an object of interest that exhibits a desired pattern.
\par Likewise, many participants envisioned use cases for pattern loading. The ability to load in data patterns as a query would enable users to compare visualizations between different experiments, species, or surveys, query with known patterns from an external reference catalog (e.g. important genes of interest, objects labeled as supernovae), or verify the results of a simulation or downstream analysis by finding similar patterns in their existing dataset. Users can also specify a more precise query that captures the essential shape features of a desired pattern (e.g. amplitude, width of peak), that cannot be precisely sketched. For example, the width of a supernovae light curve is characteristic to the radioactive decay rate of its chemical signature~\cite{Nugent1997}, so querying with an exact pattern template would be helpful for distinguishing the patterns of interest from noise.
\par The prevalence of bottom-up approaches not only point to the need for supporting result querying in VQSs, but also to the need for providing recommendation for users who may not have a desired pattern in mind. We found that geneticists often gain their intuition about the data from the recommended representative trends. One example of rapid insight discovery comes from G2 and G3, who identified that the three representative patterns shown in \zv---induced genes (profiles with expression levels staying up), repressed genes (started high but went down), and transients (go up and then come down at different time points)---corresponded to the same three groups of genes discussed in a recent publication\cite{Gloss2017}. The clusters provoked G2 to generate a hypothesis regarding the properties of transients: \textit{``Is that because all the transient groups get clustered together, can I get sharp patterns that rise and ebb at different time points?''} To verify this hypothesis, G2 increased the parameter controlling the number of clusters and noticed that the cluster no longer exhibited the clean, intuitive patterns he had seen earlier. G3 expressed a similar sentiment and proceeded by inspecting the visualizations in the cluster via drag-and-drop. He found a group of genes that all transitioned at the same timestep, while others transitioned at different timesteps. G3 described the process of using VQSs as doing ``detective work'' that provoked him to generate further scientific hypotheses as well as data actions.
\par By browsing through the ranked list of result, representative, and outlier in \zv, participants were also able to gain a peripheral overview of the data and spot anomalies during exploration. For example, A1 spotted time series that were too faint to look like stars after applying a filter constraint of CLASS\_STAR=1. After a series of query results browsing and consultation with an external database, he concluded that the dataset had been incorrectly labelled with all the stars with CLASS\_STAR=0 as 1 during data cleaning. These examples show that both the browsing-act through recommendations and performing search via these results are essential for `closing the loop' between the sensemaking acts in VQSs.
\subsection{DP3: Enriching Search with Context}
\par Past studies in taxonomies of visualization tasks have shown that it is important to design features that enable users to select relevant subsets of data in visual analytics\cite{Amar2005,Heer2012}. %We designed two dynamic faceting features coupled with coordinated views that enabled users to specify subsets of data they are querying on and see immediate changes updated in the query, representative, and outlier results.
We found that all participants either envisioned a use case or utilized components of the context creation paradigm offered in \zv to explore and compare subsets of their data.
\par A1 expressed that even though the filtering step could be easily done programmatically on the dataset and reloaded into \zv, filtering on-the-fly was a powerful way to dynamically test his hypothesis. Interactive filtering lowers the barrier between the iterative hypothesize-then-compare cycle, thereby enabling participants to test conditions and tune values that they would not have otherwise modified.
% echoing our previous finding that segmented workflow prevents extensive exploration.
During the study, participants used filtering to address questions such as: \textit{Are there more genes similar to a known activator when we subselect only the differentially expressed genes?} \texttt{DIFFEXP=1} (G2) or \textit{Can I find more supernovae candidates if I query only on objects that are bright and classified as a star?} \texttt{flux\textgreater10 AND CLASS\_STAR=1} (A1). Three participants had also used filtering as a way to pick out individual objects of interest to query with. For example, G2 set the filter as gene=9687 and explained that since ``this gene is regulated by the estrogen receptor, when we search for other genes that resemble this gene, we can find other genes that are potentially affected by the same factors.''
\par While filtering enabled users to narrow down to a selected data subset, dynamic class creation enabled users to compare relationships between multiple attributes and between subgroups of data. For example, M2 divided solvents in the database to eight different categories based on voltage properties, state of matter, and viscosity levels, by dynamically setting the cutoff values on the quantitative variables to create these classes. By exploring these custom classes, M2 learned that the relationship between viscosity and lithium solvation energy is independent of whether a solvent belongs to the class of high voltage or low voltage solvents and cited that dynamic class creation was central to learning about this previously-unknown attribute properties:
\begin{quote}
All this is really possible because of dynamic class creation, so this allows you to bucket your intuition and put that together. [...] I can now bucket things as high voltage stable, liquid stable, viscous, or not viscous and start doing this classification quickly and start to explore trends. [...] And look how quickly we can do it! Quite good!
\end{quote}
Context creation enables users to change the lens in which they look through when preforming visual querying, thereby creating more opportunities to see the queried data from different perspectives.
%Context creation is a useful ---- despite the --- pattern instance. Filtering still useful
%\par Participants employed \emph{a mix of bottom-up and top-down approaches when faceting through data in VQS}, including narrowing the search space based on some intuition about a phenomena, selecting individual visualizations, or specifying high-level groupings to compare and query with.
LIMITATIONS

% %!TEX root = main.tex
\section{Conclusion\label{sec:conclusion}}
point to focus on sci data but future work on quantified self and social viz


Our work closes the loop in VQS sensemaking so that VQS works in scenarios  where XY is unknown or Z is unknown. These areas were previously unexplored by past works.

As presented in this section, our study is the first that contributes towards a holistic understanding of the sensemaking process for visual querying.

\bibliographystyle{ACM-Reference-Format}
\bibliography{reference}
\end{document}
