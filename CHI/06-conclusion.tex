%!TEX root = main.tex
\section{Conclusion\label{sec:conclusion}}
point to focus on sci data but future work on quantified self and social viz


Our work closes the loop in VQS sensemaking so that VQS works in scenarios  where XY is unknown or Z is unknown. These areas were previously unexplored by past works.

As presented in this section, our study is the first that contributes towards a holistic understanding of the sensemaking process for visual querying.
----- how they are used in practice. %Rather than assuming visual querying through sketch is useful, we 

Our work is the first that evaluate on multiple case study participatory design, longitudinal study. 
Participants intermixed functionalities across different sensemaking paradigms to address their problem contexts.


Each sensemaking process fulfills a central role in participants' analysis to address their high-level research objectives. In discovering two `close the ----'----- en ---opening up pathways  potential pathway worflow, while clo

 more thoroughly characterize the problem design space of VQSs and taxonomy abstraction for understanding the sensemaking process in VQSs. Moreover, we performed design studies with three different subject areas with a diverse set of questions, datasets, and challenges to further generalize our findings.

 facilitating a diverse set of potential workflows and 

 Both query by sketch and equations adopt a problematic assumption that analysts start with a known and easy-to-specify search pattern in mind. 