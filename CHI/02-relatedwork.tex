%!TEX root = main.tex
\section{Related Works \label{sec:relatedworks}}
Visual query systems enable users to directly search for visualizations matching certain patterns through an intuitive specification interface. In the influential work on TimeSearcher~\cite{Hochheiser2001,Hochheiser2004}, the query is composed of one or more box constraints. Similarly, in QuerySketch~\cite{wattenberg2001sketching} and Google Correlate~\cite{mohebbi2011google}, the query is sketched as a pattern. Subsequent work have focussed on improving the expressiveness and flexibility of sketched queries through finer specification interface and pattern-matching algorithms, including specification of soft constraints~\cite{ryall2005querylines} and implicit relaxed selection techniques~\cite{Holz2009}. Recent work have also performed crowdsourced perceptual studies to understand how humans rank similarity in patterns subjectively, as well as to develop novel specification interface and algorithms~\cite{Eichmann2015,correll2016semantics,Mannino2018}. We summarize the list of features offered by these existing systems in Table~\ref{table:relatedwork}. While these systems have shown to be effective for visual querying in controlled lab studies, they have not been evaluated in-situ on real-world use cases. In this work, we attempt to more throughoutly understand the design space of VQSs and identified key components of VQSs beyond the sketch specification focus taken by existing work.
\begin{table*}[ht!]
    \begin{tabular}{l
>{\columncolor[HTML]{67FD9A}}l
>{\columncolor[HTML]{67FD9A}}l
>{\columncolor[HTML]{FD6864}}l
>{\columncolor[HTML]{FD6864}}l
>{\columncolor[HTML]{FD6864}}l
>{\columncolor[HTML]{FD6864}}l }
 & \multicolumn{1}{c}{\cellcolor[HTML]{DAE8FC}{\color[HTML]{000000} \thead{Pattern \\ Specification}}} & \multicolumn{1}{c}{\cellcolor[HTML]{DAE8FC}{\color[HTML]{000000} \thead{Match \\ Specification}}} & \multicolumn{1}{c}{\cellcolor[HTML]{FFCE93}{\color[HTML]{000000} \thead{View \\ Specification}}} & \multicolumn{1}{c}{\cellcolor[HTML]{FFCE93}{\color[HTML]{000000} \thead{Slice-\\and-Dice}}} & \multicolumn{1}{c}{\cellcolor[HTML]{FFFFC7}{\color[HTML]{000000} \thead{Result \\ Querying}}} & \multicolumn{1}{c}{\cellcolor[HTML]{FFFFC7}{\color[HTML]{000000} \thead{Recommend \\ Result}}} \\
Timesearcher \cite{Hochheiser2001,Hochheiser2004} & \cellcolor[HTML]{FD6864}{\color[HTML]{000000} } &  & \cellcolor[HTML]{67FD9A}{\color[HTML]{67FD9A} } & {\color[HTML]{FE0000} } & \cellcolor[HTML]{67FD9A} & {\color[HTML]{FE0000} } \\
QuerySketch \cite{wattenberg2001sketching} &  &  &  &  &  &  \\
QueryLines \cite{ryall2005querylines} &  &  &  &  &  &  \\
SoftSelect \cite{Holz2009} &  &  &  &  &  &  \\
Google Correlate \cite{mohebbi2011google} &  &  &  &  &  &  \\
TimeSketch \cite{Eichmann2015} &  &  &  &  &  &  \\
SketchQuery \cite{correll2016semantics} &  &  &  &  & \cellcolor[HTML]{67FD9A} &  \\
Qetch \cite{Mannino2018} &  &  &  &  &  & \cellcolor[HTML]{67FD9A} \\
Zenvisage (prototype) \cite{Siddiqui2017} &  &  &  &  & \cellcolor[HTML]{67FD9A} & \cellcolor[HTML]{67FD9A} \\
Zenvisage (after design study) &  &  & \cellcolor[HTML]{67FD9A} & \cellcolor[HTML]{67FD9A} & \cellcolor[HTML]{67FD9A} & \cellcolor[HTML]{67FD9A}
\end{tabular}
    \caption{Table summarizing the key components of VQSs (columns) covered by past systems (row). Green/red cell color indicates whether this feature exist in the system or not. Column header colors blue, orange, yellow represents processes: top-down querying, search with context, and bottom-up querying respectively, described more in Section~\ref{sec:guidelines}.}
    \label{table:relatedwork}
\end{table*}



% TimeSearcher
% [Hochheiser & Shneiderman 2001, 2004]
% Direct manipulation and composition of Timeboxes for pattern searching
% QuerySketch [Wattenberg 2001]
% Simple drawing interface for sketch querying stocks prices
% QueryLines [Ryall et al 2005]
% Techniques for specifying soft constraints for visual querying
% SoftSelect [Holz & Feiner, 2009]
% System that allow users to implicitly define a level of similarity that can
% vary across the search pattern depending on how they sketch
% TimeSketch
% [Eichmann & Zgraggen 2015]
% Collect dataset and evaluate how people perceive accuracy in pattern matching
% SketchQuery
% [Correll & Gleicher 2016]
% Resolving ambiguity of sketch through “invariants” (properties of sketch that is irrelevant to similarity matching)
% Qetch
% [Mannino & Abouzied 2018]
% scale-less freehand query without specifying query length or amplitude
% Zenvisage
% Full-fledge sketch-based visual query system

% \par \tvcg{Visual analytics systems, such as Tableau\cite{tableau}, support powerful visualization construction interfaces that enable users to specify their desired visual encoding and data subset for generating a visualization. However, during data exploration, users might only have a vague, high-level idea of what they want to address, rather than specific instances of what they want to visualize.} To address this issue, recent studies have explored the use of visualization recommendations to accelerate data exploration. The techniques used include using statistical and perceptual measures~\cite{key2012vizdeck,Wongsuphasawat2016}, past user history~\cite{gotz2009behavior}, and visualizations that look ``different'' from the rest~\cite{Vartak2015}.
% \par \tvcg{Instead of providing generic visualization recommendations, VQSs are a special class of visualization systems that enable users to more directly search for visualizations through an intuitive interface. We elaborate on our description of VQSs in the introduction and define VQSs as \emph{systems that allow users to specify the desired pattern via some high-level specification language or interface, with the system returning recommendations of visualizations that match the specified pattern.} One instantiation of VQSs are sketch-to-query interfaces that allows users to sketch the desired ``shape'' of a visualization with the system returning visualizations that look similar}~\cite{mohebbi2011google, wattenberg2001sketching,ryall2005querylines}. Other work has explored the types of shape features a user may be interested in for issuing more specific queries~\cite{correll2016semantics}. Additionally, some VQSs support specifying patterns through boxed constraints for range-queries~\cite{Hochheiser2004}, regular expressions~\cite{Zgraggen2015}, and natural language \cite{Gao2015,Setlur2016}.
