%!TEX root = main.tex
\section{Participants and Datasets}
During the design study, we observed participants as they conducted a cognitive walkthrough demonstrating their existing data analysis workflow. In this section, we describe our study participants and their use cases to highlight behaviors that participants have adopted for conducting certain analysis tasks.
\par\noindent\stitle{Astronomy:} The Dark Energy Survey (DES) is a multi-institution project that surveys 300 million galaxies over 525 nights to study dark energy~\cite{Drlica-Wagner2017}. The telescope also focuses on smaller patches of the sky on a weekly interval to discover astrophysical transients (objects whose brightness changes dramatically as a function of time), such as supernova explosions or quasars. Their data consist of a large collection of time series brightness observations associated with each object. For over five months, we worked closely with A1, an astronomer on the project's data management team working at a supercomputing facility. The scientific goal is to identify a smaller set of potential candidates that may be astrophysical transients in order to study their properties in more detail. \techreport{These insights can help further constrain physical models regarding the formation of these objects.}
\par In order to identify astronomical transients, astronomers programmatically generate visualizations of candidate objects with \texttt{matplotlib} and visually examine each light curve. While an experienced astronomer who has examined many transient light curves can often distinguish an interesting transient object from noise by sight, manual searching is time-consuming and error prone as the large majority of the objects are not astronomical transients. A1 was interested in \zv as he recognized how specific pattern queries could help scientists directly search for these rare objects.
\techreport{\par If an object of interest or region is identified through the visual analysis, then the astronomer may be interested in inspecting the image of the region for cross-checking that the significant change in brightness of the object is not due to an imaging artifact. This could be done using a custom built web-interface that facilitates the access of cutout images for a queried region of the sky.}
\par\noindent\stitle{Genetics:} Gene expression is a common measurement in genomics obtained via microarray experiments. \techreport{In these experiments, a grid containing thousands of DNA fragments are exposed to stimuli and measurements for the level at which a gene is expressed are recorded as a function of time.} We worked with a graduate student (G1) and professor (G3) at a research university who were using gene expression data to better understand how genes are related to phenotypes expressed during early development~\cite{Peng2016,Gloss2017}. Their data consisted of a collection of gene expression data over time for mouse stem cells aggregated over multiple experiments.\techreport{, downloaded from an online database\footnote{\url{ncbi.nlm.nih.gov/geo/}}.} %They were interested in using \zv to cluster gene expression data before conducting analysis with a downstream machine learning workflow.
\par To analyze the data, G1 loads the preprocessed data into a custom desktop application for visualizing and clustering gene expression data\techreport{\footnote{\url{www.cs.cmu.edu/~jernst/stem/}}}. After setting several system parameters and executing the clustering algorithm, the overlaid time series for each cluster is displayed on the interface. G1 visually inspects that the patterns in each cluster looks ``clean'' and checks that the number of outlier genes that do not fall into any of the clusters is low.  If the number of outliers is high or the clustered visualizations look ``unclean'', she reruns the analysis by increasing the number of clusters. When the visualized clusters look ``good enough'', G1 exports the cluster patterns to be used as features in their downstream regression tasks.
\par Prior to the study, G1 and G3 spent over a month attempting to determine the best number of clusters for their upstream analysis based on a series of static visualizations and statistics computed after clustering. While regenerating their results took no more than 15 minutes every time they made a change, the multi-step, segmented workflow meant that all changes had to be done offline.\techreport{, so that valuable meeting time was not wasted trying to regenerate results.} The team were interested in \zv as they saw how the ability to interactively query other time series with clustering results could dramatically speed up their collaborative analysis process.
\par\noindent\stitle{Material Science:} We collaborated with material scientists at a research university who are working to identify solvents that can improve battery performance and stability. These scientists work with large simulation dataset containing chemical properties for more than 280,000 different solvents. We worked closely with a a postdoctoral researcher (M1), professor (M2), and graduate student (M3) for over a year to design a sensible way of exploring their data using VQSs. Each row of their dataset represents a unique solvent with 25 different chemical attributes. They wanted to use \zv to identify solvents that not only have similar properties to known solvents but are also more favorable (e.g. cheaper or safer to manufacture). To search for these desired solvents, they need to understand how changes in certain chemical attributes affects other properties under specific conditions.
\par M1 starts his data exploration process by iteratively applying filters to a list of potential battery solvents using SQL queries. When the remaining list of the solvents is sufficiently small, he examines each solvent in more detail to weigh in the cost and availability to determine experimental feasibility. The scientists were interested in using \zv as it was impossible for them to uncover hidden relationships between different variables across large number of solvents manually.%(such as how changing one attribute affects another attribute)
