\newpage
\begin{table*}[ht!]
\begin{tabular}{|l|l|l|l|l|}
\hline
Component                                                                                                  & Feature                                                                                             & Purpose                                                                                                                                                                                                        & Task Example                                                                                                                                                                                                                    & \begin{tabular}[c]{@{}l@{}}Similar Features\\ in Past VQSs\end{tabular}                                                                                                              \\ \hline
\rowcolor[HTML]{AADFFD}
\cellcolor[HTML]{AADFFD}                                                                                   & \begin{tabular}[c]{@{}l@{}}Sketch-to-query\\ (Figure \ref{zvOverview}B1)\end{tabular}               & \begin{tabular}[c]{@{}l@{}}Freehand sketching for \\ specifying pattern query.\end{tabular}                                                                                                                    & \begin{tabular}[c]{@{}l@{}}\A Find patterns with a peak \\ and long-tail decay that\\ may be supernovae candidates.\end{tabular}                                                                                                & \begin{tabular}[c]{@{}l@{}}All include sketch \\ canvas except~\cite{Hochheiser2004}.\end{tabular}                                                                                   \\ \cline{2-5}
\rowcolor[HTML]{AADFFD}
\cellcolor[HTML]{AADFFD}                                                                                   & \begin{tabular}[c]{@{}l@{}}Input Equation\\ (Figure \ref{zvOverview}A1)\end{tabular}                & \begin{tabular}[c]{@{}l@{}}Specify a exact functional form as \\ a pattern query (e.g., y=$x^2$).\end{tabular}                                                                                                 & \begin{tabular}[c]{@{}l@{}}\M Find patterns exhibiting \\ inversely proportional \\ chemical relationship.\end{tabular}                                                                                                         & ----                                                                                                                                                                                 \\ \cline{2-5}
\rowcolor[HTML]{AADFFD}
\multirow{-3}{*}{\cellcolor[HTML]{AADFFD}\begin{tabular}[c]{@{}l@{}}Pattern \\ Specification\end{tabular}} & \begin{tabular}[c]{@{}l@{}}Pattern Upload\\ (Figure \ref{zvOverview}D2)\end{tabular}                & \begin{tabular}[c]{@{}l@{}}Upload a pattern consisting of \\ a sequence of points as a query.\end{tabular}                                                                                                     & \begin{tabular}[c]{@{}l@{}}\A Find supernovae based on \\ previously discovered sources.\end{tabular}                                                                                                                           & \cite{mohebbi2011google}                                                                                                                                                             \\ \hline
\rowcolor[HTML]{AADFFD}
\cellcolor[HTML]{AADFFD}                                                                                   & \begin{tabular}[c]{@{}l@{}}Smoothing\\ (Figure \ref{zvOverview}D2)\end{tabular}                     & \begin{tabular}[c]{@{}l@{}}Interactively adjusting the level of \\ denoising on visualizations,\\  effectively changing the degree of \\ shape approximation when \\ performing pattern matching.\end{tabular} & \begin{tabular}[c]{@{}l@{}}\textbf{A, M:} Eliminate patterns \\ matched to spurious noise.\end{tabular}                                                                                                                         & \begin{tabular}[c]{@{}l@{}}Smoothing~\cite{Mannino2018}\\ Angular slope queries~\cite{Hochheiser2004}\\ Trend querylines~\cite{ryall2005querylines}\end{tabular}                     \\ \cline{2-5}
\rowcolor[HTML]{AADFFD}
\cellcolor[HTML]{AADFFD}                                                                                   & \begin{tabular}[c]{@{}l@{}}Range \\ Selection\\ (Figure \ref{zvOverview}B2, D4)\end{tabular}        & \begin{tabular}[c]{@{}l@{}}Restrict to query only in specified\\  x/y ranges of interest through \\ brushing selected x-range and \\ filtering selected y-range.\end{tabular}                                  & \begin{tabular}[c]{@{}l@{}}\A Matching only around \\ shape exhibiting a peak.\\ \M Matching only around \\ shape region that exhibit linear\\ or exponential relationships\end{tabular}                                        & \begin{tabular}[c]{@{}l@{}}Text Entry~\cite{wattenberg2001sketching,Mannino2018}\\ Min/max boundaries~\cite{ryall2005querylines}\\ Range Brushing~\cite{Hochheiser2001}\end{tabular} \\ \cline{2-5}
\rowcolor[HTML]{AADFFD}
\multirow{-3}{*}{\cellcolor[HTML]{AADFFD}\begin{tabular}[c]{@{}l@{}}Match \\ Specification\end{tabular}}   & \begin{tabular}[c]{@{}l@{}}Range \\ Invariance\\ (Figure \ref{zvOverview}D1,4)\end{tabular}         & \begin{tabular}[c]{@{}l@{}}Ignoring vertical or horizontal \\ differences in pattern matching \\ through option for x-normalization\\ and y-invariant similarity metrics .\end{tabular}                        & \begin{tabular}[c]{@{}l@{}}\A Searching for existence of a\\ peak above a certain amplitude.\\ \G Searching for a \\ ``generally-rising" pattern.\end{tabular}                                                                  & Temporal invariants~\cite{correll2016semantics}                                                                                                                                      \\ \hline
\rowcolor[HTML]{FBE39C}
\cellcolor[HTML]{FBE39C}                                                                                   & \begin{tabular}[c]{@{}l@{}}Data selection\\ (Figure \ref{zvOverview}A)\end{tabular}                 & \begin{tabular}[c]{@{}l@{}}Changing the collection of \\ visualizations to iterate over.\end{tabular}                                                                                                          & \begin{tabular}[c]{@{}l@{}}\M Explore tradeoffs and relations\\ between physical attributes.\end{tabular}                                                                                                                       & ----                                                                                                                                                                                 \\ \cline{2-5}
\rowcolor[HTML]{FBE39C}
\multirow{-2}{*}{\cellcolor[HTML]{FBE39C}\begin{tabular}[c]{@{}l@{}}View \\ Specification\end{tabular}}    & \begin{tabular}[c]{@{}l@{}}Display control\\ (Figure \ref{zvOverview}D4)\end{tabular}               & \begin{tabular}[c]{@{}l@{}}Changing details of how \\ visualizations should be displayed.\end{tabular}                                                                                                         & \begin{tabular}[c]{@{}l@{}}\M Non-time-series data should \\ be displayed as scatterplot.\end{tabular}                                                                                                                          & ----                                                                                                                                                                                 \\ \hline
\rowcolor[HTML]{FBE39C}
\cellcolor[HTML]{FBE39C}                                                                                   & \begin{tabular}[c]{@{}l@{}}Filter\\ (Figure \ref{zvOverview}D3)\end{tabular}                        & \begin{tabular}[c]{@{}l@{}}Display and query only on data \\ that satisfies the composed \\ filter constraints.\end{tabular}                                                                                   & \begin{tabular}[c]{@{}l@{}}\A Eliminate unlikely \\ candidates by navigating to\\ more probable data regions.\\ \textbf{M, G:} Compare how overall \\ patterns change when filtered to \\ particular data subsets.\end{tabular} & ----                                                                                                                                                                                 \\ \cline{2-5}
\rowcolor[HTML]{FBE39C}
\multirow{-2}{*}{\cellcolor[HTML]{FBE39C}Slice-and-Dice}                                                   & \begin{tabular}[c]{@{}l@{}}Dynamic Class \\ (Figure~\ref{dcc})\end{tabular}                         & \begin{tabular}[c]{@{}l@{}}Create custom classes of data that \\ satisfies one or more specified range\\ constraints. Display aggregate \\ visualizations of separate classes.\end{tabular}                    & \begin{tabular}[c]{@{}l@{}}\textbf{A, M:} Examine aggregate patterns\\ for different data subsets.\end{tabular}                                                                                                                 & ----                                                                                                                                                                                 \\ \hline
\rowcolor[HTML]{B5E1A4}
\begin{tabular}[c]{@{}l@{}}Result\\ Querying\end{tabular}                                                  & \begin{tabular}[c]{@{}l@{}}Drag-and-drop\\ (Figure \ref{zvOverview}C, E)\end{tabular}               & \begin{tabular}[c]{@{}l@{}}Querying with any selected result \\ visualization as pattern query (either \\ from recommendations or results).\end{tabular}                                                       &                                                                                                                                                                                                                                 & \begin{tabular}[c]{@{}l@{}}Drag-and-drop~\cite{Hochheiser2001}\\ Double-Click~\cite{correll2016semantics}\end{tabular}                                                               \\ \hline
\rowcolor[HTML]{B5E1A4}
Recommendation                                                                                             & \begin{tabular}[c]{@{}l@{}}Representative \\ and Outliers\\ (Figure \ref{zvOverview}E)\end{tabular} & \begin{tabular}[c]{@{}l@{}}Displaying visualizations of \\ representative trends and outlier\\ instances based on clustering.\end{tabular}                                                                     & \begin{tabular}[c]{@{}l@{}}\A Examine anomalies and debug \\ data errors through outliers.\\ \textbf{G, M:} Understand representative \\ trends common to this dataset \\ (or filtered subset).\end{tabular}                    & ----                                                                                                                                                                                 \\ \hline
\end{tabular}
\caption{\change{List of major features incorporated throughout participatory design. We organize each feature based on its functional component. Table cells are further colored according to the sensemaking process that each component corresponds to (Blue: Top-down, Yellow: Context creation, Green: Bottom-up). We list the functional purpose of each feature based on how it is implemented in \zvpp, example use cases from participatory design (\A astronomy, \M material science, \G genetics), and how similar features have been incorporated in past VQSs.}\label{bigfeaturetable}}
\end{table*}
\newpage
\clearpage
