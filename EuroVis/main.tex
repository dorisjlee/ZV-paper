% ---------------------------------------------------------------------------
% Author guideline and sample document for EG publication using LaTeX2e input
% D.Fellner, v1.14, Nov. 29, 2017
\PassOptionsToPackage{table}{xcolor}
\documentclass{egpubl}
\usepackage{eurovis2019}

% --- for  Annual CONFERENCE
% \ConferenceSubmission   % uncomment for Conference submission
% \ConferencePaper        % uncomment for (final) Conference Paper
% \STAR                   % uncomment for STAR contribution
% \Tutorial               % uncomment for Tutorial contribution
% \ShortPresentation      % uncomment for (final) Short Conference Presentation
% \Areas                  % uncomment for Areas contribution
% \MedicalPrize           % uncomment for Medical Prize contribution
% \Education              % uncomment for Education contribution
% \Poster                 % uncomment for Poster contribution
% \DC                     % uncomment for Doctoral Consortium
%
% --- for  CGF Journal
% \JournalSubmission    % uncomment for submission to Computer Graphics Forum
% \JournalPaper         % uncomment for final version of Journal Paper
%
% --- for  CGF Journal: special issue
% \SpecialIssueSubmission    % uncomment for submission to , special issue
\SpecialIssuePaper         % uncomment for final version of Computer Graphics Forum, special issue
%                          % EuroVis, SGP, Rendering, PG
% --- for  EG Workshop Proceedings
% \WsSubmission      % uncomment for submission to EG Workshop
% \WsPaper           % uncomment for final version of EG Workshop contribution
% \WsSubmissionJoint % for joint events, for example ICAT-EGVE
% \WsPaperJoint      % for joint events, for example ICAT-EGVE
% \Expressive        % for SBIM, CAe, NPAR
% \DigitalHeritagePaper
% \PaperL2P          % for events EG only asks for License to Publish

% --- for EuroVis
% for full papers use \SpecialIssuePaper
% \STAREurovis   % for EuroVis additional material
% \EuroVisPoster % for EuroVis additional material
% \EuroVisShort  % for EuroVis additional material

%
 \electronicVersion % can be used both for the printed and electronic version

% !! *please* don't change anything above
% !! unless you REALLY know what you are doing
% ------------------------------------------------------------------------

% for including postscript figures
% mind: package option 'draft' will replace PS figure by a filname within a frame
\ifpdf \usepackage[pdftex]{graphicx} \pdfcompresslevel=9
\else \usepackage[dvips]{graphicx} \fi
\usepackage{multirow}

\PrintedOrElectronic

% prepare for electronic version of your document
\usepackage{t1enc,dfadobe}

\usepackage{egweblnk}
\usepackage{cite}
\usepackage{balance}  % to better equalize the last page
\usepackage{graphics} % for EPS, load graphicx instead
\usepackage[T1]{fontenc}
\usepackage{txfonts}
\usepackage{mathptmx}
\usepackage[htt]{hyphenat}
% \usepackage[pdftex]{hyperref}

% \usepackage{booktabs}
% \usepackage{textcomp}
\usepackage{array}
\usepackage{booktabs}
\usepackage{pifont}
\usepackage{xspace}
\usepackage{setspace}
\usepackage{graphicx}
\usepackage[skip=5pt,font=small]{caption}
\usepackage[textsize=tiny]{todonotes}
% Some optional stuff you might like/need.
\usepackage{microtype} % Improved Tracking and Kerning
% \usepackage[all]{hypcap}  % Fixes bug in hyperref caption linking
\usepackage{ccicons}  % Cite your images correctly!
% \usepackage[utf8]{inputenc} % for a UTF8 editor only
\usepackage{verbatim}
\usepackage{relsize}
\usepackage{etoolbox}
\usepackage{lipsum}   % for filler text
\usepackage{setspace} % for \onehalfspacing and \singlespacing macros
\usepackage[normalem]{ulem}
\usepackage{enumitem}
\usepackage{relsize,etoolbox}% http://ctan.org/pkg/{relsize,etoolbox}
\usepackage{makecell}
\renewcommand\theadalign{bc}
\renewcommand\theadgape{\Gape[4pt]}
\renewcommand\cellgape{\Gape[4pt]}
\newcommand{\tabitem}{~\llap{{\tiny\textbullet}}~}
\AtBeginEnvironment{quote}{\small}% Step font down one size relative to current font.

\def\plaintitle{You can't always sketch what you want: Understanding Sensemaking in Visual Query Systems}
\def\emptyauthor{}
\def\plainkeywords{Visual analytics, exploratory analysis, visual query}

\newenvironment{denselist}{
  \vspace{-5pt}
    \begin{list}{\small{$\bullet$}}%
    {\setlength{\itemsep}{0ex} \setlength{\topsep}{0ex}
    \setlength{\parsep}{0pt} \setlength{\itemindent}{0pt}
    \setlength{\leftmargin}{1.5em}
    \setlength{\partopsep}{0pt}}}%
    {\end{list}\vspace{-5pt}}
\newcommand{\squishlist}{
   \begin{list}{$\bullet$}
    { \setlength{\itemsep}{0pt}
      \setlength{\parsep}{2pt}
      \setlength{\topsep}{0pt}
      \setlength{\partopsep}{0pt}
      \leftmargin=25pt
\rightmargin=0pt
\labelsep=5pt
\labelwidth=10pt
\itemindent=0pt
\listparindent=0pt
\itemsep=\parsep
    }
}
\newcommand*{\img}[1]{%
    \raisebox{-.3\baselineskip}{%
        \includegraphics[
        height=\baselineskip,
        width=\baselineskip,
        keepaspectratio,
        ]{#1}%
    }%
}
\newcommand{\squishend}{\end{list}}
\newcommand{\npar}{\par\noindent}
% use extensively to toggle between paper and TR
\newcommand{\eat}[1]{}
% \newcommand{\papertext}[1]{{\leavevmode\color{blue}{#1}}}
% \newcommand{\techreport}[1]{{\leavevmode\color{red}{#1}}}
\newcommand{\papertext}[1]{#1}
\newcommand{\techreport}[1]{}
\newcommand{\nonannon}[1]{#1}
\newcommand{\annon}[1]{}
\newcommand{\boldpara}[1]{\par\noindent\textbf{#1}}
% de-facto paragraph format
\newcommand{\stitle}[1]{\noindent\textbf{#1}}
\newcommand{\nstitle}[1]{\\\noindent\textbf{#1}}
\newcommand{\cut}[1]{{\leavevmode\color{lightgray}{#1}}}
\newcommand{\ccut}[1]{} %confirmed cut
% \newcommand{\rot}[1]{\rotatebox{90}{#1}}
\newcommand{\rot}[1]{\rotatebox[origin=c]{90}{\parbox{2.6cm}{\centering#1}}}
\newcommand{\A}{\textbf{A: }}
\newcommand{\M}{\textbf{M: }}
\newcommand{\G}{\textbf{G: }}
\newcommand{\problemlist}{
    \nstitle{Motivating Use Case Challenges:}
    \vspace{-5pt}
    \begin{enumerate}[label=\textbf{C\arabic*:},leftmargin=0.7cm]
}
\newcommand{\featurelist}{
    \stitle{Instantiated Feature:}
    \vspace{-5pt}
    \begin{enumerate}[label=\textbf{F\arabic*:},leftmargin=0.7cm]
}
\newcommand{\enumend}{
  \end{enumerate}
  \vspace{-5pt}
}
\urlstyle{leo}

% To make various LaTeX processors do the right thing with page size.
\def\pprw{8.5in}
\def\pprh{11in}
\special{papersize=\pprw,\pprh}
\setlength{\paperwidth}{\pprw}
\setlength{\paperheight}{\pprh}
\setlength{\pdfpagewidth}{\pprw}
\setlength{\pdfpageheight}{\pprh}

% create a shortcut to typeset table headings
% \newcommand\tabhead[1]{\small\textbf{#1}}
\newcommand{\zv}{\textit{zenvisage}\xspace}
\newcommand{\zvpp}{\textit{zenvisage++}\xspace}
\newcommand{\astro}{\textit{astro}\xspace}
\newcommand{\bio}{\textit{genetics}\xspace}
\newcommand{\matsci}{\textit{matsci}\xspace}
\newcommand{\change}[1]{{\leavevmode\color{red}#1}}
% \newcommand{\change}[1]{{\leavevmode#1}}

\newcommand{\agp}[1]{\textcolor{blue}{Aditya: #1}}
\newcommand{\dor}[1]{\textcolor{green}{Doris: #1}}
\newcommand{\kar}[1]{\textcolor{magenta}{Karrie: #1}}

\usepackage{xparse}   % http://ctan.org/pkg/xparse
% Rotation: \rot[<angle>][<width>]{<stuff>}
% \NewDocumentCommand{\rot}{O{320} O{2.2em} m}{\makebox[#2][r]{\rotatebox{#1}{#3}}}%
\newcommand*\OK{\ding{51}}
\renewenvironment{quote}{%
   \list{}{%
     \leftmargin0.15cm
     \rightmargin\leftmargin
   }
   \item\relax
}
{\endlist}
%%%%%%%%%%%%%%%%%%%%%%%%%%%%%%%%%%%%%%%%%%%%%%%%%%%%%%%%%%%%%%%%
%%%%%%%%%%%%%%%%%%%%%% START OF THE PAPER %%%%%%%%%%%%%%%%%%%%%%
%%%%%%%%%%%%%%%%%%%%%%%%%%%%%%%%%%%%%%%%%%%%%%%%%%%%%%%%%%%%%%%%%

\begin{document}
% \title{Understanding usage patterns for query specification in sketch-based visual query systems: A Case Study with Zenvisage}
%\title{I can't always sketch what I want: \\ Understanding Sensemaking in Visual Query Systems through Design Studies with Zenvisage++}
\title{\emph{You can't always sketch what you want}: \\ Understanding Sensemaking in Visual Query Systems\vspace{-20pt}}
\author{Submission ID: 1051\vspace{-50pt}}
%How are visual query systems used in practice?\\ A Design Study with Zenvisage}
\maketitle
\begin{abstract}
Visual query systems (VQSs) \change{empower} users to interactively search for line charts
with desired visual patterns typically specified using intuitive sketch-based interfaces. Despite their potential in accelerating data exploration, more than a decade of past work on VQSs has not been translated to adoption in practice. Through a year-long collaboration with experts from three diverse domains, we examine the role of VQSs in real data exploration workflows, \change{enhance an existing VQS to support these workflows via a participatory design process}, and evaluate how \change{VQS components} are used in practice. Via these observations, we formalize a taxonomy of key \change{capabilities} for VQSs, organized by three sensemaking processes. Perhaps somewhat surprisingly, we find that ad-hoc sketch-based querying is not commonly used during data exploration, since
analysts are \change{often unable} to precisely articulate the patterns they are interested in. We find that there is a spectrum of VQS-centric data exploration workflows, depending on the application \change{domain}, and that many of these workflows are not effectively supported in present-day VQSs. Our insights can pave the way for next-generation VQSs to be adopted in a variety of real-world applications.
% On characterizing the spectrum of workflows
% adopted,
% we find that there is
% Moreover, we find that there is a
% variety of typical workflows
% We offer guidelines for the next generation of
% VQSs

% We advocate that next-generation VQSs should support two other forms of sensemaking to pave way for these systems to be adopted to a variety of real-world applications.
% The increasing availability of rich and complex data in a variety of scientific domains poses a pressing need for tools to enable scientists to rapidly make sense of and gather insights from data. One proposed solution is to design visual query systems (VQSs) that allow scientists to interactively search for desired patterns in their datasets. While many existing VQSs promise to accelerate exploratory data analysis by facilitating this search, they are not widely used in practice. Through a year-long collaboration with scientists in three distinct domains---astronomy, genetics, and material science---we study the impact of various features within VQSs that can aid rapid visual data analysis, and how VQSs fit into scientists' analysis workflow. Our findings offer design guidelines for improving the usability and adoption of next-generation VQSs, paving the way for VQSs to be applied to a variety of scientific domains.
\end{abstract}
\keywords{Visual analytics, exploratory analysis, visual query}
\raggedbottom
%!TEX root=main.tex
 \vspace{-5pt}
 \section{Introduction\label{sec:intro}}
 % one for each key finding: a) many features deemed to be of importance to VQSs by domain experts, not all supported by present-day VQSs b) sketch is inefficient, perhaps explaining why present-day VQSs are not popular c) identify 3 typical workflows involving various sensemaking modalities in different proportions, depending on the application
 %To discover patterns of interest, analysts construct line chart visualizations \cut{using toolkits like \texttt{ggplot} or \texttt{matplotlib}, or visualization construction interfaces like Excel or Tableau,} \change{by} specifying {\em exactly} what they want to visualize. For example, when trying to find celestial objects corresponding to supernovae, which have a specific pattern of brightness over time, astronomers individually inspect the corresponding line chart for each object---numbering in the hundreds---until they find ones that match the pattern.
 Line charts are commonly employed during data exploration---the intuitive connected patterns often illustrate complex underlying processes
 and yield interpretable and visually compelling data-driven narratives~\cite{Few2012}. 
%To discover patterns of interest, analysts often have to construct and inspect thousands of line chart visualizations manually to find ones that match their desired pattern.
 \achange{However, discovering line charts that display certain meaningful patterns, trends, or characteristics of interest is often  an overwhelming and error-prone process, consisting of manual examination of large numbers of line charts. For example, when trying to find supernovae, which exhibits a unique pattern of brightness over time (an initial peak followed by a long-tail decay), astronomers often have to construct and inspect thousands of line chart visualizations manually to find ones that match their desired pattern.}
 %\techreport{For example, when trying to find celestial objects corresponding to supernovae, which have a specific pattern of brightness over time, astronomers individually inspect the corresponding line chart for each object---numbering in the hundreds---until they find ones that match the pattern.}\ccut{Similarly, when trying to infer relationships between two physical properties for different subsets of battery electrolytes, scientists need to individually visualize these properties for each subset (out of an unbounded number of such subsets) until they identify relationships that make sense to them.} 
 %This process of manual exploration of large numbers of line charts \change{for pattern identification} is not only error-prone, but also overwhelming for analysts. 
 To address this \change{exploration} challenge, there \change{has} been a large number of papers dedicated to building \emph{Visual Query Systems} (VQSs)\change{---the term coined by Ryall et al.~\cite{ryall2005querylines} to describe} systems that allow users to specify and search for desired \change{line chart patterns via visual} interfaces~\cite{mohebbi2011google,Hochheiser2004,wattenberg2001sketching,Siddiqui2017VLDB,ryall2005querylines,correll2016semantics,Mannino2018,Eichmann2015,Holz2009}. %visual patterns via interactive 
 These interfaces typically include a sketching canvas where users can draw a visual pattern of interest, with the system automatically traversing all potential visualization candidates to find those that match the specification. 
 % \par While this intuitive specification interface appears to be a promising solution
 \par While these intuitive specification interfaces were proposed as a promising solution to the problem of painful manual exploration of visualizations \change{for time-series analysis}~\cite{ryall2005querylines,wattenberg2001sketching}, to the best of our knowledge, VQSs have not lived up to these expectations and are not very commonly used in practice. \achange{One likely reason for the lack of VQS adoption may be attributed to how prior work have focused almost solely on optimizing for better pattern-matching algorithms and interactions, with few that invested in understanding actual user needs and how VQSs can be used for solving real-world problems.} {\em Our paper seeks to understand how VQSs can actually be used in practice, as a first step towards the broad adoption of VQSs in data analysis}. Unlike prior work on VQSs, we set out to not only evaluate VQSs in-situ on real problem domains, but also involve participants from these domains in the VQS design. We present findings from a series of interviews, contextual inquiry, participatory design, and user studies with scientists from three different domains---{\em astronomy, genetics,} and {\em material science}---over the course of
 a year-long collaboration. \change{The amount of time we invested in each of these three diverse domains surpasses the norm in this field and is key to uncovering the insights presented in this paper.} As illustrated in Figure~\ref{science_goal}, these domains were selected to capture a diverse set of goals and datasets wherein VQSs can help address important scientific questions, such as: How does a treatment affect the expression of a gene in a breast cancer cell-line? Which battery components have sustainable levels of energy-efficiency and are safe and
 cheap to manufacture in production?
 \begin{figure}[ht!]
 	\centering
 	\includegraphics[width=\linewidth]{figures/science_goal.pdf}
 	\caption{Desired insights, problem and dataset challenges for each of the three application domains in our study.}
 	\label{science_goal}
 	\vspace*{-15pt}
 \end{figure}
 \par Via contextual inquiry and interviews, we first identified challenges in existing data analysis workflows in these domains
 that could be potentially addressed by a VQS. Building on top of an existing open-source VQS, \zv~\cite{Siddiqui2017,Siddiqui2017VLDB}, we engaged participants in a process of participatory design (PD)~\cite{Muller1993,BodkerGronbaek,HoltzblattJones} to enable them to better compose data exploration workflows that lead to insight discovery, over the course of a year. \change{Rather than targeting a domain-specific solution, we chose to perform participatory design across multiple domains (an uncommon practice in visualization design studies) to observe differences and commonalities across domains to synthesize \change{high-level} insights regarding the use of VQSs.} \achange{While designing and performing this multi-phased, mixed-methods research agenda across three different use cases was an ambitious feat, this endeavor was necessary for addressing the qualitative, participant-centered research questions investigated in this work.}
 \par We organize our PD findings into a taxonomy of VQS capabilities, involving three sensemaking processes inspired by Pirolli and Card's notional model of analyst sensemaking~\cite{Pirolli}. The sensemaking processes include \emph{top-down pattern search} (translating a pattern ``in-the-head'' into a visual query), \emph{bottom-up data-driven inquiries} (querying or recommending based on data), and \emph{context-creation} (navigating across different collections of visualizations). We find that prior VQSs have focused on enabling top-down processes \change{(via sketching \achange{capabilities})}, \achange{but have largely overlooked the two other processes that we found to be essential in all three domains. These missing \achange{capabilities} partially explains why \achange{prior VQSs} have not been widely adopted in practice}.
% other two processes that we found to be crucial for all three domains.
 %to gather feedback and iterate on VQS feature designs, culminating in a new enhanced VQS, \zvpp.
 \par To study how various VQSs are used in practice, we conducted a final evaluation study with nine participants using our final VQS prototype to address their research questions on their own datasets. During this 1.5-hour study, participants \achange{gained} novel scientific insights,
 such as identifying a star with a transient pattern that was known to harbor a Jupiter-sized planet\achange{, discovering a previously-unknown relationship between solvent properties,} and finding characteristic gene expression profiles confirming the results of a related publication. %\techreport{Participants also gain additional insights about their datasets, including debugging mislabeled features and uncovering erroneous data preprocessing procedure applied to a collaborator's dataset.}%\techreport{, and discovering that the dip in an astronomical light curve is caused by saturated imaging equipment overlooked by the existing error-detection pipeline.}
 %\agp{Explain why these findings are important.}\dor{I think saying that planetary discovery is related to future colonization is a bit too much here and significance of characteristic gene expression profiles. Also we already described the significance of each domain earlier with the `important scientific questions' part.}
 %that goes from a pattern in-the-head to a desired visualization
 \par By analyzing the evaluation study results, we \achange{were somewhat surprised to discover} that sketching a pattern for querying is often ineffective on its own. This is due to the fact that sketching makes the problematic assumption that users know the pattern that they want to sketch and are able to sketch it precisely.\achange{ However, this is not the case in practice. For example, the geneticists from our study often did not have a preconceived knowledge of what to sketch and search for and relied heavily on recommended common patterns and outlying ones provided by the VQS to jumpstart their queries. Likewise, while the material scientists from our study were interested in datapoints that fall within specific value-ranges, they did not have an apriori notion of what these desired patterns would look like. Overall,} participants typically opted to combine sketching with other means of pattern specification---one common mechanism was to drag-and-drop a recommended pattern onto the canvas, and then modify it (e.g., by smoothing it out). %\cut{However, most VQSs do not support these other mechanisms (as we argued earlier, they typically focus only on top-down sensemaking processes, without covering bottom-up and context creation)}\dor{cutting this out since already mentioned 2 paragraphs ago}.
 %Participants were, however, able to apply the two other sensemaking processes to gain novel scientific insights, such as identifying a star with a transient pattern that was known to harbor a Jupiter-sized planet, finding characteristic gene expression profiles confirming the results of a related publication, and discovering mislabelled features from a data preprocessing mistake.
 % Further analysis of how participants transition between different sensemaking processes during analysis using a Markov model illustrated
 \par \change{To further understand how participants engaged with VQSs in their analytical workflows, we used a Markov model to \achange{characterize} how participants transitioned between different sensemaking processes during their analysis. \achange{We found that} participants often constructed a diverse set of \achange{analytical} workflows tailored to their domains by focusing} around a primary sensemaking process, while iteratively interleaving their analysis with the two other processes. This finding points to how all three sensemaking processes, along with seamless transitions between them, are \achange{crucial} for enabling \achange{the effective use and adoption of VQSs for addressing real-world challenges.}%for data exploration.%For example, participants often center on a main sensemaking process, while interleaving variations with other two processes as they iterate on an analytic task.
 %---including the construction of a Markov model---
 \par To the best of our knowledge, our study is the \emph{first to holistically examine how VQSs can be designed to fit the needs of real-world, analysts and how they are actually used in practice}. Working with participants from multiple domains enabled us to compare the differences and commonalities across different domains, thereby identifying general VQS challenges and requirements for supporting common analytical goals. Our contributions include:
 \begin{denselist}
 \item a characterization of the problems addressable by VQSs through design studies with three different domains,
 \item the construction of a taxonomy of essential VQSs capabilities leading to a sensemaking model for VQSs, grounded in participatory design findings, %, as well as an articulation of the problem space that is amenable to VQSs
 \item an integrative VQS, \zvpp, post participatory design, capable of facilitating rapid hypothesis generation and insight discovery,
 \item study findings on how VQSs are used in practice, leading to the development of a novel sensemaking model for VQSs. %including the ineffectiveness of
 %evaluation
 % sketching and the ---- workflow
 \end{denselist}
 Our work not only opens up a new space of opportunities beyond the narrow use cases considered by prior studies, but also advocates common design guidelines and end-user considerations for building next-generation VQSs.
%!TEX root = main.tex
\section{Methods\label{sec:methods}}
\subsection{Background and Motivation}
\par Visual query systems enable users to directly search for visualizations matching certain patterns through an intuitive specification interface. Early work in this space focused on interfaces to search for time series with specific patterns, including TimeSearcher~\cite{Hochheiser2001,Hochheiser2004}, where the query specification mechanism is a rectangular box, filtering out all of the time series that does not pass through it, QuerySketch~\cite{wattenberg2001sketching} and Google Correlate~\cite{mohebbi2011google}, where the query is sketched as a pattern on canvas, filtering out all of the time series that have a different shape. Subsequent work recognized the ambiguity in sketching by studying how humans rank the similarity in patterns~\cite{Eichmann2015,correll2016semantics,Mannino2018} and improving the expressiveness of sketched queries through finer-grained specification interfaces and pattern-matching algorithms~\cite{ryall2005querylines,Holz2009}.
%performed crowdsourced perceptual studies to understand how humans rank similarity in patterns subjectively
% , including the use of soft constraints~\cite{ryall2005querylines} and implicit relaxed selection techniques~\cite{Holz2009}.
% In addition to this ongoing work, recent work have also performed crowdsourced perceptual studies to understand how humans rank similarity in patterns subjectively~\cite{Eichmann2015,correll2016semantics,Mannino2018}.
\par While these systems have been effective in controlled lab studies, they have never been designed and evaluated in-situ on real-world use cases. Even when use cases were involved~\cite{Hochheiser2004,correll2016semantics}, the inclusion of these use cases had a narrow objective and had little influence on the major design decisions of the system. In the context of Munzner's nested model~\cite{munzner2009nested}, this represents the common ``downstream threat'' of jumping prematurely into the deep levels of \textit{encoding, interaction, or algorithm design}, before a proper \textit{domain problem characterization} and \textit{data/operation abstraction design}. \change{In this work, we performed design studies~\cite{lam2012empirical,shneiderman2006strategies,Sedlmair2012} with three different subject areas for \textit{domain problem characterization}. Comparing and contrasting between the diverse set of questions, datasets, and challenges across these three use cases revealed new generalizable insights and enabled us to better understand how VQSs can be extended for novel and unforeseen use cases.} Based on these findings, we develop a feature taxonomy for understanding the sensemaking process in VQSs as part of the \textit{data/operation abstraction design}. Finally, we validated the abstraction design with grounded evaluation~\cite{Plaisant2004,Isenberg2008}, where we invite participants to bring in their own datasets and research problems that they have a vested interest in to test our final deployed system. \change{Next, we will describe these two phases of our study in more detail.}
%  on our final system ----
% , drawing from ---- grounded evaluation
% validated with usage of deployed system, target users , 
% grounded evaluation
% evaluated on 
% participatory design with multiple case studies to ----- target domain, while 
% opted for a qualitative mixed-methods approach drawing from ethnographic methods, participatory design, and grounded evaluation~\cite{Plaisant2004,lam2012empirical,shneiderman2006strategies,Sedlmair2012,Isenberg2008} to more thoroughly characterize the problem design space of VQSs and taxonomy abstraction for understanding the sensemaking process in VQSs. Moreover, we performed design studies with three different subject areas with a diverse set of questions, datasets, and challenges to further generalize our findings.
%Most of these systems have not been evaluated in-situ on real-world use cases. Even when design study was performed~\cite{correll2016semantics,Hochheiser2001}, the focus these  narrow use case in mind, and these --- never adopted in practice,either ---did not --- didn't influence the design decisions made to --- . We found there is a genuine need in the community to more thoroughly understand the design space of VQSs and how various components of VQSs are used in practice. In this work, We make use of multiple case studies ....
%\par While these systems have been shown to be effective for visual querying in controlled lab studies, they have not been evaluated in-situ on real-world use cases. 
%In this work, we adopted a mixed methods research methodology that draws inspiration from ethnographic methods, iterative and participatory design, and controlled studies~\cite{Plaisant2004,lam2012empirical,shneiderman2006strategies,Muller1993} Participatory design has been successfully used in the development of interactive visualization systems in the past~\cite{Aragon2008,Chuang2012}. Sedlmair et al. \cite{Sedlmair2012} advocate that design study methodology is suitable for use cases in which the data is available for prototyping, but the task is only partially known and the information is partially in the user's head. 
%to understand how VQSs can be used in scientific data analysis. %In that regard, our scientific use cases with VQS is well-suited for a design study methodology, as we learn about the scientist's data and analysis requirements and design interactions that helps users translate their ``in-the-head'' specifications into actionable visual queries.
\vspace{-10pt}
\begin{table} 
    \includegraphics[width=0.8\linewidth]{figures/related_works_table.pdf}
    \caption{Table summarizing whether key functionalities of VQSs (columns) are covered by past systems (row), indicated by checked cells. Column header colors blue, orange, green represents three sensemaking process (top-down querying, search with context, and bottom-up querying) described in Section~\ref{sec:pd_findings}. The heavily-used, practical features in our study for context-creation and bottom-up inquiry is largely missing from prior VQSs.}
    \label{table:relatedwork}
    \vspace{-29pt}
\end{table}
\subsection{Phase I: Participatory Design}
\par We recruited participants by reaching out to research groups via email and word of mouth, who have experienced challenges in data exploration. Based on our early conversations with analysts from 12 different potential application areas, we narrowed down to three use cases in astronomy, genetics, and material science for our participatory design study, chosen based on their suitability for VQSs as well as diversity in use cases. Six scientists from three research groups participated in the design of \zv. On average, participants had more than 6 years of research experience working in their respective fields. Via interviews and cognitive walkthroughs with researchers from the three different scientific research groups, we identified the needs and challenges of these use cases. 
\par For the participatory design study, we built on an existing VQS, \zv~\cite{Siddiqui2017,Siddiqui2017VLDB}, that allowed users to sketch a pattern or drag-and-drop an existing visualization as a query, with the system returning visualizations that had the closest Euclidean distance from the queried pattern. We chose to build on top of \zv, since it was open-source, extensible, and encompassed a large selection of features compared to existing systems, which focused largely on features for pattern and match specification (as compared in Table~\ref{table:relatedwork}).
%The details of the system is described in \cite{Siddiqui2017,Siddiqui2017VLDB}, which focused on the system and scalability aspects of the VQSs. 

 %Table~\ref{table:relatedwork} summarizes the list of features offered by these existing systems.
%  that ---
% identified potential opportunities for VQSs 
% and potential opportunities for VQSs
% We initially spoke to analysts from 12 different potential application areas and narrowed down to three use cases in astronomy, genetics, and material science for our participatory design study, based on their suitability for VQS as well as diversity in use cases. Six scientists from three research groups participated in the design of \zv. On average, participants had more than 8 years of research experience working in their respective fields. %\techreport{We list the participants in Table~\ref{participants}, and will refer to them by their anonymized ID as listed in the table throughout the paper.}
% \par Given our early conversations with participants, we built a basic VQS to serve as the functional prototype in the design study. This early VQS prototype allowed users to sketch a pattern or drag-and-drop an existing visualization as a query, then the system would return visualizations that had the closest Euclidean distance from the queried pattern. The details of the system is described in \cite{Siddiqui2017,Siddiqui2017VLDB}, which focused on the system and scalability aspects of the VQSs.
% 	% \begin{figure}[ht!]
% 	% \centering
% 	% \includegraphics[width=\linewidth]{figures/oldZV_nozql.pdf}
% 	% \caption{The \zv prototype allowed users to sketch a pattern in (a), which would then return (b) results that had the closest Euclidean distance from the sketched pattern. The system also displays (c) representative patterns obtained through K-Means clustering and (d) outlier patterns to help the users gain an overview of the dataset.}
% 	% \label{oldZV}
% 	% \end{figure}
% \par The use of functional prototypes is common and effective in participatory design to provide a starting point for the participants, as studied by Ciolfi et al.\cite{Ciolfi2016}. %For example, Ciolfi et al.\cite{Ciolfi2016} studied two different alternatives to co-design (starting with open brief versus functional prototype) in the development of museum guidance systems and found that while both approaches were equally fruitful, functional prototypes can make addressing a specific challenge more immediate and focused.
% Our motivation for providing a functional prototype at the beginning of the participatory design sessions is to showcase capabilities of VQSs. Especially since VQSs are not common in the existing workflows of these scientists, participants may not be able to imagine their use cases without a starting point.
\par During participatory design, we collaborated with each team closely with an average of two meetings per month, where we learned about their datasets, objectives, and how VQSs could help address their research questions. A summary timeline of our collaboration with participants over a year and features inspired by their use cases can be found in Figure \ref{timeline}. 
%Participants provided datasets they were exploring from their domain, whereby they had a vested interest in using a VQS to address their own research questions. 
Through this process, we identified and incorporated more than 20 desired features into \change{the new version of our VQS, \zvpp, described more in Section~\ref{sec:pd_findings}.}
\begin{figure*}[ht!]
	\centering
	\captionsetup{justification=centering,margin=2cm}
	\vspace{-10pt}
	\includegraphics[width=6in]{figures/timeline_anon.pdf}
	\vspace{-6pt}\caption{Timeline for progress in participatory design studies.}
	\label{timeline}
	\vspace{-10pt}
\end{figure*}
\vspace{-10pt}
\subsection{Phase II: Evaluation Study}
% \par Visualization systems are often evaluated using controlled studies that measure the user's performance against an existing visualization baseline~\cite{Plaisant2004}. Techniques such as artificially inserting ``insights'' or setting predefined tasks for example datasets work well for objective tasks, \techreport{such as debugging data errors~\cite{kandel2011wrangler,Patel2010},} but they are unsuitable for trying to learn about the types of real-world queries users may want to pose on VQSs. %Due to the unrealistic nature of controlled studies, many have proposed using a more multi-faceted, ethnographic approach to understand how analysts perform visual data analysis and reasoning~\cite{Plaisant2004,lam2012empirical,shneiderman2006strategies,munzner2009nested,Sedlmair2012}.
At the end of our participatory design study, we performed a qualitative evaluation to study how analysts interact with different VQS components in practice. In order to make the evaluation more realistic, we invited participants to use datasets that they have a vested interest in exploring to address unanswered research questions. \change{As shown in Table~\ref{participants},} the evaluation study participants included the six scientists from participatory design, along with three additional ``blank-slate'' participants who had never encountered \zvpp before. While participatory design subjects actively provided feedback on \zvpp with their data, they only saw us demonstrating their requested features and explaining the system to them, rather than actively using the system on their own. So the evaluation study was the first time that all participants used \zvpp to explore their datasets.
\par \change{Evaluation study participants} were recruited from each of the three aforementioned research groups, as well as domain-specific mailing lists. Prior to the study, we asked potential participants to fill out a pre-study survey to determine eligibility. Eligibility criteria included: being an active researcher in the subject area with more than one year of experience, and having worked on a research project involving data of the same nature used in participatory design. The nine participants brought a total of six different datasets to the study. \techreport{The research questions and objectives of the participants were diverse even among the same subject area. Examples included understanding gene expression profiles of breast cancer cells after a particular treatment and comparing common patterns among stars that exhibit planetary transits versus stars that do not.\techreport{from the Kepler space telescope\footnote{\url{www.nasa.gov/mission_pages/kepler/main/index.html}}.}} 
% Four of the evaluation studies were conducted remotely. Participants had the option of exploring their own dataset or an existing dataset that they provided to us during the participatory design process. All three blank-slate participants opted to explore their own datasets.
 %After loading their dataset, we emailed them a screenshot of a visualization from our tool to verify that we configured the system to meet their needs.
\par At the start, participants were provided with an interactive walk-through explaining the system details and given approximately ten minutes for a guided exploration of \zvpp with a preloaded real-estate example dataset from Zillow \cite{zillow}.\techreport{This dataset contained housing data for various cities, metropolitan areas, and states in the U.S. from 2004-15.} After familiarizing themselves with the tool, we loaded the participant's dataset and encouraged them to talk-aloud during data exploration and use external tools. If the participant was out of ideas\ccut{ for three minutes}, we suggested one of the ten main VQS functionalities \techreport{\footnote{query by sketching, drag-and-drop, pattern loading, input equations, representative and outliers, narrow/ignore x-range options, filtering, data smoothing, creating dynamic classes,  data export}}that they had not yet used. If any of these operations were not applicable to their specific dataset, they were allowed to skip the operation after having considered how it may or may not be applicable to their workflow. The user study ended after they covered all ten main functionalities. On average, data exploration lasted for 63 minutes. After the study, we asked them open-ended questions about their experience.% and suggested an appropriate choice of axis to begin the exploration.
%\par During the exploration phase, participants were informed that they could use other tools as needed.
\begin{table}[h!]
\centering
\vspace{-10pt}
\includegraphics[width=\linewidth]{figures/participant_info.pdf}
\caption{Participant information. The Likert scale used for dataset familiarity ranges from 1 (not at all familiar) to 5 (extremely familiar).}
\label{participants}
\vspace{-15pt}
\end{table}
%!TEX root = main.tex
\section{Participants and Datasets\label{sec:participantdatasets}}
At the start of our design study, we observed participants as they conducted cognitive walkthroughs demonstrating their existing data analysis workflows. Next, we describe our study participants and their preferred analysis workflows.%use cases to highlight behaviors that participants have adopted for conducting certain analysis tasks.
\par\noindent\stitle{Astronomy:} The Dark Energy Survey is a multi-institution project that surveys 300 million galaxies over 525 nights to study dark energy~\cite{Drlica-Wagner2017}. The telescope \change{used to survey these galaxies} also focuses on smaller patches of the sky on a weekly interval to discover astrophysical transients (objects whose brightness changes dramatically as a function of time), such as supernovae or quasars. Their dataset consists of a large collection of brightness observations over time\change{, one} associated with each astrophysical object, called a {\em light curve}, and plotted as a time series. For over five months, we worked closely with A1, an astronomer on the project's data management team working at a supercomputing facility. Their scientific goal is to identify potential astrophysical transients in order to study their properties. \techreport{These insights can help further constrain physical models regarding the formation of these objects.}
\par To identify transients, astronomers programmatically generate visualizations of candidate objects with \texttt{matplotlib} and visually examine each light curve. While an experienced astronomer who has examined many transient light curves can often distinguish an interesting transient object from noise by sight, manual searching for transients is time-consuming and error prone\change{, since} the large majority of the objects are false positives. A1 was interested in VQSs as he recognized how specific pattern queries could help astronomers directly search for these rare transients.
\techreport{\par If an object of interest or region is identified through the visual analysis, then the astronomer may be interested in inspecting the image of the region for cross-checking that the significant change in brightness of the object is not due to an imaging artifact. This could be done using a custom built web-interface that facilitates the access of cutout images for a queried region of the sky.}
\par\noindent\stitle{Genetics:} Gene expression is a common measurement in genetics obtained via microarray experiments~\cite{Gloss2017}. \techreport{In these experiments, a grid containing thousands of DNA fragments are exposed to stimuli and measurements for the level at which a gene is expressed are recorded as a function of time.} We worked with a graduate student (G1) and professor (G3) at a research university who were using gene expression data to understand how genes are related to phenotypes expressed during early development\techreport{\cite{Peng2016,Gloss2017}}. Their data consisted of a collection of gene expression profiles over time for mouse stem cells, aggregated over multiple experiments.\techreport{, downloaded from an online database\footnote{\url{ncbi.nlm.nih.gov/geo/}}.} %They were interested in using \zv to cluster gene expression data before conducting analysis with a downstream machine learning workflow.
\par Their typical workflow is as follows: G1 first loads the preprocessed gene expression data into a custom desktop application for visualizing and clustering it\techreport{\footnote{\url{www.cs.cmu.edu/~jernst/stem/}}}. After setting several system parameters and executing the clustering algorithm, the overlaid time series for each cluster is displayed on the interface. G1 visually inspects that the patterns in each cluster looks ``clean'' and checks that the number of outlier genes (i.e., those that do not fall into any of the clusters) is low.  If the number of outliers is high or the clustered visualizations look ``unclean'', she reruns the analysis by increasing the number of clusters. When the visualized clusters look ``good enough'', G1 exports the cluster patterns to her downstream regression tasks.
\par Prior to the study, G1 and G3 spent over a month attempting to determine the best number of clusters based on a series of static visualizations and statistics computed after clustering. While regenerating their results took no more than 15 minutes every time they made a change, the multi-step, segmented workflow meant that all changes had to be done offline.\techreport{, so that valuable meeting time was not wasted trying to regenerate results.} The team were interested in VQSs as they saw how interactively querying time series with clustering results could dramatically speed up their collaborative analysis process.
%that can improve battery performance and stability
\par\noindent\stitle{Material Science:} We collaborated with material scientists at a research university who are working to identify solvents for energy efficient and safe batteries. These scientists work on a large simulation dataset containing chemical properties for more than 280,000 solvents. Each row of their dataset represents a unique solvent with 25 different chemical attributes. We worked closely with a a postdoctoral researcher (M1), professor (M2), and graduate student (M3) for over a year to design a sensible way of exploring their data. They wanted to use VQSs to identify solvents that not only have similar properties to known solvents but are also more favorable (e.g., cheaper or safer to manufacture). To search for these desired solvents, they need to understand how changes in certain chemical attributes affect other properties under specific conditions.
\par M1 typically starts his data exploration process by iteratively applying filters to a list of potential battery solvents using SQL queries. When the remaining list of the solvents is sufficiently small, he examines each solvent in more detail to factor in the cost and availability to determine experimental feasibility. The scientists were interested in VQSs as it was impossible for them to uncover hidden relationships between different attributes across large number of solvents manually.%(such as how changing one attribute affects another attribute)
%!TEX root = main.tex
\begin{figure*}[ht!]
  \centering
  \vspace{-5pt}
  \includegraphics[width=0.95\linewidth]{figures/zvpp_system.pdf} %5.5
  \vspace{-5pt}\caption{\change{The \zvpp system consist of : (A) data selection panel (where users can select visualized dataset and attributes), (B) query canvas (where the queried data pattern is submitted and displayed), (C) results panel (where the visualizations most similar to the queried pattern is displayed as a ranked list), (D) control panel (where users can adjust various system-level settings), and (E) recommendation (where the typical and outlying trends in the dataset is displayed).}}
  \label{zvOverview}
  \vspace{-5pt}
\end{figure*}
\section{System-level Participatory Design Findings\label{sec:pd_findings}}
%All of the three domains described in the previous section recognized the need for a VQS. As discussed in Section~\ref{sec:methods},
%we worked closely with participants to develop features to address their problems and challenges.
\change{
  Holzblatt and Jones~\cite{HoltzblattJones} describes contextual inquiry as a technique that forms the basis for ``\textit{developing a system model that will support user's work}'' that subsequently ``\textit{fosters participatory design}''. Given the need for a VQS highlighted in contextual inquiry interviews, we further collaborate with participants to develop features to address their problems and challenges. In this section, we first reflect on our feature discovery process to introduce the participatory design (PD) findings, then we provide a high-level system overview of our PD product, \zvpp.
  \subsection{The Collaborative Feature Discovery Process}
  \par Throughout the PD process, we identified various subtasks based on the scientist's workflow and elicited intermediate feedback.
  Bodker et. al.~\cite{BodkerGronbaek} cites the importance of encouraging user participation and creativity in cooperative design through different techniques, such as future workshops, critiques, and situational role-playing. Similarly, our PD objective was to collect as many feature proposals as possible, while being inclusive across different domains. We further organized these features into Table~\ref{bigfeaturetable} through a coding process~\cite{Muller2012} by one of the authors. Using the feature list, example usage scenarios from PD, and similar functionalities in existing VQSs as open codes, we then organized this list into axial codes representing `components': core functionalities that are essential in VQSs. Finally, as we will describe in Section~\ref{sec:sensemaking}, the selective codes represents each of the sensemaking process (denoted by cell colors in Table~\ref{bigfeaturetable}). Given the highly-evolving, undirected nature of exploratory data analysis~\cite{Keim2006,Tukey1970}, this technique comes with its advantages and limitations. User requests (or lack thereof) may not always translate to a direct need. For instance, we found that introducing the newly-added features from \zvpp that addressed a particular use case often results in discovering an unexpected practice usage of the feature with other groups of participants. Having feature proposals inspired by multiple use cases can also lead to more generalized design choice. For example, we spoke to astronomers who wanted to eliminate sparse time series from their visual queries. In the same week, our material science collaborators expressed a need for inspecting only solvents with properties above a certain threshold. Through these use cases, data filtering arose as a crucial, common operation that was later incorporated into \zvpp to support this class of queries.
  %, leading to a comprehensive list of added features listed in Table~\ref{bigfeaturetable}.
  \par While our collective brainstorming led to the cross-pollination and generalization of features, this technique can also lead to unnecessary features that result in wasted engineering efforts. During the design phase, there were numerous problems and features proposed by participants, but not all were incorporated in the tool. Based on our meeting logs with participants, we found that reasons for not taking a feature from the design to implementation stage included:
  \begin{denselist}
  \item Nice-to-haves: One of the most common reasons for unincorporated features comes from participant's requests for nice-to-have features. The amount of nice-to-have features that one could envision for the tool is endless. We use
  two criteria to heuristically judge whether to implement a particular feature:
  \begin{enumerate}[leftmargin=*]
  \item \textit{Necessity:} Without this feature, can the analyst still work with this dataset using the VQS and meet their information needs?
  \item \textit{Generality:} Will this feature benefit only this specific use case or be potentially useful for other domains as well?
  \end{enumerate}
  \item ``One-shot'' operations: We decided not to include features that only needed to be performed once and remain fixed thereafter during the analysis workflow. For example, certain data preprocessing operations such as filtering null values only needed to be performed once with an external tool.
  \item Substantial research or engineering effort: Some proposed features did not make sense in the context of VQS. For example, M3 proposed functional fitting to obtain fitting coefficients. Other features requires a completely different set of research questions. For example, the question of how to properly compute similarity between time series with non-uniform number of datapoints arose in the astronomy and genetics use case, but requires the development of a novel distance metric and algorithm that is out of the scope of our design study objective.
  \item Underdeveloped ideas: Other feature requirements came from casual specification that were underspecified. For example, A1 wanted to look for objects that have deficiency in one band and high emission in another band, but the scientific definition of ``deficiency'' in terms of brightness levels was ambiguous.
  \end{denselist}
  \par Failure to identify these early signs in the design phase may result in feature implementations that turn out not to be useful for the participants. Given exhaustive nature of Table~\ref{bigfeaturetable}, each motivated by example use cases from one or more domains, we further organize the features list in terms of the Section~\ref{sec:sensemaking} sensemaking framework and assess its effectiveness in the Section~\ref{sec:eval_findings} evaluation study.
}
%Our experimental evaluation shows that some of our feature choices also suffer from these pitfalls. For example, we incorporated the feature to reverse the y-axis so that astronomers could better understand magnitude measurements (as larger magnitudes counter-intuitively corresponds to dimmer objects). In hindsight, the feature was not crucial for the analysis since another derived measure present in dataset could be selected instead and the feature was solely specific to the astronomy use case. In the end, we found that this feature was not used by any of the participants (including the proposer) in the user study.

% These longitudinal collaboration led to a list of features that help support these subtasks. Given the , we encouraged participants to ----how VQS features could assist them in their PD
%  which is later organized into a ------ framework in Section ----, and evaluated in ----

% collective brainstorm exercise
% partnership
% during this ---- `what-if'
% with the participant's dataset
% The result of our ----
% The goal of our participatory design process was to collect brainstorm --- a comphrensive list of ----
% encourage creativity and inclusiveness across domains---
% EDA is inherently unstructured process, highly evolving ,etc
%
\begin{table*}[ht]
  \begin{tabular}{|l|l|l|l|l|}
\hline
Component & Feature & Purpose & Task Example & \begin{tabular}[c]{@{}l@{}}Similar Features\\ in Past VQSs\end{tabular} \\ \hline
\rowcolor[HTML]{AADFFD}
\cellcolor[HTML]{AADFFD} & \begin{tabular}[c]{@{}l@{}}Query by Sketch\\ (Figure \ref{zvOverview}B1)\end{tabular} & \begin{tabular}[c]{@{}l@{}}Freehand sketching for \\ specifying pattern query.\end{tabular} & \begin{tabular}[c]{@{}l@{}}\A Find patterns with a peak \\ and long-tail decay that\\ may be supernovae candidates.\end{tabular} & \begin{tabular}[c]{@{}l@{}}All include sketch \\ canvas except~\cite{Hochheiser2004}.\end{tabular} \\ \cline{2-5}
\rowcolor[HTML]{AADFFD}
\cellcolor[HTML]{AADFFD} & \begin{tabular}[c]{@{}l@{}}Input Equation\\ (Figure \ref{zvOverview}A1)\end{tabular} & \begin{tabular}[c]{@{}l@{}}Specify a exact functional \\ form as a pattern query \\ (e.g., y=$x^2$).\end{tabular} & \begin{tabular}[c]{@{}l@{}}\M Find patterns exhibiting \\ inversely proportional \\ chemical relationship.\end{tabular} & ---- \\ \cline{2-5}
\rowcolor[HTML]{AADFFD}
\multirow{-3}{*}{\cellcolor[HTML]{AADFFD}\begin{tabular}[c]{@{}l@{}}Pattern\\ Specification\end{tabular}} & \begin{tabular}[c]{@{}l@{}}Pattern Upload\\ (Figure \ref{zvOverview}D2)\end{tabular} & \begin{tabular}[c]{@{}l@{}}Upload a pattern consisting\\ of a sequence of points as \\ a query.\end{tabular} & \begin{tabular}[c]{@{}l@{}}\A Find supernovae based on \\ previously discovered sources.\end{tabular} & \begin{tabular}[c]{@{}l@{}}Upload CSV\\ \cite{mohebbi2011google}\end{tabular} \\ \hline
\rowcolor[HTML]{AADFFD}
\cellcolor[HTML]{AADFFD} & \begin{tabular}[c]{@{}l@{}}Smoothing\\ (Figure \ref{zvOverview}D2)\end{tabular} & \begin{tabular}[c]{@{}l@{}}Interactively adjusting the level \\ of denoising on visualizations,\\ effectively changing the degree\\ of shape approximation when \\ performing pattern matching.\end{tabular} & \begin{tabular}[c]{@{}l@{}}\textbf{A, M:} Eliminate patterns \\ matched to spurious noise.\end{tabular} & \begin{tabular}[c]{@{}l@{}}Smoothing\\ ~\cite{Mannino2018}\\ Angular slope queries\\ ~\cite{Hochheiser2004}\\ Trend querylines\\ ~\cite{ryall2005querylines}\end{tabular} \\ \cline{2-5}
\rowcolor[HTML]{AADFFD}
\cellcolor[HTML]{AADFFD} & \begin{tabular}[c]{@{}l@{}}Range \\ Selection\\ (Figure \ref{zvOverview}B2, D4)\end{tabular} & \begin{tabular}[c]{@{}l@{}}Restrict to query only in \\ specific x/y ranges of interest \\ through brushing selected\\ x-range and filtering \\ selected y-range.\end{tabular} & \begin{tabular}[c]{@{}l@{}}\A Matching only around \\ shape exhibiting a peak.\\ \M Matching only around \\ shape region that exhibit linear\\ or exponential relationships\end{tabular} & \begin{tabular}[c]{@{}l@{}}Text Entry\\ ~\cite{wattenberg2001sketching,Mannino2018}\\ Min/max boundaries\\ ~\cite{ryall2005querylines}\\ Range Brushing\\ ~\cite{Hochheiser2001}\end{tabular} \\ \cline{2-5}
\rowcolor[HTML]{AADFFD}
\multirow{-3}{*}{\cellcolor[HTML]{AADFFD}\begin{tabular}[c]{@{}l@{}}Match\\ Specification\end{tabular}} & \begin{tabular}[c]{@{}l@{}}Range \\ Invariance\\ (Figure \ref{zvOverview}D1,4)\end{tabular} & \begin{tabular}[c]{@{}l@{}}Ignoring vertical or horizontal \\ differences in pattern matching \\ through option for x-range\\ normalization and y-invariant\\ similarity metrics .\end{tabular} & \begin{tabular}[c]{@{}l@{}}\A Searching for existence of a\\ peak above a certain amplitude.\\ \G Searching for a \\ ``generally-rising" pattern.\end{tabular} & \begin{tabular}[c]{@{}l@{}}Temporal invariants\\ ~\cite{correll2016semantics}\end{tabular} \\ \hline
\rowcolor[HTML]{FBE39C}
\cellcolor[HTML]{FBE39C} & \begin{tabular}[c]{@{}l@{}}Data selection\\ (Figure \ref{zvOverview}A)\end{tabular} & \begin{tabular}[c]{@{}l@{}}Changing the collection of \\ visualizations to iterate over.\end{tabular} & \begin{tabular}[c]{@{}l@{}}\M Explore tradeoffs and \\ relationships between \\ physical attributes.\end{tabular} & ---- \\ \cline{2-5}
\rowcolor[HTML]{FBE39C}
\multirow{-2}{*}{\cellcolor[HTML]{FBE39C}\begin{tabular}[c]{@{}l@{}}View\\ Specification\end{tabular}} & \begin{tabular}[c]{@{}l@{}}Display control\\ (Figure \ref{zvOverview}D4)\end{tabular} & \begin{tabular}[c]{@{}l@{}}Changing the details of \\ how visualizations should\\ be displayed.\end{tabular} & \begin{tabular}[c]{@{}l@{}}\M Non-time-series data should \\ be displayed as scatterplot.\end{tabular} & ---- \\ \hline
\rowcolor[HTML]{FBE39C}
\cellcolor[HTML]{FBE39C} & \begin{tabular}[c]{@{}l@{}}Filter\\ (Figure \ref{zvOverview}D3)\end{tabular} & \begin{tabular}[c]{@{}l@{}}Display and query only on data \\ that satisfies the composed \\ filter constraints.\end{tabular} & \begin{tabular}[c]{@{}l@{}}\A Eliminate unlikely \\ candidates by navigating to \\ more probable data regions.\\ \textbf{M, G:} Compare how overall\\ patterns change when filtered \\ to particular data subsets.\end{tabular} & ---- \\ \cline{2-5}
\rowcolor[HTML]{FBE39C}
\multirow{-2}{*}{\cellcolor[HTML]{FBE39C}Slice-and-Dice} & \begin{tabular}[c]{@{}l@{}}Dynamic Class \\ (Figure~\ref{dcc})\end{tabular} & \begin{tabular}[c]{@{}l@{}}Create custom classes of data \\ that satisfies one or more \\ specified range constraints. \\ Display aggregate \\ visualizations for separate\\ data classes.\end{tabular} & \begin{tabular}[c]{@{}l@{}}\textbf{A, M:} Examine aggregate \\ patterns of different data \\ classes.\end{tabular} & ---- \\ \hline
\rowcolor[HTML]{B5E1A4}
\begin{tabular}[c]{@{}l@{}}Result \\ Querying\end{tabular} & \begin{tabular}[c]{@{}l@{}}Drag-and-drop\\ (Figure \ref{zvOverview}C, E)\end{tabular} & \begin{tabular}[c]{@{}l@{}}Querying with any selected\\ result visualization as pattern\\ query (either from \\ recommendations or results).\end{tabular} & \begin{tabular}[c]{@{}l@{}}\textbf{A, G, M:} Find other objects that\\ are similar to X; Examine what \\ other objects similar to X look \\ like overall.\end{tabular} & \begin{tabular}[c]{@{}l@{}}Drag-and-drop\\ ~\cite{Hochheiser2001}\\ Double-Click\\ ~\cite{correll2016semantics}\end{tabular} \\ \hline
\rowcolor[HTML]{B5E1A4}
Recommendation & \begin{tabular}[c]{@{}l@{}}Representative \\ and Outliers\\ (Figure \ref{zvOverview}E)\end{tabular} & \begin{tabular}[c]{@{}l@{}}Displaying visualizations of \\ representative trends and outlier\\ instances based on clustering.\end{tabular} & \begin{tabular}[c]{@{}l@{}}\A Examine anomalies and debug \\ data errors through outliers.\\ \textbf{G, M:} Understand representative \\ trends common to this dataset \\ (or filtered subset).\end{tabular} & ---- \\ \hline
\end{tabular}
  \caption{\change{List of major features incorporated throughout participatory design. We organize each feature based on its functional component. Table cells are further colored according to the sensemaking process that each component corresponds to (Blue: Top-down, Yellow: Context creation, Green: Bottom-up). We list the functional purpose of each feature based on how it is implemented in \zvpp, example use cases from participatory design (\A astronomy, \M material science, \G genetics), and how similar features have been incorporated in past VQSs.}\label{bigfeaturetable}}
\end{table*}
%\dor{Trouble with vertically centering text in Component column}
\clearpage

\change{
  \subsection{System Overview\label{sec:system}}
  The aforementioned features were incrementally incorporated and improved over time as described in the Figure~\ref{timeline} timeline, leading to our final PD product, \zvpp. The \zvpp interface is organized into 5 major regions that dynamically updates upon user interactions. Typically, analysts begin their analysis by selecting the dataset and attribute to visualize in the \emph{data selection panel} (Figure~\ref{zvOverview}A). Then, they specify a pattern of interest as a query (hereafter referred to as \emph{pattern query}), through either sketching, inputting an equation, uploading a data pattern, or dragging and dropping an existing visualization, displayed on the \emph{query canvas} (Figure~\ref{zvOverview}B). \zvpp performs shape-matching between the queried pattern and other possible visualizations and returns a ranked list of visualizations that are most similar to the queried pattern, displayed in the \emph{results panel} (Figure~\ref{zvOverview}C). At any point during the analysis, analysts can adjust various system-level settings through the \emph{control panel} (Figure~\ref{zvOverview}D) or browse through the list of \emph{recommendations} provided by \zvpp (Figure~\ref{zvOverview}E). Our \zvpp system is open source and available at: \url{github.com/[Annonymized for Submission]}.
}

%!TEX root = main.tex
\section{The Sensemaking Model for VQSs\label{sec:sensemaking}}
\change{
  To convey how the features in \zvpp addresses the analytical needs posed by each domain, we organize our PD findings into a sensemaking framework for VQSs. In this section, we first describe the space of problems addressable by VQSs. Then, as shown in Figure~\ref{fig:taxonomy}, we develop a taxonomy for organizing VQS functionalities into three sensemaking processes. From top to bottom, we first describe the design objectives of each sensemaking process, then we outline the design challenge addressed by each of the functional components that supports the sensemaking process. The mapping between specific \zvpp features and these functional components and sensemaking processes can be found in Table~\ref{bigfeaturetable}.
}
 %Starting from the bottom level of the taxonomy, we first describe what each component in our taxonomy encompasses, then we proceed onto the upper level of the taxonomy, .
%We first describe features that we incorporated into our enhanced VQS, \zvpp, thematically organized by components (grouping features in the bottom-most level to components in the second level of Figure~\ref{fig:taxonomy}).
%collaboratively-designed
%Next, we describe features that we incorporated into our enhanced VQS, \zvpp, thematically organized by component. Then, we introduce a taxonomy for organizing these components into three sensemaking processes, spanning different problem areas that VQSs are aimed to solve.
%\change{In this section, we will first introduce a taxonomy for organizing these components into}
%Based on feature requests and discussion with our participants, we incorporated key features missing in our original VQS.
%From these discussion and analysis of past VQSs, we identify nine components of VQSs, described below. T
% Along with analysis of past literature, we develop a taxonomy of key functionalities in VQSs.
% novel contribution on  ---
% contribute to holistic understanding on how sensemaking --- in VQS.
% study on how users
% Implication ---
% •	What types of questions/ dataset/ problem challenges are asked to VQS or can be addressed by VQS? (S3)
% •	What kind of features needs to be designed to address these challenges (S4 PD)
%We employed participatory design with our scientists to incorporate key features missing in our original VQS, and unaddressed in their existing workflows. From these discussion and analysis of past VQSs, we identify nine components of VQSs, described below.
\subsection{Characterizing the Problem Space for VQSs}
%Based on example use cases and feature components from participatory design, we further characterize the design space of VQSs. further characterize three sensemaking process within the problem space of VQSs.
Given our earlier description of VQS features organized into components, we now introduce the three sensemaking processes by characterizing how they fit into different problem areas that VQSs are aimed to solve. Visual querying often consists of searching for a desired pattern instance (Z) across a visualization collection specified by some given attributes (X,Y). \change{Correspondingly, }we introduce two axes depicting the amount of information known about the visualized attribute and pattern instance.%, as shown in Figure~\ref{2dmodel}.
%(e.g., only interested in patterns related to a specific gene)
\par Along the \textbf{pattern instance} axis,
the visualization that contains
the desired pattern may already be \texttt{known} to the analyst,
exist as a pattern \texttt{in-the-head} of the analyst,
or completely \texttt{unknown} to the analyst.
In the \texttt{known} pattern instance region (Figure~\ref{2dmodel} grey), visualization-at-a-time systems such as Tableau,
where analyst manually create and examine each visualization one at a time,
is more well-suited than VQSs, since analysts can directly work with the selected instance without the need for visual querying.
Inspired by Pirolli and Card's information
foraging framework~\cite{Pirolli}, which distinguishes
between information processing tasks that are \textit{top-down}
(from theory to data) and \textit{bottom-up} (from data to theory),
we define \textit{top-down pattern specification} as the search-oriented paradigm where analysts query based on their
in-the-head pattern (Figure~\ref{2dmodel} blue) in a fixed collection.
On the other hand, in the realm of \textit{bottom-up
data-driven inquiry} (Figure~\ref{2dmodel} green),
the pattern of interest is unknown\change{, external
to the user, }and must be driven by recommendations
or queries that originate from the data (or equivalently, the visualization).
As we will discuss later, this process is crucial
but underexplored in past work on VQSs.
%analysts often do not start with a known pattern instance. T
\par The second axis, \textbf{visualized attributes},
depicts how much the analyst
knows about which X and Y axes
they are interested in visualizing.
In both the astronomy and genetics use cases,
as well as past work in this space,
data was in the form of a time series
with \texttt{known} visualized attributes.
In the case of our material science participants,
they wanted to explore relationships between different
X and Y variables.
In this realm of \texttt{unknown} attributes,
context creation (Figure~\ref{2dmodel} yellow) is
essential for allowing users
to pivot across different visualization collections.%subspaces. %Most past VQSs assume that the analyst has a desired pattern in-the-head that could be conveyed through visual specification, such as a sketch.%---i.e., setting the stage for bottom-up or top-down processes---\dor{this clause is awkward?}
\begin{figure}[h!]
  \centering
  \includegraphics[width=0.9\linewidth]{figures/2dmodel.pdf}
  \caption{The problem space for VQSs is characterized by how much the analyst knows about the visualized attributes and the pattern instance. Colored areas highlight the three sensemaking processes in VQSs for addressing these characteristic problems. While prior work has focused solely on use cases in the blue region, we envision opportunities for VQSs beyond this to a larger space of use cases covered by the yellow and green regions.}
  \label{2dmodel}
  \vspace{-10pt}
\end{figure}
\subsection{Design Goals for the Sensemaking Processes}
After understanding how each sensemaking process fits into the problem space \change{addressable} by VQSs, we further explore the design objectives and challenges in supporting each sensemaking process, grounded in our collaborative design experience. % based on the taxonomy orgaby developing a taxonomy for organizing the aforementioned components.% fits into the paradigms of sensemaking in VQSs, as shown in Figure~\ref{fig:taxonomy}. In particular, we will describe the main form of inquiry addressed by each paradigm\cut{(\textit{what, where, which})}, its characteristic use case, and design challenges in supporting these paradigms.
% \par Drawing from our participatory design experience, evaluation study, and literature review in this space, we design a taxonomy for understanding the key functionalities in VQSs. In Figure~\ref{fig:taxonomy}, we show how each use cases makes use of the different features in \zv, then we organize the features into key components for VQSs, which belongs to one of the three paradigms in the VQS design space.%, effectively moving rightwards to the gray area in Figure~\ref{2dmodel}, where the pattern instance is known.
\boldpara{Top-down Pattern Specification} begins with the user's intuition
about how their desired patterns should look like based on `theory', including visualizations from past experience or abstract conceptions based on external knowledge. The goal of top-down pattern specification is to address the \textit{which} question of visual sensemaking: \textit{which pattern instance exhibits this pattern?} Based on this preconceived notion of what to search for, the design challenge is to translate the query in the
analyst's head to a query executable by the VQS.
\change{In the Figure~\ref{fig:taxonomy} taxonomy},
this includes both components for specifying the pattern,
as well as controls governing the underlying
algorithm of how shape-matching is performed.
For example, A1 knows intuitively
what a supernovae pattern looks like
and the detailed constraints on the shape,
such as the width and height of the peak
or the level of error tolerance for defining a match.
He can search for transient patterns through sketching,
select the option to ignore differences
on the x axis, and changes the similarity metric for flexible matching.  %The design challenge of top-down pattern specification is to ----- enable users to How to translate the in-the-head query to visual query and how matching is done.
\boldpara{Bottom-up data-driven inquiry} is
a browsing-oriented sensemaking process
that goes from data to theory to
addresses the \textit{what} questions
in the sensemaking process.
% While the usage of each querying feature may vary from one participant to the next, generally, result querying and pattern upload are considered bottom-up approaches that go from data to theory by enabling users to query via examples of known visualizations. Bottom-up data-driven inquiries
 For example, genetics participants do not
 have a preconceived knowledge of what to search
 for in the dataset.
 They were mostly interested in
 \textit{what types of patterns exist in the dataset}
 through representative trends, as a means to
 jumpstart further queries. %The goal of data-driven inquiry is to move towards the blue area in Figure~\ref{2dmodel} to help analysts gain more information about patterns of interest in-the-head.
% notion of what the pattern looks like
The design challenge include developing
the right set of `stimuli' that could
provoke further data-driven inquiries,
as well as low-effort mechanisms to search via these results.
\boldpara{Context Creation} addresses the \textit{where}
question of sensemaking by enabling analysts
to navigate across different parts of the visualization
collection to learn about \textit{where \change{in the dataset do} the patterns of interest lie}.
For example, material scientists often do not start
with a pattern in-the-head, but recognize salient
trends such as inverse correlation or linear correlation.
They switch between different visualized attributes or dynamic
classes to study their data from alternative perspectives.
The design challenge of context creation is to develop
features that act as a `lens': navigating users to desired data subsets,
visualizing and comparing how the data changes between the different lenses, and ensuring that context is dynamically reflected across other VQS functionalities.
\par\noindent The three aforementioned sensemaking processes are akin to the well-studied sensemaking paradigms of search, browse, and faceted navigation on the Web~\cite{Hearst2009,Olston2003}. Due to each of their advantages and limitations given different information seeking tasks, search interfaces have been designed to support all three complementary acts and transition smoothly between them to combine the strength of all three paradigms. \change{Similarly for VQSs, our main design objective in developing \zvpp is to integrate all the three sensemaking in the same system. As we discover in the evaluation study in the following section, this integration encourages and accelerates the process of visualization discovery.}

\change{
  \subsection{Functional Components of VQSs\label{sec:component}}
    Here, we discuss the motivation for each functional component in the lower-level of our Figure~\ref{fig:taxonomy} taxonomy and how they address specific challenges posed by the problem and dataset characteristics from each domain.
    \boldpara{Pattern Specification} interfaces allow users to submit exact descriptions of a pattern query with the VQS returning a list of most similar matches. This is useful when the dataset contains \emph{large numbers of potentially-relevant pattern instance}.
    Since it is often difficult to sketch precisely, additional characteristics of the pattern query can be used to further winnow our undesired matches (e.g., pattern query expressible in a functional form, or has specific shape characteristics).
    \boldpara{Match Specification} addresses the well-known problem in VQSs where pattern queries are imprecise~\cite{correll2016semantics,Holz2009,Eichmann2015} by allowing users to clarifying how matching should be performed.
    Match specification is useful when the dataset is \emph{noisy} (i.e., containing large numbers of false-positives that could be matched). When the pattern query additional constraints, adjusting match specification to prune away these false-positives help reveal true candidates.
    \boldpara{View specification} settings alter the specifications for all of the candidate visualizations being explored on the VQS. The ability to work with different collections of visualizations is useful when the dataset is \emph{multidimensional} and the axes of interest is unknown. Modifying the view specification offers analysts different perspectives on the data to locate visualization collections of interest.
    \boldpara{Slice-and-Dice} empowers users to navigate and compare different collections of visualizations constructed from different portions of the data. Slice-and-dice is useful when the dataset has \emph{large numbers of non-visualized attributes} that may be related to the visualized attributes (e.g, geographical location may influence the time series pattern for housing prices). Analysts can either make use of pre-existing knowledge regarding these `support' attributes to navigate to a data region that is more likely to contain the desired pattern (e.g., filtering to popular cities such as New York to find expensive houses) or discover unknown patterns and relationships between different data subsets (e.g., housing prices is lower around winter than compared to summer).% by gaining a better understanding of characteristic patterns in particular data region.
    \boldpara{Result querying} allows users to submit a query based on the results, essentially asking for patterns similar to the selected data pattern. Typically, analyst associate selected visualizations with some \emph{semantic or visual properties} and make use of results querying to understand characteristic properties of similar instances.
    \boldpara{Recommendation} displays visualizations that may be of interest to users based on the data context. Representative trends and outliers are useful when a \emph{small number of common patterns} is exhibited in the dataset. Understanding \emph{characteristic} patterns in dataset can help analysts discover other pattern queries of interest to jumpstart further queries.
  % % In this section, we first describe a model to help characterize the design space for VQS based on the analytical workload and usage patterns from different use cases. Then, we present design challenges related to each of the process.
  % \begin{table}
  % 		\centering
  %     \includegraphics[width=0.8\linewidth]{figures/related_works_table.pdf}
  %     \caption{Table summarizing whether key \change{functional components} (columns) are covered by past systems (row), indicated by checked cells. Column header colors blue, orange, green represents three sensemaking process (top-down querying, search with context, and bottom-up querying)\cut{ described in Section~\ref{sec:pd_findings}}. The heavily-used, practical features in our study for context-creation and bottom-up inquiry is largely missing from prior VQSs.}
  %     \label{table:relatedwork}
  %     \vspace{-10pt}
  % \end{table}
}
  \begin{figure*}[ht!]
    \centering
    \includegraphics[width=0.9\linewidth]{figures/taxonomy.pdf}
    \caption{Taxonomy of functionalities in VQSs. Each of the three sensemaking process is broken down into key functional components in VQSs. \change{We list the types of questions addressed by each component from a system's perspective.}} %, which is instantiated as features in \zvpp.}% The bottom-most layer connects the use cases features that have practical or envisioned usage based on the evaluation study.}
    \label{fig:taxonomy}
  \end{figure*}

%!TEX root = main.tex
\section{Evaluation Study Findings\label{sec:eval_findings}}
% \begin{figure*}[t!]
% \minipage{0.6\textwidth}
%   \includegraphics[width=\linewidth]{figures/evalstudytimeline.pdf}
%   \caption{Timeline of event code and component usage, with every timepoint as an event on the x axis. For clarity, we hide most of the event coding labels other than the insight labels. Black vertical tick indicates a session break, signaling the beginning of a new line of inquiry.}\label{fig:evalstudytimeline}
% \endminipage\hfill
% \minipage{0.4\textwidth}
%   \includegraphics[width=0.8\linewidth]{figures/PENcoding.pdf}
%   \caption{Heatmap of features categorized as practical usage (P), envisioned usage (E), and not useful (N).  \techreport{We find that participants preferred to query using bottom-up methods such as drag-and-drop over top-down approaches such as sketching or input equations. Participants found that data faceting via filter constraints and dynamic class creation were powerful ways to compare between subgroups or filtered subsets. The columns are arranged in the order of subject areas and the features are arranged in the order of the three foraging acts.}}\label{fig:feature_heatmap}
% \endminipage
% \end{figure*}
Based on audio, video screen capture,
and click-stream logs recorded
during our evaluation study,
we performed thematic analysis via open coding
and categorized every event with a coded label.
Event codes included specific feature usage,
insights,
provoked actions, confusion,
request for functionalities unaddressed
by the system, and use of external tools\change{, detailed in Appendix~\ref{apdx:studydetails}}. To characterize the usefulness
of each feature, we further labeled whether each
feature was useful to a particular participant's analysis.
A feature was deemed \textit{useful}
if the feature was either used in a sensible
and meaningful way during the study,
or has envisioned usage outside of the constrained
time limit during the study
(e.g., if data was available or downstream analysis was conducted).
We derived these labels from the study transcript
to circumvent self-reporting bias,
which can often artificially inflate
the usefulness of the feature under examination.
In this section, we will apply our thematic analysis results to understand how each sensemaking process occurs in practice.%real-world analytic tasks.}
%\agp{Can't parse the previous sentence}
%categorized the features based on whether there was a sensible usage of the feature
% into one of the three usage types based on how each feature was used during the study:
% \begin{denselist}
%     \item Practical: Features used in a sensible and meaningful way.
%     \item Envisioned: Features which could be used practically if the envisioned data was available or if they conducted downstream analysis, but was not performed due to the limited time during the study.
%     \item Not useful: Features that are not useful or do not make sense for the participant's research question and dataset.
% \end{denselist}
% \par Given these initial findings, we further investigated where the `sketch'
% Our interactions with the scientists showed that different modalities for inputting a query can be useful for different problem contexts. In addition, the three paradigms of sensemaking described earlier are not mutually exclusive. In fact, we find that participants often construct a central workflow focused on features from one of the main paradigms and interleave variations with the feature usage from the two other paradigms as they iterate on the analytic task. As shown in Figure~\ref{fig:usagefreqbysubject}, the central paradigm adopted by each use case is tightly coupled with characteristics of the analytic challenges presented by each subject area.
% interplay
% Next, we will describe some of the design principles (DP) based on our study findings.
%focus on understanding the design space of VQSs and highlight the takeaways of our study.%developing a process model and design guideline for insight formation in VQSs and divert our thematic analysis of how VQSs fit into the context of an analysis workflow to our technical report.% These observation inform our ----- search-browse paradigm
% \subsubsection{Discovery of Real-world insights}
% \par Our participants' original workflow often required them to compare between many visualizations manually through separate analysis and visualization steps. Three of the participants cited that this segmented analyze-then-visualize workflow was one of their chief bottlenecks. The cognitive overhead from the segmented workflow made them more hesitant to visualize the results of different parameters and data operations, as A2 noted:
% \begin{quote}
% The quick visualization is something that I could not do on my current framework. I could not query as fast as you do; I need to wait for it, plot, and then compare. Every time I plot, I need to define subplots for 12 visualizations, then its slower. That's the reason why I sometimes plot less, and I rely more on the statistics from the likelihood tests. Sometimes I plot less than I really should be doing.
% \end{quote}
% The ability to rapidly experiment with large numbers of hypotheses in real time is a crucial step in the agile creative process in helping analysts discover actionable insights~\cite{Shneiderman2007a}. Five out of nine participants discussed how the dynamic, interactive update of the visualization in \zv was the main advantage for using VQSs over their original workflow.
% \begin{figure}[h!]
%   \includegraphics[width=\linewidth]{figures/usagefreqbysubject.pdf}
%   \caption{The number of times each component is used during the evaluation study, broken down by subject areas.}\label{fig:usagefreqbysubject}
% \end{figure}
\subsection{The Ineffectiveness of Sketch}
% \subsection{DC3: Closing the loop in VQS sense-making cycle with bottom-up data-driven inquries}
\par To our surprise,
despite the prevalence of sketch-to-query
systems in the literature, \techreport{Figure \ref{fig:feature_heatmap} shows that} only two out of our nine participants
found it useful to directly
sketch \change{desired} pattern onto the canvas. %Overall, bottom-up querying via drag-and-drop was more intuitive and more commonly used than top-down querying methods, such as sketching or input equations.
The \change{reason why most} participants
did not find \change{sketching useful} was that
they often do not start their analysis with a specific pattern in mind.
Instead, their intuition about what to query is derived
from other visualizations they encounter
during exploration, in which case it makes
more sense to query using those visualizations
as examples directly (e.g., by dragging and dropping
that visualization onto the \change{canvas to submit the query}).
Even if a user has a pattern in mind,
translating that pattern into a sketch is often hard
to do. For example,
A2 wanted to search for a highly-varying signal
enveloped by a sinusoidal pattern indicating
planetary rotation \includegraphics[width=2\baselineskip,keepaspectratio]{figures/impossible_sketch.png}, which is hard to draw by hand.
\begin{figure}[h!]
  \includegraphics[width=\linewidth]{figures/the_origins_of_sketch.pdf}
  \vspace{-5pt}
  \caption{The number of times a pattern query originates from one of the workflows. We find that pattern queries are \change{far} more commonly generated via bottom-up than top-down processes.}\label{fig:origins_of_sketch}
  \vspace{-5pt}
\end{figure}
\par Given these initial findings,
we further investigated where the pattern
on the canvas typically originates, as presented in Figure~\ref{fig:origins_of_sketch}.
Pattern queries can be generated by
either top-down (sketching) or
bottom-up (drag-and-drop) processes,
driven by various different querying intentions.
Within top-down processes,
a pattern query could arise
from users directly sketching
a new pattern (sketch-to-query)
or by modifying an existing sketch (sketch-to-modify). For example, M2 first sketched a pattern
to find solvent classes with anticorrelated
properties without much success in returning a desired match.
%However, the sketched query did not return visualizations of interest.
So he instead dragged and dropped one
of the peripheral visualizations similar
to his desired visualization and then smoothed
out the noise in the visualization yielding
a straight line,
as shown in Figure \ref{query_modification} (left).
M2 repeated this workflow twice in separate
occurrences during the study and
was able to derive insights from the results.
Likewise, Figure~\ref{query_modification} (right)
illustrates how A3 first picked out a regular pattern
(suspected star spot), then modified it slightly
so that the pattern looks more irregular (to find pulsating stars).
%Within these actions, there can be different intentions behind the sketch. While all visualizations that could be drag-and-dropped must come from the result or recommendation pane, a query can come from a particular object that the participant is interested in or simply through peripheral browsing of visualization results.%, described in the next section.
%\par The latter case is also supported by the
%\par There are also many unexpected use cases where sketching was simply used as a mechanism to modify an existing pattern query.
%Likewise, A3 was interested in pulsating stars that looked similar to stellar hotspots in terms of its dramatic amplitude fluctuations, but differ in that their patterns exhibited irregularities. Figure \ref{query_modification} (right) showed how she first picked out a regular pattern (suspected star spot), then modified it slightly so that the pattern looks more irregular.
%Likewise, A3 was interested in pulsating stars characterized by dramatic changes in the amplitudes of the light curves. During the search, hotspots on stellar surfaces often show up as false positives as they also result in dramatic amplitude fluctuations, but happen at a regular intervals. In the VQS, A3 looked for patterns that exhibits amplitude variations, but also some irregularities. As shown in Figure \ref{query_modification} (right), she first picked out a regular pattern (suspected star spot), then modified it slightly so that the pattern looks more irregular.\par While all visualizations that could be drag-and-dropped must come from the result or recommendation pane, a query can come from a particular object that the participant is interested in or simply through peripheral browsing of visualization results.
As described in the following section,
bottom-up pattern queries can come from either
the ranked list of results,
recommendations, or by selecting a
particular object of interest as a drag-and-drop query.
Figure~\ref{fig:origins_of_sketch} shows that
bottom-up processes are more common
than top-down processes for generating a pattern query.
\begin{figure}[h!]
    \centering
    \includegraphics[width=\columnwidth]{figures/QueryModificationBySketch.pdf}
    \caption{Canvas traces from M2 (left) and A3 (right) during the study demonstrating query modification. The original drag-and-dropped query is shown in blue and sketch-modified queries in red.}
    \label{query_modification}
    \vspace{-10pt}
\end{figure}
\par The lack of practical use of top-down pattern
specification is also reflected in the fact
that none of the participants queried using an equation.
In both astronomy and genetics, the visualization patterns
resulting from complex physical processes
that could not be written down as an equation analytically.
Even in the case of material science when analytical
relationships do exist, it is challenging to formulate
functional forms in \change{a prescriptive} manner.
% Despite functional fitting being common in scientific data analysis, Figure \ref{feature_heatmap} shows that
% . However,
\par Our findings suggest that while sketching
is an useful construct for people to express their queries,
\emph{the existing ad-hoc, sketch-only model for VQSs
is insufficient without data examples
that can help analysts jumpstart their exploration}.
In fact, from Figure~\ref{fig:origins_of_sketch},
we can see that sketch-to-query was only used
8 times, while the remaining \change{querying modalities} were used 29 times altogether,
more than three times as much as sketch-to-query.
This finding has profound implications
on the design of future VQSs, since Table~\ref{table:relatedwork}
\change{suggests} that past work have primarily focused
on optimizing top-down process components,
without considering how useful these features
are in real-world analytic tasks.
%missing out largely on the key components in the other two paradigms \cut{(indicated by the absence of green features on the right hand side of the table)}.
We suspect that these limitations
may be why existing VQSs are not commonly adopted in practice. %This result points to a need for ----- in future VQSs. %This, however, points to an exciting direction for sketching interface in VQSs for developing advanced drawing and modification tools that enable more precise visualization query specification.} %ed coverage in addressing different types of analytics use cases
%For instance, material science discovered a known inverse relationship during e xploration
%Which is really interesting. Which is something that we observed experimentally also. That is an interesting insight right htere. This seems to suggest that there is a fundamental issue in if you want to try to get better on this axis, and get as low as possible, you lose out on the other axis.
%once they see it they know it but they don't know beforehand

\subsection{Context Creation and Bottom-up \change{Applications}}%Approaches}
\par As alluded to earlier,
\emph{bottom-up data-driven inquiries
and context creation are far more commonly
used than top-down pattern specification
when users have no desired patterns in mind},
which is typically the case for exploratory data analysis.
In particular, we find that top-down approaches
were only useful for 29\% of the use cases,
whereas it was useful for 70\% of the use cases
for bottom-up approaches and 67\%
for context creation\footnote{\change{See Appendix~\ref{apdx:studydetails} for details on how this measure is computed.}}. We now highlight some of the exemplary workflows demonstrating the efficacy of the latter two sensemaking processes.
%number of features labeled as useful divided by the product of total number of features and total number of users}
%%The prevalence of bottom-up approaches not only point to the need for result querying, but also providing recommendations for users without desired patterns in mind.
\par As shown in Figure~\ref{fig:origins_of_sketch},
the most common use of bottom-up querying
is via recommended visualizations. For example, G2 and G3 identified that
the three representative patterns
recommended in \zvpp corresponded
to the same three groups of genes discussed
in a recent publication~\cite{Gloss2017}:
induced genes (profiles with expression levels \change{going up} \includegraphics[width=2\baselineskip,keepaspectratio]{figures/induced.png}),
repressed genes (\change{starting high then decreasing} \includegraphics[width=2\baselineskip,keepaspectratio]{figures/repressed.png}),
and transients (\change{rising first then dropping at another time point} \includegraphics[width=2\baselineskip,keepaspectratio]{figures/transient.png}). The clusters provoked G2 to generate a hypothesis
regarding the properties of transients:
\textit{``Is that because all the transient groups
get clustered together, or can I get sharp patterns
that rise and ebb at different time points?''}
To verify this hypothesis, G2 increased the parameter controlling the number of clusters and noticed that the clusters
no longer exhibited the clean,
intuitive patterns he had seen earlier.
G3 expressed a similar sentiment and proceeded
by inspecting the visualizations
in the cluster via drag-and-drop.
He found a group of genes that all transitioned
at the same timestep, while others transitioned
at different timesteps.
\techreport{G3 described the process of using
VQSs as doing ``detective work'' that provoked
him to generate further scientific hypotheses
as well as data actions.}
\par By browsing through the ranked list of
results in \zvpp, participants were also able to
gain a peripheral overview of the data
and spot anomalies during exploration.
For example, A1 spotted time series
that were too faint to look like stars
after applying the filter CLASS\_STAR=1,
which led him to discover that all stars
have been mislabeled with CLASS\_STAR=0 as 1 during data cleaning.
%This includes inspecting the top-most similar visualizations that lie in the queried cluster. and finding visualizations that are similar to an object of interest that exhibits a desired pattern. %. After browsing through a series query results and checking with an external database, he concluded that
 %We found that geneticists often gain their intuition about the data from the recommended representative trends. One example of rapid insight discovery
%the dataset had been incorrectly labelled with all the stars with CLASS\_STAR=0 as 1 during data cleaning.
%Examples of how recommended trends can provoke further insightful actions comes from G2 and G3, who identified that the three representative patterns shown in \zvpp---induced genes (profiles with expression levels staying up), repressed genes (started high but went down), and transients (go up and then come down at different time points)---corresponded to the same three groups of genes discussed in a recent publication~\cite{Gloss2017}.
\techreport{
	\begin{figure}[h!]
	  \includegraphics[width=0.8\linewidth]{figures/PENcoding.pdf}
	  \caption{Heatmap of features categorized as practical usage, envisioned usage, and not useful. \techreport{We find that participants preferred to query using bottom-up methods such as drag-and-drop over top-down approaches such as sketching or input equations. Participants found that slicing and dicing via filter constraints and dynamic class creation were powerful ways to compare between subgroups or filtered subsets. The columns are arranged in the order of subject areas and the features are arranged in the order of the three sensemaking processes.}}
	  \label{fig:feature_heatmap}
	\end{figure}
}
% \subsection{Enriching Search with Context}
\par Past studies in visual analytics
have shown that it is important to design features
that enable users to select relevant subsets of data~\cite{Amar2005,Heer2012}.
Context creation in VQSs enables users to change the `lens'
by which they look through the data
when performing visual querying,
thereby creating more opportunities
to explore the data from different perspectives.
All participants found at least
one of the features in context creation to be useful.
%We designed two dynamic faceting features coupled with coordinated views that enabled users to specify subsets of data they are querying on and see immediate changes updated in the query, representative, and outlier results.
%either envisioned a use case or utilized features in the context creation paradigm to explore and compare subsets of their data.
%ven though the filtering step could be easily done with an external tool and reloaded into \zv, filtering on-the-fly was a powerful way to dynamically test his hypothesis. I
\par Both A1 and A2 expressed that
interactive filtering enabled
them to test conditions and tune values
that they would not have otherwise modified,
effectively lowering the barrier between
the iterative hypothesize-then-compare cycle during sensemaking.
% echoing our previous finding that segmented workflow prevents extensive exploration.
During the study, participants used filtering
to address questions such as:
\textit{Are there more genes similar
to a known activator when we subselect
only the differentially expressed genes?} \techreport{\texttt{DIFFEXP=1} }(G2) or \textit{Can I find more supernovae candidates if I query only on objects that are bright and classified as a star?} \techreport{\texttt{flux\textgreater10 AND CLASS\_STAR=1} }(A1). Three participants had also used filtering as a way to pick out individual objects of interest to query with, as shown in Figure~\ref{fig:origins_of_sketch}. For example, G2 set the filter as gene=9687 and explained that since ``\textit{this gene is regulated by the estrogen receptor, when we search for other genes that resemble this gene, we can find other genes that are potentially affected by the same factors.}''
\par While filtering enabled users to
narrow down to a selected data subset,
dynamic class creation enabled users to compare
relationships between multiple attributes and subgroups of data.
For example, M2 divided solvents in the database
into eight different categories based on voltage properties,
state of matter, and viscosity levels,
by dynamically setting the cutoff values
on the quantitative variables to create these classes.
By exploring these custom classes, M2 discovered that the relationship between viscosity and lithium solvation energy is independent of whether a solvent belongs to the class of high voltage or low voltage solvents\change{. He cited }that dynamic class creation was central to learning about this previously-unknown attribute properties:
\begin{quote}
All this is really possible because of dynamic class creation, so this allows you to bucket your intuition and put that together. [...] I can now bucket things as high voltage stable, liquid stable, viscous, or not viscous and start doing this classification quickly and start to explore trends. [...] look how quickly we can do it!% Quite good!
\end{quote}
%Context creation is a useful ---- despite the --- pattern instance. Filtering still useful
%\par Participants employed \emph{a mix of bottom-up and top-down approaches when faceting through data in VQS}, including narrowing the search space based on some intuition about a phenomena, selecting individual visualizations, or specifying high-level groupings to compare and query with.
\subsection{\change{Sensemaking in VQS Workflows}}
Given our observations so far as to
how participants
make use of each sensemaking process in practice,
we further investigate the interplay
between these sensemaking processes
in the context of an analysis workflow. %interplay with each other dynamically i% - Bottom up and context creation much more common than top-down. Stats \%. Brief Examples of each (How they are used in practice).
% - BUT All three process are equally important.
% - participants can go from one to the next and there is no single progression (e.g. context --> bottom -up --> top-down).
% - Both the PageRank score (how important/“central” is the state is to the analysis?), raw occurrence of each state (how frequently is a feature categorized as part of the state used?) and the normalized self-directed edge score (how much user stays in that state?) coincide with what each subject area focuses on.
% We first examine the popularity of each sensemaking process based on how frequently they are used in the study. Figure~\ref{fig:feature_heatmap} show that features categorized as bottom-up (useful for 70\% of the use cases) and context creation (67\%) are much more useful compared top-down features (29\%). [Examples of Bottom up]. [Examples of Context Creation].
%Despite differing in levels of usage, each sensemaking process fulfills a central role in participants' analysis.
% illustrates the state transitions computed based on event sequences from the evaluation study. %stay in the same state.
%Figure~\ref{fig:taxonomy},
The event sequences from the evaluation study
consist of labels describing when specific features were used.
Using the taxonomy in \change{Section~\ref{sec:pd_findings}}, we map each usage of a feature
to one of the three sensemaking processes.
Each participant's event sequence
is divided into sessions,
each indicating a separate lines of inquiry
during the analysis.
Based on these event sequences---one for each session,
we compute the aggregate state transition probabilities
(shown as edge weights in Figure~\ref{fig:transition})
to characterize how participants from each domain
move between different sensemaking processes.
For example, in material science,
bottom-up exploration
leads to context creation 60\% of the time
and to top-down pattern-specification
the rest of the time.
Self-directed edges indicate the probability that the participant
would continue with the same type of sensemaking process.
For example, when an astronomer performs top-down pattern specification,
it is followed by another top-down specification
64\% of the time and context creation the rest of the time,
but never followed by a bottom-up processes.
This high self-directed transition probability
reflects how astronomers often need to iteratively
refine their top-down query through pattern
or match specification when looking for a specific pattern. %when A1 looks for supernovae, he needs to iteratively refine his top-down query through pattern or match specification interfaces. %He could also chose to refine ----- , control --- to issue the desired query.
%Each event sequence is separated by labeled session breaks signaling the beginning of a new line of inquiry. The
\begin{figure}[h!]
  \includegraphics[width=\linewidth]{figures/markov_transition.pdf}
  \caption{Markov models computed based on the evaluation study event sequences, with edges denoting the probability that a participant in the particular domain will go from one sensemaking process to the next. Nodes are scaled according to the eigenvector centrality, which represents the percentage of time users would spend in a particular \change{sensemaking process in steady state}.}\label{fig:transition}
\end{figure}
% Similar to the sensemaking model proposed by Pirolli and Card~\cite{Pirolli}, the ---- sensemaking loop representing iterative process
% ---- highlights how the two newly discovered VQS sensemaking process in this paper are essential for `closing the loop' between the sensemaking acts in VQSs. %, equally important
% both the browsing-act through recommendations and performing search via these results are
%These examples show that both the browsing-act through recommendations and performing search via these results are
%The three sensemaking ----- not mutually exclusive, participants can go from one to the next and there is no single progression (e.g. context --> bottom -up --> top-down). Iterative blah blah. three process are equally important.
%The three paradigms of sensemaking described earlier are not mutually exclusive.
%Different sensemaking processes can be useful for different problem contexts.
\par To study how important each sensemaking process
is for participant's overall analysis,
we compute the eigenvector centrality of each graph,
displayed as node labels in Figure~\ref{fig:transition}.
These values represent the percentage of time the participants
spend in each of the sensemaking processes
when the transition model has evolved to a steady state~\cite{pierre2011}.
Given that nodes in Figure~\ref{fig:transition}
are scaled by this value, in all domains,
we observe that there is always a prominent node
connected to two less prominent ones---but it is also clear
that all three nodes are essential to all domains.
Our observation demonstrates how \emph{participants
often construct a central workflow
around a main sensemaking process
and interleave variations with the two other processes
as they iterate on the analytic task}
.
For example, material scientists focus
on context creation 56\% of the time,
mainly through dynamic class creation,
followed by bottom-up inquiries (such as drag-and-drop)
 and top-down pattern specification (such as sketch modification).
%through dynamic classes than top-down pattern specification. %astronomers focus largely on performing top-down pattern specification, while filtering on the visualization space.
The central process adopted by each domain
is tightly coupled with characteristics
of the analytic challenges associated with their subject area\change{, as illustrated in Appendix Table~\ref{science_task}.}%in Section~\ref{sec:pd_findings}}.
For example, without an initial query in-the-head,
geneticists relied heavily on bottom-up querying
through recommendations to jumpstart their queries.
%Despite the differing levels of usage from each subject area, we learn that \textit{each sensemaking process fulfills a central role in participants' analysis to address their high-level research objectives}.
% \agp{maybe point to figure?}.
\par The \change{Markov} transition model exemplifies how participants
adopted a diverse set of workflows
based on the unique set of research questions they brought to the study.
The bi-directional and cyclical nature
of the transition graphs in Figure~\ref{fig:transition} highlight how the three sensemaking processes do not simply follow a linear progression, going from unknown to known \change{in the Figure~\ref{2dmodel} problem space}.%pattern instance and visualized attributes
 Instead, the high connectivity of the transition model illustrates how these three equally-important processes form a sensemaking loop\change{, representing} iterative acts of dynamic foraging and hypothesis generation. This flexibility is enabled by the diverse set of potential workflows that could be constructed in \change{an integrative} VQS like \zvpp, for addressing a wide range of analytical inquiries.%single-directional%. The VQS sensemaking loop%full-fledged
\subsection{Limitations}
\par Although evidence from our evaluation study
suggests that direct sketch is inefficient,
we have not performed controlled studies
with a sketch-only system as a baseline to validate this hypothesis.
The goal of our study is to uncover qualitative insights
that might reveal why VQSs are not widely used in practice;
further validation of specific findings is out of the scope of this paper.
While we have generalized our findings by employing
three different and diverse domains (see Figure~\ref{fig:transition}),
our case studies have so far
been focused on scientific data analysis,
as a first step towards greater adoption of VQSs.
Other potential domains that could benefit from VQSs include:
financial data for business intelligence,
electronic medical records for healthcare,
and personal data for ``Quantified Self''.
These different domains may each pose different
 sets of challenges unaddressed
 by the findings in this paper,
 pointing to a promising direction for future work.
% preliminary
% While our results indicate that sketch is inefficient, we have not performed any controlled studies with sketch-only system as a baseline comparison to evaluate this hypothesis. support by evidence from our study
% - no A/B testing evaluate our hypothesis, direct comparison with a sketching. Our results points to an issue in this, further controlled studies required to validate these.
% , promising direction for future work.
% - study focussed on scientific workflow not BI or others --> towards social data , Quantified self
% - only 3 domain area breadth, but more than any existing work

\input{07-conclusion}
\bibliographystyle{eg-alpha}
\bibliography{reference}
\newpage
\appendix
% \npar\textbf{{\huge Appendix}}
\change{\npar In Appendix A, we first describe additional details about the participatory design process, as well as domain-specific artifacts collected from contextual inquiry. Next, in Appendix B, we articulate the space of problems amenable to VQSs and describe how the sensemaking processes introduced in Section~\ref{sec:sensemaking} into different parts of the problem space. Finally, in Appendix C, we provide supplementary information regarding our analysis methods and results.}
\section{Artifacts from Participatory Design\label{apdx:pdartifact}}
\change{Our collaboration with participants is illustrated in Figure~\ref{timeline}, where we began with an existing VQS (\zv, as illustrated in Figure~\ref{oldZV}) and incrementally incorporated features, such as dynamic class creation (Figure~\ref{dcc}), throughout the participatory design process.}
\begin{figure}[h!]
	\centering
	% \captionsetup{justification=centering,margin=2cm}
	\vspace{-10pt}
	% \includegraphics[width=6in]{figures/timeline.pdf}
  \includegraphics[width=\linewidth]{figures/timeline.pdf}
	\caption{Timeline for progress in participatory design studies.}
	\label{timeline}
	\vspace{-10pt}
\end{figure}
\begin{figure}[h!]
	\centering
	\includegraphics[width=0.9\linewidth]{figures/oldZV_nozql.pdf}
	\caption{The existing \zv prototype allowed users to sketch a pattern in (a), which would then return (b) results that had the closest Euclidean distance from the sketched pattern. The system also displays (c) representative patterns obtained through K-Means clustering and (d) outlier patterns to help the users gain an overview of the dataset.}
	\label{oldZV}
\end{figure}
\begin{figure}[h!]
  \centering
  \includegraphics[width=0.9\linewidth]{figures/dcc.pdf}
  \vspace{-6pt}
  \caption{Example of dynamic classes. (a) Four different classes with different Lithium solvation energies (li) and boiling point (bp) attributes based on user-defined data ranges. (b) Users can hover over the visualizations for each dynamic class to see the corresponding attribute ranges for each class. The visualizations of dynamic classes are aggregate across all the visualizations that lie in that class based on the user-selected aggregation method.}
  \label{dcc}
  \vspace{-10pt}
\end{figure}
\newpage
\change{\npar During the contextual inquiry, participants demonstrated the use of external tools for conducting analysis in their existing workflow, as shown in Figure~\ref{workflow}, including:
  \begin{denselist}
    \item \href{http://descut.cosmology.illinois.edu}{Image Cutout Service (Astronomy)}
    \item \href{http://cs.cmu.edu/~jernst/stem/}{Short Time-series Expression Miner (Genetics)}
    \item \href{http://srdata.nist.gov/solubility/}{Solubility Database (Material Science)}
  \end{denselist}
}
\begin{figure}[h!]
  \centering
  \includegraphics[width=0.9\linewidth]{figures/workflow.png}
  \caption{Screenshots from contextual inquiry: a) A1 examines a light curve manually using the Jupyter notebook environment, b) G2 uses a domain-specific software to examine clustering outputs.}
  \label{workflow}
\end{figure}
\vspace{-5pt}
\npar Based on our meeting logs with participants, we found that reasons for not carrying a feature from the design to implementation stage included:
\begin{denselist} %he amount of nice-to-have features that one could envision for the tool is endless.
\item Nice-to-haves: One of the most common reasons for unincorporated features comes from participant's requests for nice-to-have features. To this end, we use two criteria to heuristically judge whether to implement a particular feature:
\begin{enumerate}[leftmargin=*]
\item \textit{Necessity:} Without this feature, can participants still work with this dataset using the tool and meet their information needs?
\item \textit{Generality:} Will this feature benefit only this specific use case or be potentially useful for other domains as well?
\end{enumerate}
\item ``One-shot'' operations: We decided not to include features that only needed to be performed once and remain fixed thereafter in the analysis workflow. For example, certain preprocessing operations such as filtering null values only needed to be performed once with an external tool.
\item Substantial research or engineering effort: Some proposed features did not make sense in the context of VQS or required a completely different set of research questions. For example, the question of how to properly compute similarity between time series with non-uniform number of datapoints arose in the astronomy and genetics use case, but requires the development of a novel distance metric and algorithm that is out of the scope of our design study objective. %. For example, M3 proposed functional fitting to obtain fitting coefficients. Other features
\item Underdeveloped ideas: Other feature requirements came from casual specification that were underspecified. For example, A1 wanted to look for objects that have deficiency in one band and high emission in another band, but the scientific definition of ``deficiency'' in terms of brightness levels was ambiguous.
\end{denselist}
\change{\npar Figure~\ref{science_task} illustrates how each of the subtasks in participant's workflow can be addressed by a sensemaking process.}
\begin{table}[h!]
	\centering
	\includegraphics[width=\linewidth]{figures/science_task.pdf}
	\vspace{-6pt}\caption{Each VQS sensemaking process maps to scientific tasks and goals from each use case, from pattern search to comparing visualization collections to gaining overall data understanding. We find that our scientific participants typically have one focussed goal expressible through a single sensemaking process, but since their desired insights may not always be achievable with a single operation, they make use of the two other sensemaking processes to support them in accomplishing their main goal.}
	\label{science_task}
	\vspace{-10pt}
\end{table}

\section{Characterizing the Problem Space for VQSs}
We now characterize how each sensemaking process fits into different problem areas that VQSs are aimed to solve. Visual querying often consists of searching for a desired pattern instance (Z) across a visualization collection specified by some given attributes (X,Y). Correspondingly, we introduce two axes depicting the amount of information known about the visualized attribute and pattern instance.
\par Along the \textbf{pattern instance} axis, the visualization that contains the desired pattern may already be \texttt{known} to the analyst, exist as a pattern \texttt{in-the-head} of the analyst, or \change{be} completely \texttt{unknown} to the analyst. In the \texttt{known} pattern instance region (Figure~\ref{2dmodel} grey), visualization-at-a-time systems such as Tableau, where analyst manually create and examine each visualization one at a time, is more well-suited than VQSs, since analysts can directly work with the selected instance without the need for visual querying. Inspired by Pirolli and Card's information foraging framework~\cite{Pirolli}, which distinguishes between information processing tasks that are \textit{top-down} (from theory to data) and \textit{bottom-up} (from data to theory), we define \textit{top-down pattern search} as \change{the process where analysts query a fixed collection of visualizations based on their in-the-head pattern (Figure~\ref{2dmodel} blue)}. On the other hand, \change{\textit{bottom-up data-driven inquiries} (Figure~\ref{2dmodel} green) are driven by recommendations or queries that originate from the data (or equivalently, the visualization), since the pattern of interest is unknown and external to the user.} As we will discuss later, this process is crucial but underexplored in past work on VQSs.
%analysts often do not start with a known pattern instance. T
\par The second axis, \textbf{visualized attributes},
depicts how much the analyst
knows about which X and Y axes
they are interested in visualizing.
In both the astronomy and genetics use cases,
as well as past work in this space, \change{the attribute to be visualized is \texttt{known}, as data was in the form of a time series.} In the case of our material science participants, they wanted to explore relationships between different
X and Y variables. In this realm of \texttt{unknown} attributes, context creation (Figure~\ref{2dmodel} yellow) is
essential for allowing users to pivot across different visualization collections.
\begin{figure}[h!]
  \centering
  \includegraphics[width=0.9\linewidth]{figures/2dmodel.pdf}
  \caption{The problem space for VQSs is characterized by how much the analyst knows about the visualized attributes and the pattern instance. Colored areas highlight the three sensemaking processes in VQSs for addressing these characteristic problems. While prior work has focused solely on use cases in the blue region, we envision opportunities for VQSs beyond this to a larger space of use cases covered by the yellow and green regions.}
  \label{2dmodel}
  \vspace{-10pt}
\end{figure}

\section{Evaluation Study Analysis Details\label{apdx:studydetails}}
We analyzed the transcriptions of the evaluation study recordings through open-coding and
categorized every event in the user study using the following coding labels:
\begin{denselist}
    \item Insight (Science) \textbf{[IS]}: Insight that connected back to the science (e.g. ``This cluster resembles a repressed gene.'')
    \item Insight (Data) \textbf{[ID]}: Data-related insights (e.g. ``A bug in my data cleaning code generated this peak artifact.'')
    \item Provoke (Science) \textbf{[PS]}: Interactions or observations that provoked a scientific hypothesis to be generated.
    \item Provoke (Data) \textbf{[PD]}: Interactions or observations that provoked further data actions to continue the investigation.
    \item Confusion \textbf{[C]}: Participants were confused during this part of the analysis.
    \item Want \textbf{[W]}: Additional features that participant wants, which is not currently available on the system.
    \item External Tool \textbf{[E]}: The use of external tools outside of \zvpp to complement the analysis process.
    \item Feature Usage \textbf{[F]}: One of the features in \zvpp was used.
    \item Session Break \textbf{[BR]}: Transition to a new line of inquiry.
\end{denselist}

\begin{table}[h!]
  \begin{tabular}{lrrrrrrrrr}
  \hline
   Domain           &   IS &   ID &   PS &   PD &   C &   W &   E &   BR &   F \\
  \hline
   astro            &    4 &   12 &   13 &   57 &   2 &  18 &  20 &   22 &  67 \\
   genetics         &    8 &   12 &    7 &   35 &   4 &  13 &   1 &   21 &  52 \\
   mat sci          &   14 &    8 &    7 &   44 &   8 &  11 &   3 &   12 &  48 \\
  \hline
  \end{tabular}
  \caption{Count summary of thematic event code across all participants of the same subject area.}
\end{table}
\npar In addition, based on the usage of each feature during the user study, we categorized the features into one of the three usage types:
\begin{denselist}
    \item Practical \textbf{[P]}: Features used in a sensible and meaningful way.
    \item Envisioned usage \textbf{[E]}: Features which could be used practically if the envisioned data was available or if they conducted downstream analysis, but was not performed due to the limited time during the user study.
    \item Not useful \textbf{[N]}: Features that are not useful or do not make sense for the participant's research question and dataset.
\end{denselist}
The feature usage labels for each user is summarized in Figure~\ref{feature_heatmap}. A feature is regarded as \emph{useful} if it has a \textbf{P} or \textbf{E} code label. Using the matrix from Figure~\ref{feature_heatmap}, we compute the percentage of useful features for each sensemaking process as: $\frac{\textrm{\# of useful features in process}}{\textrm{total \# of features in process} \times \textrm{total \# of users}}$.
\vspace{-10pt}
\begin{figure}[h!]
    \centering
    \includegraphics[width=0.7\columnwidth]{figures/PENcoding.pdf}
    \vspace{-6pt}\caption{Heatmap of features categorized as practical usage (P), envisioned usage (E), and not useful (N). Columns are arranged in the order of subject areas and the features are arranged in the order of the three foraging acts. Participants preferred to query using bottom-up methods such as drag-and-drop over top-down approaches such as sketching or input equations. Participants found that context creation via filter constraints and dynamic class creation were powerful ways to compare between subgroups or filtered subsets.}
    \label{feature_heatmap}
    \vspace{-5pt}
\end{figure}

\end{document}
