\PassOptionsToPackage{table}{xcolor}
\documentclass[review,journal]{vgtc}                % final (journal style)
%\documentclass[review,journal]{vgtc}         % review (journal style)
%\documentclass[widereview]{vgtc}             % wide-spaced review
%\documentclass[preprint,journal]{vgtc}       % preprint (journal style)

%% Uncomment one of the lines above depending on where your paper is
%% in the conference process. ``review'' and ``widereview'' are for review
%% submission, ``preprint'' is for pre-publication, and the final version
%% doesn't use a specific qualifier.

%% Please use one of the ``review'' options in combination with the
%% assigned online id (see below) ONLY if your paper uses a double blind
%% review process. Some conferences, like IEEE Vis and InfoVis, have NOT
%% in the past.

%% Please note that the use of figures other than the optional teaser is not permitted on the first page
%% of the journal version.  Figures should begin on the second page and be
%% in CMYK or Grey scale format, otherwise, colour shifting may occur
%% during the printing process.  Papers submitted with figures other than the optional teaser on the
%% first page will be refused. Also, the teaser figure should only have the
%% width of the abstract as the template enforces it.

%% These few lines make a distinction between latex and pdflatex calls and they
%% bring in essential packages for graphics and font handling.
%% Note that due to the \DeclareGraphicsExtensions{} call it is no longer necessary
%% to provide the the path and extension of a graphics file:
%% \includegraphics{diamondrule} is completely sufficient.
%%
\ifpdf%                                % if we use pdflatex
  \pdfoutput=1\relax                   % create PDFs from pdfLaTeX
  \pdfcompresslevel=9                  % PDF Compression
  \pdfoptionpdfminorversion=7          % create PDF 1.7
  \ExecuteOptions{pdftex}
  \usepackage{graphicx}                % allow us to embed graphics files
  \DeclareGraphicsExtensions{.pdf,.png,.jpg,.jpeg} % for pdflatex we expect .pdf, .png, or .jpg files
\else%                                 % else we use pure latex
  \ExecuteOptions{dvips}
  \usepackage{graphicx}                % allow us to embed graphics files
  \DeclareGraphicsExtensions{.eps}     % for pure latex we expect eps files
\fi%

% for including postscript figures
% mind: package option 'draft' will replace PS figure by a filname within a frame

\usepackage{multirow}

%% it is recomended to use ``\autoref{sec:bla}'' instead of ``Fig.~\ref{sec:bla}''
\graphicspath{{figures/}{pictures/}{images/}{./}} % where to search for the images

\usepackage{microtype}                 % use micro-typography (slightly more compact, better to read)
\PassOptionsToPackage{warn}{textcomp}  % to address font issues with \textrightarrow
\usepackage{textcomp}                  % use better special symbols
\usepackage{mathptmx}                  % use matching math font
\usepackage{times}                     % we use Times as the main font
\renewcommand*\ttdefault{txtt}         % a nicer typewriter font
\usepackage{cite}                      % needed to automatically sort the references
\usepackage{tabu}                      % only used for the table example
\usepackage{booktabs}                  % only used for the table example
\usepackage{cite}
\usepackage{balance}  % to better equalize the last page
\usepackage{graphics} % for EPS, load graphicx instead
\usepackage[T1]{fontenc}
\usepackage{txfonts}
\usepackage{mathptmx}
\usepackage[htt]{hyphenat}
% \usepackage[pdftex]{hyperref}

% \usepackage{booktabs}
% \usepackage{textcomp}
\usepackage{array}
\usepackage{booktabs}
\usepackage{pifont}
\usepackage{xspace}
\usepackage{setspace}
\usepackage{titlesec}
\usepackage{graphicx}
\usepackage[skip=5pt,font=small]{caption}
\usepackage[textsize=tiny]{todonotes}
% Some optional stuff you might like/need.
\usepackage{microtype} % Improved Tracking and Kerning
% \usepackage[all]{hypcap}  % Fixes bug in hyperref caption linking
\usepackage{ccicons}  % Cite your images correctly!
% \usepackage[utf8]{inputenc} % for a UTF8 editor only
\usepackage{verbatim}
\usepackage{relsize}
\usepackage{etoolbox}
\usepackage{lipsum}   % for filler text
\usepackage{setspace} % for \onehalfspacing and \singlespacing macros
\usepackage[normalem]{ulem}
\usepackage{enumitem}
\usepackage{relsize,etoolbox}% http://ctan.org/pkg/{relsize,etoolbox}
\usepackage{makecell}
\renewcommand\theadalign{bc}
\renewcommand\theadgape{\Gape[4pt]}
\renewcommand\cellgape{\Gape[4pt]}
\newcommand{\tabitem}{~\llap{{\tiny\textbullet}}~}
\AtBeginEnvironment{quote}{\small}% Step font down one size relative to current font.
\newenvironment{denselist}{
    \begin{list}{\small{$\bullet$}}%
    {\setlength{\itemsep}{0ex} \setlength{\topsep}{0ex}
    \setlength{\parsep}{0pt} \setlength{\itemindent}{0pt}
    \setlength{\leftmargin}{1.5em}
    \setlength{\partopsep}{0pt}}}%
    {\end{list}}
\newcommand{\squishlist}{
   \begin{list}{$\bullet$}
    { \setlength{\itemsep}{0pt}
      \setlength{\parsep}{2pt}
      \setlength{\topsep}{0pt}
      \setlength{\partopsep}{0pt}
      \leftmargin=25pt
\rightmargin=0pt
\labelsep=5pt
\labelwidth=10pt
\itemindent=0pt
\listparindent=0pt
\itemsep=\parsep
    }
}
% Custom section spacing
\titlespacing*{\section}
{0pt}{1ex}{0.1ex}
\titlespacing*{\subsection}
{0pt}{0.1ex}{0.05ex}

\newcommand*{\img}[1]{%
    \raisebox{-.3\baselineskip}{%
        \includegraphics[
        height=\baselineskip,
        width=\baselineskip,
        keepaspectratio,
        ]{#1}%
    }%
}
\newcommand{\squishend}{\end{list}}
\newcommand{\npar}{\par\noindent}
% use extensively to toggle between paper and TR
\newcommand{\eat}[1]{}
% \newcommand{\papertext}[1]{{\leavevmode\color{blue}{#1}}}
% \newcommand{\techreport}[1]{{\leavevmode\color{red}{#1}}}
\newcommand{\papertext}[1]{#1}
\newcommand{\techreport}[1]{}
\newcommand{\nonannon}[1]{#1}
\newcommand{\annon}[1]{}
\newcommand{\boldpara}[1]{\par\noindent\textbf{#1}}
% de-facto paragraph format
\newcommand{\stitle}[1]{\noindent\textbf{#1}}
\newcommand{\nstitle}[1]{\\\noindent\textbf{#1}}
\newcommand{\cut}[1]{{\leavevmode\color{lightgray}{#1}}} %potentially move to TR
\newcommand{\ccut}[1]{} %confirmed cut
% \newcommand{\rot}[1]{\rotatebox{90}{#1}}
\newcommand{\rot}[1]{\rotatebox[origin=c]{90}{\parbox{2.6cm}{\centering#1}}}
\newcommand{\A}{\textbf{A: }}
\newcommand{\M}{\textbf{M: }}
\newcommand{\G}{\textbf{G: }}
\newcommand{\problemlist}{
    \nstitle{Motivating Use Case Challenges:}
    \vspace{-5pt}
    \begin{enumerate}[label=\textbf{C\arabic*:},leftmargin=0.7cm]
}
\newcommand{\featurelist}{
    \stitle{Instantiated Feature:}
    \vspace{-5pt}
    \begin{enumerate}[label=\textbf{F\arabic*:},leftmargin=0.7cm]
}
\newcommand{\enumend}{
  \end{enumerate}
  \vspace{-5pt}
}
% \urlstyle{leo}

% To make various LaTeX processors do the right thing with page size.
\def\pprw{8.5in}
\def\pprh{11in}
\special{papersize=\pprw,\pprh}
\setlength{\paperwidth}{\pprw}
\setlength{\paperheight}{\pprh}
\setlength{\pdfpagewidth}{\pprw}
\setlength{\pdfpageheight}{\pprh}

% create a shortcut to typeset table headings
% \newcommand\tabhead[1]{\small\textbf{#1}}
\newcommand{\zv}{\textit{zenvisage}\xspace}
\newcommand{\zvpp}{\textit{zenvisage++}\xspace}
\newcommand{\astro}{\textit{astro}\xspace}
\newcommand{\bio}{\textit{genetics}\xspace}
\newcommand{\matsci}{\textit{matsci}\xspace}
\newcommand{\change}[1]{{\leavevmode\color{red}#1}}
%\newcommand{\change}[1]{{\leavevmode#1}}

\newcommand{\agp}[1]{\textcolor{blue}{Aditya: #1}}
\newcommand{\dor}[1]{\textcolor{green}{Doris: #1}}
\newcommand{\kar}[1]{\textcolor{magenta}{Karrie: #1}}

\usepackage{xparse}   % http://ctan.org/pkg/xparse
% Rotation: \rot[<angle>][<width>]{<stuff>}
% \NewDocumentCommand{\rot}{O{320} O{2.2em} m}{\makebox[#2][r]{\rotatebox{#1}{#3}}}%
\newcommand*\OK{\ding{51}}
\renewenvironment{quote}{%
   \list{}{%
     \leftmargin0.15cm
     \rightmargin\leftmargin
   }
   \item\relax
}
{\endlist}
%% We encourage the use of mathptmx for consistent usage of times font
%% throughout the proceedings. However, if you encounter conflicts
%% with other math-related packages, you may want to disable it.

%% In preprint mode you may define your own headline.
%\preprinttext{To appear in IEEE Transactions on Visualization and Computer Graphics.}

%% If you are submitting a paper to a conference for review with a double
%% blind reviewing process, please replace the value ``0'' below with your
%% OnlineID. Otherwise, you may safely leave it at ``0''.
\onlineid{0}

%% declare the category of your paper, only shown in review mode
\vgtccategory{Research}
%% please declare the paper type of your paper to help reviewers, only shown in review mode
%% choices:
%% * algorithm/technique
%% * application/design study
%% * evaluation
%% * system
%% * theory/model
\vgtcpapertype{application/design study}

%% Paper title.
\title{\emph{You can't always sketch what you want}: \\ Understanding Sensemaking in Visual Query Systems}

%% This is how authors are specified in the journal style

%% indicate IEEE Member or Student Member in form indicated below
\author{Submission ID: XXXX}

%other entries to be set up for journal
\shortauthortitle{Biv \MakeLowercase{\textit{et al.}}: Global Illumination for Fun and Profit}
%\shortauthortitle{Firstauthor \MakeLowercase{\textit{et al.}}: Paper Title}

\abstract{
Visual query systems (VQSs) empower users to interactively search for line charts
with desired visual patterns typically specified using intuitive sketch-based interfaces. Despite their potential in accelerating data exploration, more than a decade of past work on VQSs has not been translated to adoption in practice. Through a year-long collaboration with experts from three diverse domains, we examine the role of VQSs in real data exploration workflows, enhance an existing VQS to support these workflows via a participatory design process, and evaluate how VQS components are used in practice. Via these observations, we formalize a taxonomy of key capabilities for VQSs, organized by three sensemaking processes. Perhaps somewhat surprisingly, we find that ad-hoc sketch-based querying is not commonly used during data exploration, since
analysts are often unable to precisely articulate the patterns they are interested in. We find that there is a spectrum of VQS-centric data exploration workflows, depending on the application domain, and that many of these workflows are not effectively supported in present-day VQSs. Our insights can pave the way for next-generation VQSs to be adopted in a variety of real-world applications.
}
\keywords{Visual analytics, exploratory analysis, visual query}

%% ACM Computing Classification System (CCS). 
%% See <http://www.acm.org/class/1998/> for details.
%% The ``\CCScat'' command takes four arguments.

\CCScatlist{ % not used in journal version
 \CCScat{K.6.1}{Management of Computing and Information Systems}%
{Project and People Management}{Life Cycle};
 \CCScat{K.7.m}{The Computing Profession}{Miscellaneous}{Ethics}
}

%% Uncomment below to include a teaser figure.
% \teaser{
%   \centering
%   \includegraphics[width=\linewidth]{CypressView}
%   \caption{In the Clouds: Vancouver from Cypress Mountain. Note that the teaser may not be wider than the abstract block.}
% 	\label{fig:teaser}
% }

%% Uncomment below to disable the manuscript note
%\renewcommand{\manuscriptnotetxt}{}

%% Copyright space is enabled by default as required by guidelines.
%% It is disabled by the 'review' option or via the following command:
% \nocopyrightspace

\vgtcinsertpkg

%%%%%%%%%%%%%%%%%%%%%%%%%%%%%%%%%%%%%%%%%%%%%%%%%%%%%%%%%%%%%%%%
%%%%%%%%%%%%%%%%%%%%%% START OF THE PAPER %%%%%%%%%%%%%%%%%%%%%%
%%%%%%%%%%%%%%%%%%%%%%%%%%%%%%%%%%%%%%%%%%%%%%%%%%%%%%%%%%%%%%%%%

\begin{document}
\maketitle
\raggedbottom
%!TEX root=main.tex
 \vspace{-5pt}
 \section{Introduction\label{sec:intro}}
 % one for each key finding: a) many features deemed to be of importance to VQSs by domain experts, not all supported by present-day VQSs b) sketch is inefficient, perhaps explaining why present-day VQSs are not popular c) identify 3 typical workflows involving various sensemaking modalities in different proportions, depending on the application
 %To discover patterns of interest, analysts construct line chart visualizations \cut{using toolkits like \texttt{ggplot} or \texttt{matplotlib}, or visualization construction interfaces like Excel or Tableau,} \change{by} specifying {\em exactly} what they want to visualize. For example, when trying to find celestial objects corresponding to supernovae, which have a specific pattern of brightness over time, astronomers individually inspect the corresponding line chart for each object---numbering in the hundreds---until they find ones that match the pattern.
 Line charts are commonly employed during data exploration---the intuitive connected patterns often illustrate complex underlying processes
 and yield interpretable and visually compelling data-driven narratives~\cite{Few2012}. 
%To discover patterns of interest, analysts often have to construct and inspect thousands of line chart visualizations manually to find ones that match their desired pattern.
 \achange{However, discovering line charts that display certain meaningful patterns, trends, or characteristics of interest is often  an overwhelming and error-prone process, consisting of manual examination of large numbers of line charts. For example, when trying to find supernovae, which exhibits a unique pattern of brightness over time (an initial peak followed by a long-tail decay), astronomers often have to construct and inspect thousands of line chart visualizations manually to find ones that match their desired pattern.}
 %\techreport{For example, when trying to find celestial objects corresponding to supernovae, which have a specific pattern of brightness over time, astronomers individually inspect the corresponding line chart for each object---numbering in the hundreds---until they find ones that match the pattern.}\ccut{Similarly, when trying to infer relationships between two physical properties for different subsets of battery electrolytes, scientists need to individually visualize these properties for each subset (out of an unbounded number of such subsets) until they identify relationships that make sense to them.} 
 %This process of manual exploration of large numbers of line charts \change{for pattern identification} is not only error-prone, but also overwhelming for analysts. 
 To address this \change{exploration} challenge, there \change{has} been a large number of papers dedicated to building \emph{Visual Query Systems} (VQSs)\change{---the term coined by Ryall et al.~\cite{ryall2005querylines} to describe} systems that allow users to specify and search for desired \change{line chart patterns via visual} interfaces~\cite{mohebbi2011google,Hochheiser2004,wattenberg2001sketching,Siddiqui2017VLDB,ryall2005querylines,correll2016semantics,Mannino2018,Eichmann2015,Holz2009}. %visual patterns via interactive 
 These interfaces typically include a sketching canvas where users can draw a visual pattern of interest, with the system automatically traversing all potential visualization candidates to find those that match the specification. 
 % \par While this intuitive specification interface appears to be a promising solution
 \par While these intuitive specification interfaces were proposed as a promising solution to the problem of painful manual exploration of visualizations \change{for time-series analysis}~\cite{ryall2005querylines,wattenberg2001sketching}, to the best of our knowledge, VQSs have not lived up to these expectations and are not very commonly used in practice. \achange{One likely reason for the lack of VQS adoption may be attributed to how prior work have focused almost solely on optimizing for better pattern-matching algorithms and interactions, with few that invested in understanding actual user needs and how VQSs can be used for solving real-world problems.} {\em Our paper seeks to understand how VQSs can actually be used in practice, as a first step towards the broad adoption of VQSs in data analysis}. Unlike prior work on VQSs, we set out to not only evaluate VQSs in-situ on real problem domains, but also involve participants from these domains in the VQS design. We present findings from a series of interviews, contextual inquiry, participatory design, and user studies with scientists from three different domains---{\em astronomy, genetics,} and {\em material science}---over the course of
 a year-long collaboration. \change{The amount of time we invested in each of these three diverse domains surpasses the norm in this field and is key to uncovering the insights presented in this paper.} As illustrated in Figure~\ref{science_goal}, these domains were selected to capture a diverse set of goals and datasets wherein VQSs can help address important scientific questions, such as: How does a treatment affect the expression of a gene in a breast cancer cell-line? Which battery components have sustainable levels of energy-efficiency and are safe and
 cheap to manufacture in production?
 \begin{figure}[ht!]
 	\centering
 	\includegraphics[width=\linewidth]{figures/science_goal.pdf}
 	\caption{Desired insights, problem and dataset challenges for each of the three application domains in our study.}
 	\label{science_goal}
 	\vspace*{-15pt}
 \end{figure}
 \par Via contextual inquiry and interviews, we first identified challenges in existing data analysis workflows in these domains
 that could be potentially addressed by a VQS. Building on top of an existing open-source VQS, \zv~\cite{Siddiqui2017,Siddiqui2017VLDB}, we engaged participants in a process of participatory design (PD)~\cite{Muller1993,BodkerGronbaek,HoltzblattJones} to enable them to better compose data exploration workflows that lead to insight discovery, over the course of a year. \change{Rather than targeting a domain-specific solution, we chose to perform participatory design across multiple domains (an uncommon practice in visualization design studies) to observe differences and commonalities across domains to synthesize \change{high-level} insights regarding the use of VQSs.} \achange{While designing and performing this multi-phased, mixed-methods research agenda across three different use cases was an ambitious feat, this endeavor was necessary for addressing the qualitative, participant-centered research questions investigated in this work.}
 \par We organize our PD findings into a taxonomy of VQS capabilities, involving three sensemaking processes inspired by Pirolli and Card's notional model of analyst sensemaking~\cite{Pirolli}. The sensemaking processes include \emph{top-down pattern search} (translating a pattern ``in-the-head'' into a visual query), \emph{bottom-up data-driven inquiries} (querying or recommending based on data), and \emph{context-creation} (navigating across different collections of visualizations). We find that prior VQSs have focused on enabling top-down processes \change{(via sketching \achange{capabilities})}, \achange{but have largely overlooked the two other processes that we found to be essential in all three domains. These missing \achange{capabilities} partially explains why \achange{prior VQSs} have not been widely adopted in practice}.
% other two processes that we found to be crucial for all three domains.
 %to gather feedback and iterate on VQS feature designs, culminating in a new enhanced VQS, \zvpp.
 \par To study how various VQSs are used in practice, we conducted a final evaluation study with nine participants using our final VQS prototype to address their research questions on their own datasets. During this 1.5-hour study, participants \achange{gained} novel scientific insights,
 such as identifying a star with a transient pattern that was known to harbor a Jupiter-sized planet\achange{, discovering a previously-unknown relationship between solvent properties,} and finding characteristic gene expression profiles confirming the results of a related publication. %\techreport{Participants also gain additional insights about their datasets, including debugging mislabeled features and uncovering erroneous data preprocessing procedure applied to a collaborator's dataset.}%\techreport{, and discovering that the dip in an astronomical light curve is caused by saturated imaging equipment overlooked by the existing error-detection pipeline.}
 %\agp{Explain why these findings are important.}\dor{I think saying that planetary discovery is related to future colonization is a bit too much here and significance of characteristic gene expression profiles. Also we already described the significance of each domain earlier with the `important scientific questions' part.}
 %that goes from a pattern in-the-head to a desired visualization
 \par By analyzing the evaluation study results, we \achange{were somewhat surprised to discover} that sketching a pattern for querying is often ineffective on its own. This is due to the fact that sketching makes the problematic assumption that users know the pattern that they want to sketch and are able to sketch it precisely.\achange{ However, this is not the case in practice. For example, the geneticists from our study often did not have a preconceived knowledge of what to sketch and search for and relied heavily on recommended common patterns and outlying ones provided by the VQS to jumpstart their queries. Likewise, while the material scientists from our study were interested in datapoints that fall within specific value-ranges, they did not have an apriori notion of what these desired patterns would look like. Overall,} participants typically opted to combine sketching with other means of pattern specification---one common mechanism was to drag-and-drop a recommended pattern onto the canvas, and then modify it (e.g., by smoothing it out). %\cut{However, most VQSs do not support these other mechanisms (as we argued earlier, they typically focus only on top-down sensemaking processes, without covering bottom-up and context creation)}\dor{cutting this out since already mentioned 2 paragraphs ago}.
 %Participants were, however, able to apply the two other sensemaking processes to gain novel scientific insights, such as identifying a star with a transient pattern that was known to harbor a Jupiter-sized planet, finding characteristic gene expression profiles confirming the results of a related publication, and discovering mislabelled features from a data preprocessing mistake.
 % Further analysis of how participants transition between different sensemaking processes during analysis using a Markov model illustrated
 \par \change{To further understand how participants engaged with VQSs in their analytical workflows, we used a Markov model to \achange{characterize} how participants transitioned between different sensemaking processes during their analysis. \achange{We found that} participants often constructed a diverse set of \achange{analytical} workflows tailored to their domains by focusing} around a primary sensemaking process, while iteratively interleaving their analysis with the two other processes. This finding points to how all three sensemaking processes, along with seamless transitions between them, are \achange{crucial} for enabling \achange{the effective use and adoption of VQSs for addressing real-world challenges.}%for data exploration.%For example, participants often center on a main sensemaking process, while interleaving variations with other two processes as they iterate on an analytic task.
 %---including the construction of a Markov model---
 \par To the best of our knowledge, our study is the \emph{first to holistically examine how VQSs can be designed to fit the needs of real-world, analysts and how they are actually used in practice}. Working with participants from multiple domains enabled us to compare the differences and commonalities across different domains, thereby identifying general VQS challenges and requirements for supporting common analytical goals. Our contributions include:
 \begin{denselist}
 \item a characterization of the problems addressable by VQSs through design studies with three different domains,
 \item the construction of a taxonomy of essential VQSs capabilities leading to a sensemaking model for VQSs, grounded in participatory design findings, %, as well as an articulation of the problem space that is amenable to VQSs
 \item an integrative VQS, \zvpp, post participatory design, capable of facilitating rapid hypothesis generation and insight discovery,
 \item study findings on how VQSs are used in practice, leading to the development of a novel sensemaking model for VQSs. %including the ineffectiveness of
 %evaluation
 % sketching and the ---- workflow
 \end{denselist}
 Our work not only opens up a new space of opportunities beyond the narrow use cases considered by prior studies, but also advocates common design guidelines and end-user considerations for building next-generation VQSs.
%!TEX root = main.tex
  \section{\change{Related Works\label{sec:relatedworks}}}
  \vspace{-10pt}
\change{
  % \subsection{Background and Motivation}
  \npar We will now describe past work in visual query systems and existing evaluation methods of visualization systems to provide background and motivation to our work. Then, we will introduce Pirolli and Card's sensemaking model, which serves as a framework for contextualizing our study findings. 
  % Visual query systems enable users to directly search for visualizations matching certain patterns through an intuitive specification interface. Early work in this space focused on interfaces to search for time series with specific patterns.
  % \change{For example, since} the intent of a sketch can be ambiguous, follow-up work has developed mechanisms to enable users to clarify how a sketch should be interpreted~\cite{ryall2005querylines,correll2016semantics,Mannino2018,Eichmann2015,Holz2009}.\dor{Should we move this sentence to the related work section?}
  \par \stitle{Visual Query Systems: Definition and Brief Survey}
  \npar \emph{Visual query system} (VQS) is a term introduced by Ryall et al.~\cite{ryall2005querylines} and Correll and Gleicher~\cite{correll2016semantics} to describe systems that enable analysts to directly search for time-series visualizations matching certain patterns through a visual specification interface. Examples of such systems include TimeSearcher~\cite{Hochheiser2001,Hochheiser2004}, where the query specification mechanism is a rectangular box, with the tool filtering out all of the time series that does not pass through it, and QuerySketch~\cite{wattenberg2001sketching} and Google Correlate~\cite{mohebbi2011google}, where the query is sketched as a pattern on canvas, with the tool filtering out all of the time series that have a different shape. Subsequent work, including TimeSketch~\cite{Eichmann2015}, SketchQuery~\cite{correll2016semantics}, and Qetch~\cite{Mannino2018}, recognized the ambiguity in sketching by studying how humans rank similarity in patterns. To improve the expressiveness of sketched queries \change{and clarify how a sketch should be interpreted}, these VQSs include finer-grained specification interfaces and pattern-matching algorithms, including QueryLines~\cite{ryall2005querylines} where queries can be flexibly composed of soft constraints and preferences and SoftSelect~\cite{Holz2009} where users can vary the level of sketch similarity across a search pattern. Beyond sketching, \zv~\cite{Siddiqui2017,Siddiqui2017VLDB}, SketchQuery, and TimeSearcher allow users to select an existing visualization as a query, either via drag-and-drop or double-clicking on the existing visualization. In our work, we built on \zv, since it was open-source, extensible, and included features beyond pattern and match specification typically found in existing systems, as compared in Table~\ref{table:relatedwork}. \dor{I didn't incorporate Aditya's comment on stating that ``(the details of each component is described in Section~\ref{sec:pd_findings})'' here because it is already stated in the Table caption.}
  %with the system returning visualizations that had the closest Euclidean distance to the queried pattern. 
  % and described in Section~\ref{sec:sensemaking}. 
  %performed crowdsourced perceptual studies to understand how humans rank similarity in patterns subjectively
  % , including the use of soft constraints~\cite{ryall2005querylines} and implicit relaxed selection techniques~\cite{Holz2009}.
  % In addition to this ongoing work, recent work have also performed crowdsourced perceptual studies to understand how humans rank similarity in patterns subjectively~\cite{Eichmann2015,correll2016semantics,Mannino2018}.
  \begin{table}[ht!]
    \vspace*{-10pt}
     \centering
     \includegraphics[width=0.8\linewidth]{figures/related_works_table.pdf}
     \caption{Table summarizing whether key functional components (columns) are covered by past systems (row\change{, ordered by recency}), indicated by checked cells. Column header colors blue, orange, green represents three sensemaking process (top-down querying, search with context, and bottom-up querying) described in Section~\ref{sec:pd_findings}. The heavily-used, practical features in our study for context-creation and bottom-up inquiry is largely missing from prior VQSs.}
     \label{table:relatedwork}
     \vspace*{-10pt}
  \end{table}
  \par \stitle{Design and Evaluation Methodologies for Visualization Systems}
  \npar Visualization systems are typically evaluated via in-lab usability studies or controlled studies against existing visualization baselines~\cite{Plaisant2004,North2006,Yi2008}. However, successful lab-tested systems do not always translate to community acceptance and adoption. For instance, while decades of work have shown VQSs to be effective in controlled lab studies, they have not gained widespread adoption. \ccut{Unlike our work, past VQSs have never been designed and evaluated in-situ on multiple real-world use cases. Even when use cases were involved~\cite{Hochheiser2004,correll2016semantics}, the inclusion of these case studies served as a post-hoc demonstrative case study that had little influence on the major design decisions of the system.} The unrealistic nature of controlled studies has prompted the visualization research community to develop more participant-centered, ethnographic approaches for understanding how analysts perform visual data analysis and reasoning~\cite{Plaisant2004,lam2012empirical,shneiderman2006strategies,munzner2009nested,Sedlmair2012}. For example, multi-dimensional, in-depth, long-term case studies (MILCs) combine interviews, surveys, logging, and other empirical artifacts to create a holistic understanding of how a visualization system can be used in its intended environment \cite{shneiderman2006strategies}. 
  \par \change{Even though advanced VQS algorithms and interactions have been developed and shown to be effective in lab studies, prior work has yet to address the initial steps of evaluating: 1) whether this is even the right problem to solve, by characterizing and interviewing target users and 2) whether the chosen data and operations actually solve the user's problems, by observing how the tool is used as a part of a real-world workflow. In the context of Munzner's nested model \change{for visualization design and evaluation}~\cite{munzner2009nested}, this gap between research and adoption stems from the common ``downstream threat'' of jumping prematurely into the deep levels of \textit{encoding, interaction, or algorithm design}, before a proper \textit{domain problem characterization} and \textit{data/operation abstraction design} is performed. Our work fills this crucial gap in the existing literature and demonstrates how incorrect assumptions adopted by most prior work in this space regarding the first two stages of Munzner's model may have led to the failure in VQS adoption}.
  \par In this work, we performed design studies~\cite{lam2012empirical,shneiderman2006strategies,Sedlmair2012} with three different subject areas for \textit{domain problem characterization} by adopting participatory design practices~\cite{Gould1983,Muller1993}. Participatory design is well-established in the CHI and CSCW community and has been successfully applied to develop systems for visual analytics~\cite{Aragon2008,Chuang2012}, tangible museum experiences~\cite{Ciolfi2016}, and scientific collaborations~\cite{Poon2008,Chen2016}. We chose to perform particpatory design over other possible techniques for usability evaluation (such as long-term deployed field studies or formative testing), since our goal was to engage potential stakeholders early-on and in every step of the design process to ensure that the design decisions are based on actual user needs and ultimately to develop a system that may eventually be adopted in their analytical workflows. Holzblatt and Jones~\cite{HoltzblattJones} describe contextual inquiry as a technique where researchers observe participant in their own work environment, in order to ``\textit{[develop] a system model that will support user's work}'', and subsequently ``\textit{fosters participatory design}''. Likewise, we first perform contextual inquiry and interviews with participants to understand their research questions, the challenges associated with their existing analytical workflows, and identify design opportunities for VQSs. Past research on participatory design has found that the use of functional prototypes is a common and effective way of engaging with participants and providing a starting point for participatory design~\cite{Ciolfi2016}. Similarly, we provide a functional prototype at the beginning of the participatory design sessions to showcase the capabilities of VQSs.\ccut{Since our participants were not aware of the existence of VQSs, let alone using them in their workflows, they would not have been able to imagine use cases for VQS without a starting point.} Likewise, the use of ``\textit{simulated future work situation}'' (where users are introduced to the envisioned use of the prototype) is also a common practice in cooperative prototyping when the real use of the prototype is not feasible~\cite{Grnbak1991}. To better understand how VQSs can be used in-situ participant's existing workflow, we regularly gathered feedback from participants and collaboratively envisioned potential designs based on preview demos of preliminary versions of our protoype \zvpp. 
  \par Finally, we validated our abstraction design with grounded evaluation~\cite{Plaisant2004,Isenberg2008}, where participants were invited to bring in their own datasets and research problems that they have a vested interest in to test our final deployed system.%\dor{Regarding Aditya's question on whether we could argue that PD is the only way to truly get at the heart ofthe problem underlying adoption. I don't think we should say this, there are many other possible techniques that Munzner's paper lists for dealing with domain and abstraction threats (for domain threats: ethnographic field studies, semi-structured interviews; for abstraction threats: formative testing collecting anecdotal evidence, long-term field study with deployed system). It would be hard to argue that PD is the *right* approach amongst these.}
  %Sedlmair et al. \cite{Sedlmair2012} highlights the benefits and pitfalls of design studies in visualization research. They advocate that design study methodology is suitable for use cases in which the data is available for prototyping, but the task is only partially known and the information is partially in the user's head. In that regard, our scientific use cases with VQS is well-suited for a design study methodology, as we learn about the scientist's data and analysis requirements and design interactions that helps users translate their ``in-the-head'' specifications into actionable visual queries. 
  %\par While these systems have been effective in controlled lab studies, they have never been designed and evaluated in-situ on multiple real-world use cases. Even when use cases were involved~\cite{Hochheiser2004,correll2016semantics}, the inclusion of these \change{case studies served as a post-hoc demonstrative case study that had} little influence on the major design decisions of the system. 
 %\change{Next, we will outline these two phases of our study, deferring details of the study procedures and protocols to the technical report.}%Next, we will describe these two phases of our study in more detail.
  \par \stitle{Sensemaking Models for Visual Analytics}
  \npar Based on our participatory design findings, we contribute to the \textit{data/operation abstraction design} of VQSs in Munzner's model by developing a taxonomy for understanding how analysts make use of VQSs to accomplish their analytical tasks. To develop a sensemaking model for VQS, we draw from Pirolli and Card's seminal paper on information sensemaking based on cognitive task analysis of intelligence analysts~\cite{Pirolli}. The sensemaking framework was designed to capture how expert analysts iteratively search and re-represent gathered evidence into a conceptual model (\emph{schema}). Many papers in visual analytics have also applied the sensemaking framework to motivate tool designs, such as for exploratory browsing of visualizations in large datasets~\cite{Battle2016} and of the Web~\cite{Olston2003}. In addition, the sensemaking framework has also been used for understanding and modeling user behavior in visual analytics, including how analysts gain insights from visualizations~\cite{Yi2008}, how biases can be introduced during visual analysis~\cite{Wall2017}, and how analysts transition between natural-language generated data facts and visualizations~\cite{Srinivasan2019}. 
  \par In this framework, the sensemaking process can be organized into: 1) a foraging loop that searches for information to further schema organization and 2) a sensemaking loop for constructing a schema that best aligns with the insights obtained from the analysis. Overall, the model distinguishes between information processing tasks that are \textit{top-down} (from theory to data) and \textit{bottom-up} (from data to theory), described more in Section~\ref{sec:sensemaking}. We were inspired by this model for expert intelligence analysis as it bears semblance to our work for studying how domain experts perform visual analysis using VQSs.
% Our VQS sensemaking model is inspired by Pirolli and Card's information foraging framework~\cite{Pirolli}, which distinguishes between information processing tasks that are \textit{top-down} (from theory to data) and \textit{bottom-up} (from data to theory).
% We analyze our participatory deisng findings through the lens of
 } 
  
%!TEX root = main.tex
\section{Methods\label{sec:methods}} 
\par We adopted a mixed methods research methodology that draws inspiration from ethnographic methods, iterative and participatory design, and controlled studies~\cite{miller_salkind_miller_2002,shneiderman2006strategies,Muller1993} to understand how VQSs can be used for scientific data analysis. Working with researchers from three different scientific research groups, we identified the needs and challenges of scientific data analysis and the potential opportunities for VQSs, via interviews and cognitive walkthroughs. 
\par We recruited participants by reaching out to research groups via email and word of mouth, who have experienced challenges in dealing with large amounts of data. We initially spoke to analysts from 12 different potential application areas and narrowed down to three use cases in astronomy, genetics, and material science for our participatory design study. Six scientists from three research groups participated in the design of \zv. On average, the participants had more than 8 years of research experience working in their respective fields. We list the participants in Table~\ref{participants}, and will refer to them by their anonymized ID as listed in the table throughout the paper. 
\par Our initial inspiration for building a VQS came from informal discussions with academic and industry analysts. Their current workflows required analysts to manually examine large numbers of visualizations to derive insights from their data. Given these early conversations with the participants, we built a basic VQS to serve as the functional prototype in the design study. As shown in Figure \ref{oldZV}, this early version of \zv allowed users to sketch a pattern or drag-and-drop an existing visualization as a query,then the system would return visualizations that had the closest Euclidean distance from the queried pattern. The details of the system is described in our previous work \cite{Siddiqui2017,Siddiqui2017VLDB}, which focused on the systems and scalability aspects of the VQSs.
	\begin{figure}[ht!] 
	\centering
	\includegraphics[width=\linewidth]{figures/oldZV_nozql.pdf}
	\caption{The \zv prototype allowed users to sketch a pattern in (a), which would then return (b) results that had the closest Euclidean distance from the sketched pattern. The system also displays (c) representative patterns obtained through K-Means clustering and (d) outlier patterns to help the users gain an overview of the dataset.}
	\label{oldZV}
	\end{figure}
\par Visualization systems are often evaluated using controlled studies that measure the user's performance against an existing visualization baseline~\cite{Plaisant2004}. Techniques such as artificially inserting ``insights'' or setting predefined tasks for example datasets work well for objective tasks, such as debugging data errors~\cite{kandel2011wrangler,Patel2010}, but these contrived methods are unsuitable for trying to learn about the types of real-world queries users may want to pose on VQSs. Due to the unrealistic nature of controlled studies, many have proposed using a more multi-faceted, ethnographic approach to understand how analysts perform visual data analysis and reasoning~\cite{Plaisant2004,lam2012empirical,shneiderman2006strategies,munzner2009nested,Sedlmair2012}. In order to make the user study more realistic, we opted for a qualitative evaluation where we allowed participants to bring datasets that they have vested interests in to address unanswered research questions. Participatory design has been successfully used in the development of interactive visualization systems in the past~\cite{Aragon2008,Chuang2012}. Sedlmair et al. \cite{Sedlmair2012} advocate that design study methodology is suitable for use cases in which the data is available for prototyping, but the task is only partially known and the information is partially in the user's head. In that regard, our scientific use cases with VQS is well-suited for a design study methodology, as we learn about the scientist's data and analysis requirements and design interactions that helps users translate their ``in-the-head'' specifications into actionable visual queries.
\par The use of functional prototypes is common in participatory design to provide a starting point for the participants. For example, Ciolfi et al.\cite{Ciolfi2016} studied two different alternatives to co-design (starting with open brief versus functional prototype) in the development of museum guidance systems and found that while both approaches were equally fruitful, functional prototypes can make addressing a specific challenge more immediate and focused. Our motivation for providing a functional prototype at the beginning of the participatory design sessions is to showcase capabilities of VQSs. Especially since VQSs are not common in the existing workflows of these scientists, participants may not be able to imagine their use cases without a starting point. 

\par During the participatory design process, we collaborated with each of the teams closely with an average of two meetings per month, where we learned about their datasets, objectives, and how VQSs could help address their research questions. A detailed timeline of our engagement with the participants and the features inspired by their use cases can be found in Figure \ref{timeline}. Participants provided datasets they were exploring from their domain, whereby they had a vested interest in using a VQS to address their own research questions. Through this process, we identified and incorporated more than 20 desired features into the VQS prototype over the period of a year. Finally, we conducted a realistic, qualitative evaluation to study how analysts interact with different VQS components in practice. The evaluation study participants included the six scientists from the participatory design study, along with three additional ``blank-slate'' participants who had never encountered \zv before. While participatory design subjects actively provided feedback on \zv with their data, they only saw us demonstrating their requested features and explaining the system to them, rather than actively using the system on their own. So the evaluation study was the first time that all nine of the participants used \zv to explore their datasets.
	\begin{figure*}[!ht]
	\centering
	\captionsetup{justification=centering,margin=2cm}
	\vspace{-10pt}
	\includegraphics[width=6in]{figures/timeline_new.pdf}
	\vspace{-6pt}\caption{Participatory design timeline for the scientific use cases.}
	\label{timeline}
	\vspace{-10pt}
	\end{figure*}
%!TEX root = main.tex
\section{Participants and Datasets}
During the design study, we observed the participants as they conducted a cognitive walkthrough demonstrating every component of their current data analysis workflow. In this section, we describe our study participants and their use cases to highlight the existing workflow and behavior that participants have adopted for conducting certain analysis tasks~\cite{Nielsen1994}.
\subsection{Astronomy} 
\par The Dark Energy Survey (DES) is a multi-institutional project with over 400 scientists. Scientists use a multi-band telescope that takes images of 300 million galaxies over 525 nights to study dark energy\cite{Drlica-Wagner2017}. The telescope also focuses on smaller patches of the sky on a weekly interval to discover astrophysical transients (objects whose brightness changes dramatically as a function of time), such as supernova explosions or quasars. The output is a time series of brightness observations associated with each object extracted from the images observed.
\par For over five months, we worked closely with an astronomer on the project's data management team working at a supercomputing facility. The scientific goal is to identify a smaller set of potential candidates that may be astrophysical transients in order to study their properties in more detail. \techreport{These insights can help further constrain physical models regarding the formation of these objects.}
\par Participant A1 was interested in \zv as he recognized how specific pattern queries could help scientists directly search for these rare objects. While an experienced astronomer who has examined many transient light curves can often distinguish an interesting transient object from noise by sight, they must visually examine and iterate through large numbers of visualizations of candidate objects. Manual searching is time-consuming and error prone as the large majority of the objects are not astronomical transients.
\techreport{\par If an object of interest or region is identified through the visual analysis, then the astronomer may be interested in inspecting the image of the region for cross-checking that the significant change in brightness of the object is not due to an imaging artifact. This could be done using a custom built web-interface that facilitates the access of cutout images for a queried region of the sky.}

\subsection{Genetics}
\par Gene expression is a common data type used in genomics and is obtained via microarray experiments. \techreport{In these experiments, a grid containing thousands of DNA fragments are exposed to stimuli and measurements for the level at which a gene is expressed are recorded as a function of time.} The data used in the participatory design sessions was the gene expression data over time for mouse stem cells aggregated over multiple experiments.\techreport{, downloaded from an online database\footnote{\url{ncbi.nlm.nih.gov/geo/}}}. We worked with a graduate student and a PI at a research university over three months who were using gene expression data to better understand how genes are related to phenotypes expressed during early development~\cite{Peng2016,Gloss2017}. They were interested in using \zv to cluster gene expression data before conducting analysis with a downstream machine learning workflow.
\par To analyze the data, participant G1 loads the preprocessed data is into a desktop application for visualizing and clustering gene expression data\footnote{\url{www.cs.cmu.edu/~jernst/stem/}}. Participant G1 sets several clustering and visualization parameters on the interface before pressing a button to execute the clustering algorithm. The cluster visualizations are then displayed as overlaid time series for each cluster, as shown in the visualization in Figure \ref{workflow}b. G1 visually inspects that all the patterns in each cluster looks ``clean'' and checks the number of outlier genes that do not fall into any of the clusters.  If the number of outliers is high or the visualizations look unclean, she reruns the analysis by increasing the number of clusters. When the visualized clusters look ``good enough'', G1 exports the cluster patterns into a csv file to be used as features in their downstream regression tasks.
\par Prior to the study, the student (G1) and PI (G3) spent over a month attempting to determine the best number of clusters for their upstream analysis based on a series of static visualizations and statistics computed after clustering. While regenerating their results took no more than 15 minutes every time they made a change, the multi-step, segmented workflow meant that all changes had to be done offline, so that valuable meeting time was not wasted trying to regenerate results. The team had a vested interest in participating in the design of \zv as they saw how the interactive nature of VQSs and the ability to query other time series with clustering results could dramatically speed up their collaborative analysis process.
% \par In the context of a VQS, the main challenge  sure what to look for, need to see typical patterns in dataset

% \techreport{Participant G1 processes the raw microarray data by using a preprocessing script written in R, where she (i) sub-selects 144 genes of interest, (ii) cleans up an experimental artifact due to measurements on multiple probes, (iii) log-transforms the raw data to show a more distinct shape profile for clustering, (iv) normalizes the gene expression values into the range of 0 to 1, and (v) performs Loess smoothing with default parameters to reduce the noise in the data}.}
\subsection{Material Science}
\par We collaborated with material scientists at a research university who are working to identify solvents that can improve battery performance and stability. These scientists work with large datasets containing over 25 chemical properties for more than 280,000 different solvents obtained from simulations.
\par We worked closely with a graduate students, a postdoctoral researcher, and a PI for over a year to design a sensible way of exploring their data using VQSs. Each row of their dataset represents a unique solvent, and consists of 25 different chemical attributes. They wanted to use \zv to identify solvents that not only have similar properties to known solvents but also are more favorable (e.g. cheaper or safer to manufacture), as well as to understand how changes in certain chemical attributes affects them.
\par Participant M1 starts his data exploration process with a list of known and proven solvents as a reference. For instance, he would search for solvents which have boiling point over 300 Kelvins and the lithium solvation energy above 10 kcal/mol using basic SQL queries. This helps him narrow down the list of solvents, and move on to the other properties for similar processing. The scientist also considers the availability and the cost of the solvents while exploring the dataset. When the remaining list of the solvents is sufficiently small, he drills down to more detail (e.g., such as looking at the chemical structure of the solvents to consider the feasibility of conducting experiments with the solvent). While he could identify potential solvents through  manual lookup and comparison,  the process lacked the ability to reveal complicated trends and patterns that might be hidden, such as how the change in one attribute can affect the behavior of other attributes of a solvent. M1 was interested in using a VQS as it was infeasible for him to manually compare between large numbers of solvents and their associated properties manually.


%, as well as their initial challenges they face with VQS during the participatory design process. %We summarize the common properties of and differences between these three groups of researchers in Figure~\ref{example}.

% as well as the ability to filter to data subsets while performing the visual search (e.g. examine only objects classified as stars that have magnitudes above certain threshold). %the main challenge lies in developing mechanisms that allows them to describe the patterns, such as specifying amplitude, etc. %know what to look for but hard to describe and find
These observation inform our ----- search-browse paradigm

%!TEX root = main.tex
 %\section{System-level Participatory Design Findings\label{sec:pd_findings}}
 % From loaded tgn, generate latex from https://www.tablesgenerator.com/latex_tables/ 
% replace 
% $\sim$ --> ~
% $\textbackslash{}$ --> \
% \{ --> {
% \} --> }
\begin{table*}[ht!]
\centering
  \resizebox{0.96\textwidth}{!}{%
\begin{tabular}{|p{0.05cm}|l|l|l|l|l|}
	\hline
	                                                    & Component                                                                                                 & Feature                                                                                                              & Purpose                                                                                                                                                                                                      & Task Example                                                                                                                                                                                                                                                    & \begin{tabular}[c]{@{}l@{}}Similar Features\\ in Past VQSs\end{tabular}                                                                                                                                                                                         \\ \hline
	\rowcolor[HTML]{AADFFD} 
	\cellcolor[HTML]{AADFFD}                                   & \cellcolor[HTML]{AADFFD}                                                                                  & \begin{tabular}[c]{@{}l@{}}Query by Sketch\\ (Figure \ref{zvOverview}B1)\end{tabular}               & \begin{tabular}[c]{@{}l@{}}Freehand sketching for \\ specifying pattern query.\end{tabular}                                                                                                                  & \begin{tabular}[c]{@{}l@{}}\A Find patterns with a peak \\ and long-tail decay that\\ may be supernovae candidates.\end{tabular}                                                                                                                 & \begin{tabular}[c]{@{}l@{}}All include sketch \\ canvas except~\cite{Hochheiser2004}.\end{tabular}                                                                                                                                        \\ \cline{3-6} 
	\rowcolor[HTML]{AADFFD} 
	\cellcolor[HTML]{AADFFD}                                   & \cellcolor[HTML]{AADFFD}                                                                                  & \begin{tabular}[c]{@{}l@{}}Input Equation\\ (Figure \ref{zvOverview}A1)\end{tabular}                & \begin{tabular}[c]{@{}l@{}}Specify a exact functional \\ form as a pattern query \\ (e.g., y=$x^2$).\end{tabular}                                                                                            & \begin{tabular}[c]{@{}l@{}}\M Find patterns exhibiting \\ inversely proportional \\ chemical relationship.\end{tabular}                                                                                                                          & ----                                                                                                                                                                                                                                                            \\ \cline{3-6} 
	\rowcolor[HTML]{AADFFD} 
	\cellcolor[HTML]{AADFFD}                                   & \multirow{-5}{*}{\cellcolor[HTML]{AADFFD}\begin{tabular}[c]{@{}l@{}}\textbf{Pattern Specification:}\\\textit{What is the shape of}\\\textit{the pattern query?}\end{tabular}} & \begin{tabular}[c]{@{}l@{}}Pattern Upload\\ (Figure \ref{zvOverview}D2)\end{tabular}                & \begin{tabular}[c]{@{}l@{}}Upload a pattern consisting\\ of a sequence of points as \\ a query.\end{tabular}                                                                                                 & \begin{tabular}[c]{@{}l@{}}\A Find supernovae based on \\ previously discovered sources.\end{tabular}                                                                                                                                            & \begin{tabular}[c]{@{}l@{}}Upload CSV\\ \cite{mohebbi2011google}\end{tabular}                                                                                                                                                                  \\ \cline{2-6} 
	\rowcolor[HTML]{AADFFD} 
	\cellcolor[HTML]{AADFFD}                                   & \cellcolor[HTML]{AADFFD}                                                                                  & \begin{tabular}[c]{@{}l@{}}Smoothing\\ (Figure \ref{zvOverview}D2)\end{tabular}                     & \begin{tabular}[c]{@{}l@{}}Interactively adjusting the level \\ of denoising on visualizations,\\ effectively changing the degree\\ of shape approximation when \\ performing pattern matching.\end{tabular} & \begin{tabular}[c]{@{}l@{}}\textbf{A, M:} Eliminate patterns \\ matched to spurious noise.\end{tabular}                                                                                                                                        & \begin{tabular}[c]{@{}l@{}}Smoothing ~\cite{Mannino2018}\\ Angular slope queries ~\cite{Hochheiser2004}\\ Trend querylines ~\cite{ryall2005querylines}\end{tabular}                     \\ \cline{3-6} 
	\rowcolor[HTML]{AADFFD} 
	\cellcolor[HTML]{AADFFD}                                   & \cellcolor[HTML]{AADFFD}                                                                                  & \begin{tabular}[c]{@{}l@{}}Range \\ Selection\\ (Figure \ref{zvOverview}B2, D4)\end{tabular}        & \begin{tabular}[c]{@{}l@{}}Restrict to query only in \\ specific x/y ranges of interest \\ through brushing selected\\ x-range and filtering \\ selected y-range.\end{tabular}                               & \begin{tabular}[c]{@{}l@{}}\A Matching only around \\ shape exhibiting a peak.\\ \M Matching only around \\ shape region that exhibit linear\\ or exponential relationships\end{tabular}                                          & \begin{tabular}[c]{@{}l@{}}Text Entry ~\cite{wattenberg2001sketching,Mannino2018}\\ Min/max boundaries ~\cite{ryall2005querylines}\\ Range Brushing ~\cite{Hochheiser2001}\end{tabular} \\ \cline{3-6} 
	\rowcolor[HTML]{AADFFD} 
	\multirow{-20}{*}{\cellcolor[HTML]{AADFFD}\rot{\vspace{-2pt}Top-Down}}         & \multirow{-10}{*}{\cellcolor[HTML]{AADFFD}\begin{tabular}[c]{@{}l@{}}\textbf{Match Specification:}\\\textit{How should the pattern}\\\textit{query be matched} \\\textit{with other visualizations?}\end{tabular}}   & \begin{tabular}[c]{@{}l@{}}Range \\ Invariance\\ (Figure \ref{zvOverview}D1,4)\end{tabular}         & \begin{tabular}[c]{@{}l@{}}Ignoring vertical or horizontal \\ differences in pattern matching \\ through option for x-range\\ normalization and y-invariant\\ similarity metrics .\end{tabular}              & \begin{tabular}[c]{@{}l@{}}\A Searching for existence of a\\ peak above a certain amplitude.\\ \G Searching for a \\ ``generally-rising" pattern.\end{tabular}                                                                    & \begin{tabular}[c]{@{}l@{}}Temporal invariants ~\cite{correll2016semantics}\end{tabular}                                                                                                                                                \\ \hline
	\rowcolor[HTML]{FBE39C} 
	\cellcolor[HTML]{FBE39C}                                   & \cellcolor[HTML]{FBE39C}                                                                                  & \begin{tabular}[c]{@{}l@{}}Data selection\\ (Figure \ref{zvOverview}A)\end{tabular}                 & \begin{tabular}[c]{@{}l@{}}Changing the collection of \\ visualizations to iterate over.\end{tabular}                                                                                                        & \begin{tabular}[c]{@{}l@{}}\M Explore tradeoffs and \\ relationships between \\ physical attributes.\end{tabular}                                                                                                                                & ----                                                                                                                                                                                                                                                            \\ \cline{3-6} 
	\rowcolor[HTML]{FBE39C} 
	\cellcolor[HTML]{FBE39C}                                   & \multirow{-4}{*}{\cellcolor[HTML]{FBE39C}\begin{tabular}[c]{@{}l@{}}\textbf{View Specification:} \\ \textit{What data to visualize} \\ \textit{and how should it} \\ \textit{be displayed?}\end{tabular}}    & \begin{tabular}[c]{@{}l@{}}Display control\\ (Figure \ref{zvOverview}D4)\end{tabular}               & \begin{tabular}[c]{@{}l@{}}Changing the details of \\ how visualizations should\\ be displayed.\end{tabular}                                                                                                 & \begin{tabular}[c]{@{}l@{}}\M Non-time-series data should \\ be displayed as scatterplot.\end{tabular}                                                                                                                                           & ----                                                                                                                                                                                                                                                            \\ \cline{2-6} 
	\rowcolor[HTML]{FBE39C} 
	\cellcolor[HTML]{FBE39C}                                   & \cellcolor[HTML]{FBE39C}                                                                                  & \begin{tabular}[c]{@{}l@{}}Filter\\ (Figure \ref{zvOverview}D3)\end{tabular}                        & \begin{tabular}[c]{@{}l@{}}Display and query only on data \\ that satisfies the composed \\ filter constraints.\end{tabular}                                                                                 & \begin{tabular}[c]{@{}l@{}}\A Eliminate unlikely \\ candidates by navigating to \\ more probable data regions.\\ \textbf{M, G:} Compare how overall\\ patterns change when filtered \\ to particular data subsets.\end{tabular} & ----                                                                                                                                                                                                                                                            \\ \cline{3-6} 
	\rowcolor[HTML]{FBE39C} 
	\multirow{-10}{*}{\cellcolor[HTML]{FBE39C}\rot{\vspace{-2pt}Context Creation}} & \multirow{-7}{*}{\cellcolor[HTML]{FBE39C}\begin{tabular}[c]{@{}l@{}}\textbf{Slice-and-Dice:} \\ \textit{How does navigating} \\ \textit{to another data subset} \\ \textit{change the query result?}\end{tabular}}                                                  & \begin{tabular}[c]{@{}l@{}}Dynamic Class \\ (Figure~\ref{dcc})\end{tabular}                    & \begin{tabular}[c]{@{}l@{}}Create custom classes of data \\ that satisfies one or more \\ specified range constraints. \\ Display aggregate \\ visualizations for separate\\ data classes.\end{tabular}      & \begin{tabular}[c]{@{}l@{}}\textbf{A, M:} Examine aggregate \\ patterns of different data \\ classes.\end{tabular}                                                                                                                             & ----                                                                                                                                                                                                                                                            \\ \hline
	\rowcolor[HTML]{B5E1A4} 
	\cellcolor[HTML]{B5E1A4}                                   & \begin{tabular}[c]{@{}l@{}}\textbf{Result Querying:} \\ \textit{What other visualizations}\\ \textit{``look similar" to the} \\\textit{selected pattern?}\end{tabular}                                                & \begin{tabular}[c]{@{}l@{}}Drag-and-drop\\ (Figure \ref{zvOverview}C, E)\end{tabular}               & \begin{tabular}[c]{@{}l@{}}Querying with any selected\\ result visualization as pattern\\ query (either from \\ recommendations or results).\end{tabular}                                                    & \begin{tabular}[c]{@{}l@{}}\textbf{A, G, M:} Find other objects that\\ are similar to X; Examine what \\ other objects similar to X look \\ like overall.\end{tabular}                                                                         & \begin{tabular}[c]{@{}l@{}}Drag-and-drop ~\cite{Hochheiser2001}\\ Double-Click ~\cite{correll2016semantics}\end{tabular}                                                                                        \\ \cline{2-6} 
	\rowcolor[HTML]{B5E1A4} 
	\multirow{-6}{*}{\cellcolor[HTML]{B5E1A4}\rot{\vspace{-2pt}Bottom-Up}}        & \begin{tabular}[c]{@{}l@{}}\textbf{Recommendation:} \\ \textit{What are the key patterns} \\ \textit{in this dataset?}\end{tabular} & \begin{tabular}[c]{@{}l@{}}Representative \\ and Outliers\\ (Figure \ref{zvOverview}E)\end{tabular} & \begin{tabular}[c]{@{}l@{}}Displaying visualizations of \\ representative trends and outlier\\ instances based on clustering.\end{tabular}                                                                   & \begin{tabular}[c]{@{}l@{}}\A Examine anomalies and debug \\ data errors through outliers.\\ \textbf{G, M:} Understand representative \\ trends common to this dataset \\ (or filtered subset).\end{tabular}                    & ----                                                                                                                                                                                                                                                            \\ \hline
\end{tabular}
}
  \caption{\rchange{Taxonomy of key capabilities essential to VQSs and major features incorporated via participatory design. We organize each feature based on its functional component. From left to right, each of the three sensemaking process (first column) is broken down into key functional components (second column) in VQSs. Each component addresses a pro-forma question from a system's perspective.} Table cells are further colored according to the sensemaking process that each component corresponds to (Blue: Top-down, Yellow: Context creation, Green: Bottom-up). We list the functional purpose of each feature based on how it is implemented in \zvpp, example use cases from participatory design (\A astronomy, \M material science, \G genetics), and similar features incorporated in past VQSs. Given the exhaustive nature of Table~\ref{bigfeaturetable}, each motivated by example use cases from one or more domains, we further organize the features in terms of the Section~\ref{sec:sensemaking} sensemaking framework and assess their effectiveness in the Section~\ref{sec:eval_findings} evaluation study.}\label{bigfeaturetable}
  \vspace*{-15pt}
\end{table*}
 \section{\rchange{Design} Process and System Overview\label{sec:pd_findings}}
%All of the three domains described in the previous section recognized the need for a VQS. As discussed in Section~\ref{sec:methods},
%we worked closely with participants to develop features to address their problems and challenges.
 Given the need for a VQS, we further collaborated with participants to develop features to address their problems and challenges \rchange{in Phase II of our study}. We \rchange{first provide a high-level system overview of the \rchange{design} product, \zvpp, then we reflect on our feature discovery process}.
 \subsection{System Overview\label{sec:system}}% as described in the Figure~\ref{timeline} timeline
 % \rchange{Table~\ref{bigfeaturetable} highlights the major capabilities in our final \cut{PD product}\rchange{prototype}, \zvpp. Details of other features in \zvpp can be found in the} appendix and online documentation\footnote{Documentation: \url{http://github.com/zenvisage/zenvisage/wiki}}. 
  The \zvpp interface is organized into 5 major regions all of which dynamically update upon user interactions. Typically, participants begin their analysis by selecting the dataset and attributes to visualize in the \emph{data selection panel} (Figure~\ref{zvOverview}A). Then, they specify a pattern of interest as a query (hereafter referred to as \emph{pattern query}), through either sketching, inputting an equation, uploading a data pattern, or dragging and dropping an existing visualization, displayed on the \emph{query canvas} (Figure~\ref{zvOverview}B). \zvpp performs shape-matching between the queried pattern and other possible visualizations, and returns a ranked list of visualizations that are most similar to the queried pattern, displayed in the \emph{results panel} (Figure~\ref{zvOverview}C). At any point during the analysis, analysts can adjust various system-level settings through the \emph{control panel} (Figure~\ref{zvOverview}D) or browse through the list of \emph{recommendations} provided by \zvpp (Figure~\ref{zvOverview}E). For comparison, the existing \zv system \cut{(Figure~\ref{oldZV} in Appendix~\ref{apdx:pdartifact}) }from~\cite{Siddiqui2017} allowed users to query via sketching or drag-and-drop and displayed representative and outlier pattern recommendations, but had limited capabilities to navigate across different data subsets and had few control settings. Our \zvpp system is open source and available at: \url{http://github.com/zenvisage/zenvisage}; \rchange{other details and documentation can be found at that link}. %\techreport{Our \zvpp system is open source and available at: \url{github.com/[Annonymized for Submission]}.}
 \subsection{The Collaborative Feature Discovery Process~\label{sec:feature_dsicovery}}
 \par Throughout the \rchange{design} process, we worked closely with participants to discover VQS capabilities that were essential for addressing their high-level domain challenges. We identified various subtasks based on the participant's workflows, designed sensible features for accomplishing these subtasks that could be used in conjunction with existing VQS capabilities, and elicited feedback on intermediate feature prototypes. Bodker et al.~\cite{BodkerGronbaek} cite the importance of encouraging user participation and creativity in cooperative design through different techniques, such as future workshops, critiques, and situational role-playing. Similarly, our objective was to collect as many feature proposals as possible\cut{, while being inclusive across different domains}. We further organized these features \rchange{we added to \zvpp} into Table~\ref{bigfeaturetable} through an iterative coding process~\cite{Muller2012} by one of the authors.
 \par \cut{In grounded theory methods~\cite{Muller2012}, researchers first create \emph{open codes} to assign descriptive labels to raw data, followed by grouping open codes together by relationships or categories to form \emph{axial codes}. Finally, \emph{selective codes} are obtained by focusing on specific sets of axial codes to come up with a set of core emerging concepts. Inspired by grounded theory methods,} We first collected the list of features\rchange{, }example usage scenarios\rchange{, }and similar capabilities in existing VQSs as open codes\rchange{, corresponding to individual rows in Table~\ref{bigfeaturetable}}. Then, we further organized this list into axial codes representing ``components''\rchange{: core functionalities essential to VQSs (second} column in Table~\ref{bigfeaturetable}). Finally, \cut{as we will describe in Section~\ref{sec:sensemaking}, }the selective codes capture each of the sensemaking processes (\rchange{leftmost column} in Table~\ref{bigfeaturetable}). Instead of describing this table in detail, we present a typical example of how this table is organized. \rchange{From right to left, consider the row corresponding to the Smoothing feature (column 3) in} Table~\ref{bigfeaturetable}: one of the common challenges in astronomy and material science is that noise in the dataset can result in large numbers of false-positive matches. To address this issue, smoothing is a feature in \zvpp that enables users to adjust data smoothing algorithms and parameters on-the-fly to both denoise the data and change the degree of shape approximation applied when performing pattern matching. %smoothing is a feature in \zvpp that enables users to adjust data smoothing algorithms and parameters on-the-fly to both denoise the data and change the degree of shape approximation applied to all visualizations when performing pattern matching. This is useful for domains such as astronomy and material science where the dataset is noisy with large numbers of false positives that could be matched to any given pattern query. 
 Smoothing, along with range selection and range invariance\cut{(row 5 and 6)}, is part of the \emph{match specification} component: VQS mechanisms for clarifying how matching should be performed. Both match specification and \emph{pattern specification} (a description of what the pattern query should look like) are essential components for supporting the sensemaking process top-down pattern search (in blue\rchange{, as labeled in the leftmost column}).%, described in Section~\ref{sec:sensemaking}.
 % Smoothing is also supported in Qetch~\cite{Mannino2018}. Other interfaces have also developed constrained sketching mechanisms to allow users to partially specify certain shape characteristics, such as angular slope queries\techreport{for specifying the slope of a trend line}~\cite{Hochheiser2004} or piecewise trend querylines\techreport{over a specified data range}~\cite{ryall2005querylines}. Smoothing was chosen over these other interfaces for approximating key patterns in the data, since it was a familiar preprocessing step in our study participants' workflow.
 %, where we began with an existing VQS (\zv, as illustrated in Figure~\ref{oldZV}) and incrementally incorporated features, such as dynamic class creation (Figure~\ref{dcc}), throughout the PD process.
 \cut{\begin{figure}[h!]
 	\centering
 	% \captionsetup{justification=centering,margin=2cm}
 	% \includegraphics[width=6in]{figures/timeline.pdf}
   \includegraphics[width=\linewidth]{figures/timeline.pdf}
 	\caption{Timeline for progress in participatory design studies.}
 	\label{timeline}
 	% \vspace{-10pt}
 \end{figure}}
 \par It is important to note that while some of the proposed features in Table~\ref{bigfeaturetable} (such as data filtering and view specification) are pervasive in general visual analytics (VA) systems~\cite{Heer2012,Amar2005}, they have not been incorporated in present-day VQSs. In fact, one of the key insights here is in recognizing the need for an \emph{integrative} VQS whose sum is greater than its parts, that encourages analysts to rapidly generate hypotheses and discover insights by facilitating all three sensemaking processes. This finding is partially enabled by the unexpected benefits that come with collaborating with multiple groups of participants during the feature discovery process. \rchange{Next, we reflect on what worked and what didn't work in the feature discovery process, to inform similar design studies for visual analytics systems.
 % Given the highly-evolving, ad-hoc nature of exploratory data analysis~\cite{Keim2006,Tukey1970}, our collaborative feature discovery approach for aiding such analysis comes with its advantages and limitations. We distill these experiences into four separate themes to inform future visual analytics research that adopts a user-centered design approach.
 % 
 }
 \rchange{
 \par \stitle{Cross-pollination and Generalization via Parallel Use Cases.}} Introducing the newly-added features to \zvpp that addressed a particular domain often resulted in unexpected use cases for other domains. Considering feature proposals from multiple domains can also \rchange{result in cross-pollination of feature designs, often leading to} more generalized design choices. For example, around the same time when we spoke to astronomers who wanted to eliminate sparse time series from their search results, our material science collaborators also expressed a need for inspecting only solvents with properties above a certain threshold. \rchange{Instead of developing separate domain-specific features,} data filtering arose as a crucial, common operation that was later incorporated into \zvpp to support this class of queries. %User requests (or lack thereof) may not always translate to a direct need.
 %, leading to a comprehensive list of added features listed in Table~\ref{bigfeaturetable}.
 % \agp{should we have topic sentences to organize takeaways better}
 \begin{figure*}[ht!]
   \centering
   \vspace{-5pt}
   \includegraphics[width=0.9\linewidth]{figures/zvpp_system.pdf} %5.5
   \vspace{-5pt}\caption{The \zvpp system consists of : (A) data selection panel (where users can select visualized dataset and attributes), (B) query canvas (where the queried data pattern is submitted and displayed), (C) results panel (where the visualizations most similar to the queried pattern are displayed as a ranked list), (D) control panel (where users can adjust various system-level settings), and (E) recommendations (where the typical and outlying trends in the dataset is displayed).}
   \label{zvOverview}
   \vspace*{-10pt}
 \end{figure*}
 \rchange{
 \par \stitle{The Hidden Upfront Cost of Domain Integration.} While we expected to spend most of our collaborative design effort on figuring out the mechanics of visual query specification and matching, instead, preparing participant datasets for use in our system by meeting data and system requirements was the most time-consuming aspect of this phase. (We provide a detailed timeline in Appendix~\ref{xxx}\agp{plz add..})
 Data requirements include gaining an understanding of the problem domain, understanding the types of data suitable for a VQS, and cleaning and loading of this data. 
 System requirements include features required for the data to be visualized appropriately. 
 Often, participants could only envision the types of queries to issue and how variations to the system controls or mechanics could help better address their needs after seeing their data displayed for the first time in the prototype. 
 We also found that the time it took us to satisfy the data and system requirements decreased as we progressed to the later domains, by leveraging existing features in our prototype to satisfy some of
 the upfront needs. 
 \cut{In summary, it is important to allocate sufficient time when working with real-world datasets and users to account for initial system and data challenges, as well as developing general-purposed integration features (such as data uploading tools) that could be used across multiple use cases to decrease the upfront cost for future collaborations with new domains.}
 \par \stitle{Build Connectors, not Swiss-Army Knives.} Participants often envisioned how VQSs can be used in conjunction with other resources that they are familiar with, including those used for reference,
 computing statistics, browsing related datasets, or examining other data attributes or visualization types not supported in the VQS (scatterplots, histograms). The prevalence of external tools for supporting analytical inquiries stems from how analysts often require multiple data sources or data attributes to further develop or verify their hypothesis. For example, to determine whether a particular gene belongs to a regulatory network, G2 not only needed to look at the expression data in the VQS, but also enrichment testing and knockout data. Likewise, others used specialized tools for visualizing telescope images and 3D chemical structures. Instead of forcing our VQS prototype into a swiss-army knife, we instead focused on building connectors that enable smoother transitions between tools. For example, our data upload and pattern upload feature invites participants to bring data from an external tool into \zvpp, while our data export feature allowed users to download the similarity, representative trend, and outlier results as csv files from \zvpp into an external tool\cut{for downstream analysis, or export individual visualizations to facilitate easier sharing of visualization results with collaborators}. For example, geneticists could export the clusters directly from \zvpp as inputs to their downstream regression analysis.
 %Amongst these, data uploading is system ----, --- developing an extensive----.
 \npar \stitle{The Art of Problem Selection.}} While our collective brainstorming led to the cross-pollination and generalization of features, this technique can also lead to unnecessary features that result in wasted engineering effort. \rchange{During co-design}, there were numerous \cut{problems and associated }features proposed by participants, not all of which were incorporated. \rchange{The reasons for not carrying a feature from design to implementation stage included:
 \begin{denselist} %he amount of nice-to-have features that one could envision for the tool is endless.
 \item Nice-to-haves: One of the most common reasons for unincorporated features comes from participant's requests for nice-to-have features. We use two criteria (necessity and generality across domains) to judge whether to invest in developing a particular feature.
 % To this end, we use two criteria to heuristically judge whether to implement a particular feature:
 % \begin{enumerate}[leftmargin=*]
 % \item \textit{Necessity:} Without this feature, can participants still work with this dataset using the tool and meet their information needs?
 % \item \textit{Generality:} Will this feature benefit only this specific use case or be potentially useful for other domains as well?
 % \end{enumerate}
 \item ``One-shot'' operations: We decided not to include features that only needed to be performed once and remain fixed thereafter in the analysis workflow. For example, certain preprocessing operations such as filtering null values only needed to be performed once with an external tool, whereas data smoothing is a procedure that requires some degree of tuning and adjustments.
 \item Substantial research or engineering effort: Some proposed features did not make sense in the context of VQS or required a completely different set of research questions. For example, the question of how to properly compute similarity between time series with non-uniform number of datapoints arose in the astronomy and genetics use case, but requires the development of a novel distance metric and algorithm that is out of the scope of our design study objective. %. For example, M3 proposed functional fitting to obtain fitting coefficients. Other features
 \item Underdeveloped ideas: Other feature requirements came from casual specification that was underspecified. For example, A1 wanted to look for objects that have a deficiency in one band and high emission in another band, but the scientific definition of ``deficiency'' in terms of brightness levels was ambiguous.
 \end{denselist}
 \npar The decision of whether to invest in developing a feature requires a careful balance between promoting unforseen feature and wasted engineering efforts. Failure to identify these early signs may result in feature implementations that turn out not to be useful for the participants or result in feature bloat.
 } 
 % We detail the list of criteria that was used to determine whether to implement a proposed feature (including eliminating features that were nice-to-have, one-shot operations, non-essential, or required a substantially different set of research questions) in Appendix~\ref{apdx:pdartifact}\agp{Give Examples, organize as insights}.
 
\input{06-pd-sensemaking}
%!TEX root = main.tex
 \section{Evaluation Study Findings\label{sec:eval_findings}}
 Based on audio, video screen capture,
 and click-stream logs recorded
 during our Phase III evaluation study,
 we performed thematic analysis via open coding to label every event with a descriptive code. Event codes included specific feature usage,
 insights,
 provoked actions, confusion,
 need for capabilities unaddressed
 by the system, and use of external tools\rchange{\footnote{See Appendix~\ref{apdx:studydetails} for details on our coding protocol.}}. To characterize the usefulness
 of each feature, we further labeled whether each
 feature was useful to a particular participant's analysis.
 A feature was deemed \textit{useful}
 if it was either used in a sensible
 and meaningful way to accomplish a task or address a question during the study,
 or has envisioned usage outside of the constrained
 time limit during the study
 (e.g., if data was available or downstream analysis was conducted).
 \cut{We derived these labels from the study transcript
 to circumvent self-reporting bias~\cite{Williams2017},
 which can often artificially inflate
 the usefulness of the feature under examination.}
 In this section, we will apply our thematic analysis results to understand how each sensemaking process occurs in practice.%real-world analytic tasks.}
 %categorized the features based on whether there was a sensible usage of the feature
 % into one of the three usage types based on how each feature was used during the study:
 % \begin{denselist}
 %     \item Practical: Features used in a sensible and meaningful way.
 %     \item Envisioned: Features which could be used practically if the envisioned data was available or if they conducted downstream analysis, but was not performed due to the limited time during the study.
 %     \item Not useful: Features that are not useful or do not make sense for the participant's research question and dataset.
 % \end{denselist}
 % \par Given these initial findings, we further investigated where the `sketch'
 % Our interactions with the scientists showed that different modalities for inputting a query can be useful for different problem contexts. In addition, the three paradigms of sensemaking described earlier are not mutually exclusive. In fact, we find that participants often construct a central workflow focused on features from one of the main paradigms and interleave variations with the feature usage from the two other paradigms as they iterate on the analytic task. As shown in Figure~\ref{fig:usagefreqbysubject}, the central paradigm adopted by each use case is tightly coupled with characteristics of the analytic challenges presented by each subject area.
 % interplay
 % Next, we will describe some of the design principles (DP) based on our study findings.
 %focus on understanding the design space of VQSs and highlight the takeaways of our study.%developing a process model and design guideline for insight formation in VQSs and divert our thematic analysis of how VQSs fit into the context of an analysis workflow to our technical report.% These observation inform our ----- search-browse paradigm
 % \subsubsection{Discovery of Real-world insights}
 % \par Our participants' original workflow often required them to compare between many visualizations manually through separate analysis and visualization steps. Three of the participants cited that this segmented analyze-then-visualize workflow was one of their chief bottlenecks. The cognitive overhead from the segmented workflow made them more hesitant to visualize the results of different parameters and data operations, as A2 noted:
 % \begin{quote}
 % The quick visualization is something that I could not do on my current framework. I could not query as fast as you do; I need to wait for it, plot, and then compare. Every time I plot, I need to define subplots for 12 visualizations, then its slower. That's the reason why I sometimes plot less, and I rely more on the statistics from the likelihood tests. Sometimes I plot less than I really should be doing.
 % \end{quote}
 % The ability to rapidly experiment with large numbers of hypotheses in real time is a crucial step in the agile creative process in helping analysts discover actionable insights~\cite{Shneiderman2007a}. Five out of nine participants discussed how the dynamic, interactive update of the visualization in \zv was the main advantage for using VQSs over their original workflow.
 % \begin{figure}[h!]
 %   \includegraphics[width=\linewidth]{figures/usagefreqbysubject.pdf}
 %   \caption{The number of times each component is used during the evaluation study, broken down by subject areas.}\label{fig:usagefreqbysubject}
 % \end{figure}
 % \subsection{The Ineffectiveness of Sketch}
 \subsection{Uncovering the Myth of Sketch-to-Insight}
 % \subsection{DC3: Closing the loop in VQS sense-making cycle with bottom-up data-driven inquries}
 \par To understand the usefulness of different visual querying modalities, we analyzed their frequency of use in our evaluation study. To our surprise,
 despite the prevalence of sketch-to-query
 systems in the literature, \techreport{Figure \ref{fig:feature_heatmap} shows that} only two out of our nine participants
 found it useful to directly
 sketch a desired pattern onto the canvas. %Overall, bottom-up querying via drag-and-drop was more intuitive and more commonly used than top-down querying methods, such as sketching or input equations.
 The reason why most participants
 did not find direct sketching useful was that
 they often do not start their analysis with a specific pattern in mind.
 Instead, their intuition about what to query is derived
 from other visualizations they encountered
 during exploration, in which case it makes
 more sense to query using those visualizations
 as examples directly (e.g., by dragging and dropping
 that visualization onto the canvas to submit the query).
 Even if a user has a pattern in mind,
 translating that pattern into a sketch is often hard
 to do. For example,
 A2 wanted to search for a highly-varying signal
 enveloped by a sinusoidal pattern indicating
 planetary rotation \includegraphics[width=3.5\baselineskip,keepaspectratio]{figures/impossible_sketch.png}, which was hard to draw by hand.
 % \begin{figure}[h!]
 %   \includegraphics[width=0.95\linewidth]{figures/the_origins_of_sketch.pdf}
 %   \vspace{-5pt}
 %   \caption{The number of times a pattern query originates from one of the workflows. Pattern queries are far more commonly generated via bottom-up than top-down processes.}\label{fig:origins_of_sketch}
 %   \vspace{-5pt}
 % \end{figure}
 %where the pattern on the canvas typically originates
 \par We further investigated the processes that participants engaged in to construct pattern queries.\cut{, as presented in Figure~\ref{fig:origins_of_sketch}.} Pattern queries can be generated by either top-down (sketching \rchange{based on user's in-the-head pattern}) or bottom-up (drag-and-drop \rchange{based on what user observes from data}) processes\cut{, driven by various different querying intentions}. \rchange{While our study is not intended as a quantitative study with different querying modalities as conditions, we wanted to get an estimate of the relative frequency of different mechanisms across users. We examined the sequence of interactions that led to each pattern query and labeled each one based on one of the five ways it can be generated---two top-down and three bottom-up ways\footnote{Top-down: sketch-to-query, sketch-to-modify; Bottom-up: Result querying via object of interest, via ranked result, or via recommendations. See Appendix Figure~\ref{fig:origins_of_sketch} for more details.}.}\cut{Figure~\ref{fig:origins_of_sketch} shows that}\rchange{We find that
 \emph{bottom-up processes are 40\% more commonly used than top-down processes for generating a pattern query}}.%(used 23 times in total)(used 14 times in total) 
  \cut{We will describe the different ways in which the pattern queries are generated \cut{(corresponding to individual bars in Figure~\ref{fig:origins_of_sketch})} in this subsection and the next.}
  Within top-down processes, a pattern query could arise from users directly sketching a new pattern or by modifying an existing sketch. For example, M2 first sketched a pattern to find solvent classes with anticorrelated properties (pattern as a straight line with negative slope) without much success in finding a desired match. So he instead dragged and dropped one
 of the peripheral visualizations similar
 to his desired one and then smoothed
 out the noise in the visualization via sketching, yielding
 a straight line,
 as shown in Figure \ref{query_modification} (left).
 M2 repeated this workflow twice in separate
 occurrences during the study and
 was able to derive insights.
 \rchange{Likewise, A3 was searching for pulsating stars characterized by dramatic changes in the amplitudes of the light curves. She knows that stellar hotspots also exhibit dramatic amplitude fluctuations, but unlike pulsating stars, the variations happen at regular intervals. Figure~\ref{query_modification} (right) illustrates how A3 first picked out a regular pattern (suspected starspot), then modified it slightly so that the pattern looks more \rchange{``irregular''} (to find pulsating stars).}
% Likewise, Figure~\ref{query_modification} (right) illustrates how A3 first picked out a regular pattern (suspected starspot), then modified it slightly so that the pattern looks more \rchange{``irregular''} (to find pulsating stars).
 %Within these actions, there can be different intentions behind the sketch. While all visualizations that could be drag-and-dropped must come from the result or recommendation pane, a query can come from a particular object that the participant is interested in or simply through peripheral browsing of visualization results.%, described in the next section.
 %\par The latter case is also supported by the
 %\par There are also many unexpected use cases where sketching was simply used as a mechanism to modify an existing pattern query.
 %Likewise, A3 was interested in pulsating stars that looked similar to stellar hotspots in terms of its dramatic amplitude fluctuations, but differ in that their patterns exhibited irregularities. Figure \ref{query_modification} (right) showed how she first picked out a regular pattern (suspected star spot), then modified it slightly so that the pattern looks more irregular.
 %Likewise, A3 was interested in pulsating stars characterized by dramatic changes in the amplitudes of the light curves. During the search, hotspots on stellar surfaces often show up as false positives as they also result in dramatic amplitude fluctuations, but happen at a regular intervals. In the VQS, A3 looked for patterns that exhibits amplitude variations, but also some irregularities. As shown in Figure \ref{query_modification} (right), she first picked out a regular pattern (suspected star spot), then modified it slightly so that the pattern looks more irregular.\par While all visualizations that could be drag-and-dropped must come from the result or recommendation pane, a query can come from a particular object that the participant is interested in or simply through peripheral browsing of visualization results.
 % As described in the following section,
 % bottom-up pattern queries can come from either
 % the ranked list of results,
 % recommendations, or by selecting a
 % particular object of interest as a drag-and-drop query.
 % \begin{figure}[h!]
 %      \vspace{-5pt}
 %     \centering
 %     \includegraphics[width=0.8\columnwidth]{figures/QueryModificationBySketch.pdf}
 %     \vspace{-5pt}
 %     \caption{Example of sketch-to-modify, based on canvas traces from M2 (left) and A3 (right). The original drag-and-dropped query is shown in blue and sketch-modified queries in red.}%during the study demonstrating query modification
 %     \label{query_modification}
 %     \vspace*{-10pt}
 % \end{figure}
 \par The infrequent use of top-down pattern
 specification was also reflected in the fact
 that none of the participants queried using an equation.
 In both astronomy and genetics, the visualization patterns
 resulted from complex physical processes
 that could not be written down as equations analytically.
 Even in the case of material science when analytical
 relationships do exist, it is challenging to formulate patterns as functional forms in a prescriptive manner.
 \rchange{\par We found that some users employed match specification to remedy undesired results from their top-down pattern queries. While we did not rigorously study the effects of different analytical parameter settings, we observed that more users refined their matches by adjusting the range and degree of approximation, rather than opting for a different similarity metric. This points to future work in developing more flexible and intuitive vocabularies for modifying the match along the research directions pursued in~\cite{correll2016semantics,Mannino2018} over incorporating additional complex, off-the-shelf matching objectives in VQSs.}
 % Despite functional fitting being common in scientific data analysis, Figure \ref{feature_heatmap} shows that
 % . However,
 \begin{figure}
  \begin{minipage}[b]{.35\linewidth}
     \centering
     \includegraphics[width=\linewidth]{figures/QueryModificationBySketch.pdf}
     \caption{Example of sketch-to-modify, based on canvas traces from M2 (left) and A3 (right). The original drag-and-dropped query is shown in blue and sketch-modified queries in red.}%during the study demonstrating query modification
     \label{query_modification}
     \vspace*{-5pt}
  \end{minipage}
  \hspace{2pt}
  \begin{minipage}[b]{.63\linewidth}
    \centering
    \includegraphics[width=0.9\linewidth]{figures/related_works_table.pdf}
     \captionof{table}{\cchange{Table summarizing whether key functional components (columns) are covered by past systems (row, ordered by recency)\ccut{, indicated by checked cells}. Column header colors blue, yellow, green represent the three sensemaking processes\ccut{(top-down querying, context creation, and bottom-up querying) described in Section~\ref{sec:pd_findings}}. Heavily-used features for context-creation and bottom-up inquiry are largely missing from prior VQSs.}}
     \label{table:relatedwork}
     \vspace*{-5pt}
  \end{minipage}
  \vspace*{-25pt}
\end{figure}
  % \begin{table}[ht!]
  %   \vspace*{-10pt}
  %    \centering
  %    \includegraphics[width=0.6\linewidth]{figures/related_works_table.pdf}
  %    \caption{Table summarizing whether key functional components (columns) are covered by past systems (row, ordered by recency), indicated by checked cells. Column header colors blue, yellow, green represent three sensemaking process (top-down querying, search with context, and bottom-up querying) described in Section~\ref{sec:pd_findings}. The heavily-used features for context-creation and bottom-up inquiry are largely missing from prior VQSs.}
  %    \label{table:relatedwork}
  %    \vspace*{-5pt}
  % \end{table}
 \par Our findings suggest that while sketching
 is a useful construct for people to express their queries,
 \emph{the existing ad-hoc, sketch-only model for VQSs
 is insufficient on its own} without data examples
 that can help analysts jumpstart their exploration.
 In fact, \cut{from Figure~\ref{fig:origins_of_sketch},
 we can see that sketch-to-query was only used
 8 times, while the remaining querying modalities were used 29 times altogether,
 more than three times as much as sketch-to-query.}\rchange{ we found that sketch-to-query only accounted for about a fifth of the total number of visual queries performed during the study}. This finding has profound implications on the design of future VQSs, since \rchange{our comparison of VQS features across existing work (Table~\ref{table:relatedwork})} suggests that past work has primarily focused on top-down process components, without considering how useful these features are in real-world analytic tasks.
 %missing out largely on the key components in the other two paradigms \cut{(indicated by the absence of green features on the right hand side of the table).
 We suspect that these limitations may be why existing VQSs are not commonly adopted in practice. Note that we are not advocating for removing the natural and intuitive sketch capabilities from future VQSs completely, but instead focusing future research and design efforts to examine other (often underexplored) VQS sensemaking processes. Such processes could be applied in conjunction with sketching to help analysts more flexibly express their analytical goals, described next.

 %This result points to a need for ----- in future VQSs. %This, however, points to an exciting direction for sketching interface in VQSs for developing advanced drawing and modification tools that enable more precise visualization query specification.} %ed coverage in addressing different types of analytics use cases
 %For instance, material science discovered a known inverse relationship during e xploration
 %Which is really interesting. Which is something that we observed experimentally also. That is an interesting insight right htere. This seems to suggest that there is a fundamental issue in if you want to try to get better on this axis, and get as low as possible, you lose out on the other axis.
 %once they see it they know it but they don't know beforehand
 \subsection{Insights via Context Creation and Bottom-up Approaches}%Approaches}
 % \dor{Might need to come up with a more descriptive name for this subsection}
 \par As alluded to earlier,
 \emph{bottom-up data-driven inquiries
 and context creation are far more commonly
 used than top-down pattern search
 when users have no desired patterns in mind},
 which is typically the case for exploratory data analysis.
 In particular, top-down approaches were only useful for 29\% of the use cases,
 whereas they were useful for 70\% of the use cases
 for bottom-up approaches and 67\%
 for context creation\footnote{See Appendix~\ref{apdx:studydetails} for details on how this measure was computed.}. We now highlight some exemplary workflows demonstrating the efficacy of the latter two sensemaking processes.
 %number of features labeled as useful divided by the product of total number of features and total number of users}
 %%The prevalence of bottom-up approaches not only point to the need for result querying, but also providing recommendations for users without desired patterns in mind.
 \par \textbf{Bottom-up }pattern queries can come from either
 the ranked list of results,
 recommendations, or by selecting a
 particular object of interest as a drag-and-drop query. \cut{As shown in Figure~\ref{fig:origins_of_sketch} (\bartext{recommendations}),}
 The most common use of bottom-up querying
 is via recommended visualizations. For example, G2 and G3 identified that
 the three representative patterns
 recommended in \zvpp corresponded
 to the same three groups of genes discussed
 in a recent publication~\cite{Gloss2017}:
 induced genes (profiles with expression levels going up \includegraphics[width=2\baselineskip,keepaspectratio]{figures/up.pdf}),
 repressed genes (starting high then decreasing \includegraphics[width=2\baselineskip,keepaspectratio]{figures/down.pdf}),
 and transients (rising first then dropping at another time point \includegraphics[width=2\baselineskip,keepaspectratio]{figures/updown.pdf}). The clusters provoked G2 to generate a hypothesis
 regarding the properties of transients:
 \textit{``Is that because all the transient groups
 get clustered together, or can I get sharp patterns
 that rise and ebb at different time points?''}
 To verify this hypothesis, G2 increased the parameter controlling the number of clusters and noticed that the clusters
 no longer exhibited the clean,
 intuitive patterns he had seen earlier.
 G3 expressed a similar sentiment and proceeded
 by inspecting the visualizations
 in the cluster via drag-and-drop.
 He found a group of genes that all transitioned
 at the same timestep, while others transitioned
 at different timesteps.\techreport{G3 described the process of using
 VQSs as doing ``detective work'' that provoked
 him to generate further scientific hypotheses
 as well as data actions.} By browsing through the ranked list of
 results\cut{(\bartext{ranked results} in Figure~\ref{fig:origins_of_sketch})}, participants were also able to gain a peripheral overview of the data and spot anomalies during exploration. For example, A1 spotted time series that were too faint to look like stars after applying the filter CLASS\_STAR=1,
 which led him to discover that all stars have been mislabeled with CLASS\_STAR=0 as 1 during data cleaning.
 %This includes inspecting the top-most similar visualizations that lie in the queried cluster. and finding visualizations that are similar to an object of interest that exhibits a desired pattern. %. After browsing through a series query results and checking with an external database, he concluded that
  %We found that geneticists often gain their intuition about the data from the recommended representative trends. One example of rapid insight discovery
 %the dataset had been incorrectly labelled with all the stars with CLASS\_STAR=0 as 1 during data cleaning.
 %Examples of how recommended trends can provoke further insightful actions comes from G2 and G3, who identified that the three representative patterns shown in \zvpp---induced genes (profiles with expression levels staying up), repressed genes (started high but went down), and transients (go up and then come down at different time points)---corresponded to the same three groups of genes discussed in a recent publication~\cite{Gloss2017}.
 % \subsection{Enriching Search with Context}
 \par \textbf{Context creation} in VQSs enables users to change the ``lens''
 by which they look through the data
 when performing visual querying,
 thereby creating more opportunities
 to explore the data from different perspectives. Echoing the sentiment from past studies in visual analytics regarding the importance of designing features that enable users to select relevant subsets of data~\cite{Shneiderman1994,Amar2005,Heer2012,Lee2019}, we found that all participants found at least one of the features in context creation to be useful.
 %We designed two dynamic faceting features coupled with coordinated views that enabled users to specify subsets of data they are querying on and see immediate changes updated in the query, representative, and outlier results.
 %either envisioned a use case or utilized features in the context creation paradigm to explore and compare subsets of their data.
 %ven though the filtering step could be easily done with an external tool and reloaded into \zv, filtering on-the-fly was a powerful way to dynamically test his hypothesis. I
 \par Both A1 and A2 expressed that context creation through interactive filtering \rchange{was a powerful way to dynamically} test conditions and tune values that they would not have otherwise \rchange{experimented with}, effectively lowering the barrier between the iterative hypothesize-then-compare cycle during sensemaking.
 % echoing our previous finding that segmented workflow prevents extensive exploration.
 During the study, participants used filtering
 to address questions such as:
 \textit{Are there more genes similar
 to a known activator when we subselect
 only the differentially expressed genes?} \techreport{\texttt{DIFFEXP=1} }(G2) and \textit{Can I find more supernovae candidates if I query only on objects that are bright and classified as a star?} \techreport{\texttt{flux\textgreater10 AND CLASS\_STAR=1} }(A1). Three participants had also used filtering as a way to query with known individual objects of interest\cut{, as shown in the \bartext{object of interest} bar of Figure~\ref{fig:origins_of_sketch}}. For example, G2 set the filter as gene=9687 and explained that since ``\textit{this gene is regulated by the estrogen receptor, when we search for other genes that resemble this gene, we can find other genes that are potentially affected by the same factors.}''
 \par While filtering enabled users to
 narrow down to a selected data subset,
 dynamic classes (buckets of data points that satisfies one or more range constraints) enabled users to compare
 relationships between multiple attributes and subgroups of data.
 For example, M2 divided solvents in the database
 into eight different categories based on voltage properties,
 state of matter, and viscosity levels,
 by dynamically setting the cutoff values
 on the quantitative variables to create these classes.
 By exploring these custom classes, M2 discovered that the relationship between viscosity and lithium solvation energy is independent of whether a solvent belongs to the class of high voltage or low voltage solvents. He cited that dynamic class creation was central to learning about this previously-unknown attribute properties:
 \begin{quote}
 All this is really possible because of dynamic class creation, so this allows you to bucket your intuition and put that together. [...] I can now bucket things as high voltage stable, liquid stable, viscous, or not viscous and start doing this classification quickly and start to explore trends. [...] look how quickly we can do it!% Quite good!
 \end{quote}
 %Context creation is a useful ---- despite the --- pattern instance. Filtering still useful
 %\par Participants employed \emph{a mix of bottom-up and top-down approaches when faceting through data in VQS}, including narrowing the search space based on some intuition about a phenomena, selecting individual visualizations, or specifying high-level groupings to compare and query with.
 \subsection{Combining Sensemaking Processes in VQS Workflows}
 Given our observations so far as to how participants make use of each sensemaking process in practice, we construct a Markov model to further investigate the interplay between these sensemaking processes in the context of an analysis workflow. \rchange{Markov models have been used in the past by Reda et al.~\cite{Reda2016} in a similar manner to analyze interaction sequences from open-ended, exploratory analysis evaluation studies. The goal of such analysis is to quantitatively capture how users ``\textit{transitions between mental, interaction, and computational states}'' to afford researchers to qualitatively characterize the processes and behavioral patterns ``\textit{essential to insight acquisition}''~\cite{Reda2016}.} %interplay with each other dynamically i% - Bottom up and context creation much more common than top-down. Stats \%. Brief Examples of each (How they are used in practice).
 % - BUT All three process are equally important.
 % - participants can go from one to the next and there is no single progression (e.g. context --> bottom -up --> top-down).
 % - Both the PageRank score (how important/“central” is the state is to the analysis?), raw occurrence of each state (how frequently is a feature categorized as part of the state used?) and the normalized self-directed edge score (how much user stays in that state?) coincide with what each subject area focuses on.
 % We first examine the popularity of each sensemaking process based on how frequently they are used in the study. Figure~\ref{fig:feature_heatmap} show that features categorized as bottom-up (useful for 70\% of the use cases) and context creation (67\%) are much more useful compared top-down features (29\%). [Examples of Bottom up]. [Examples of Context Creation].
 %Despite differing in levels of usage, each sensemaking process fulfills a central role in participants' analysis.
 % illustrates the state transitions computed based on event sequences from the evaluation study. %stay in the same state.
 %Figure~\ref{fig:taxonomy},
 \par To compute the state transition probabilities in the Markov model, we make use of event sequences from the evaluation study, where each event consists of labels describing when specific features were used. Using the taxonomy in Table~\ref{bigfeaturetable}, we map each usage of a feature in \zvpp to one of the three sensemaking processes. Each participant's event sequence is divided into sessions, each indicating a separate line of inquiry during the analysis. Based on these event sequences---one for each session, we compute the aggregate state transition probabilities (edge weight labels in Figure~\ref{fig:transition}) to characterize how participants from each domain move between different sensemaking processes\rchange{\footnote{Results were broken down by domain, rather than on an individual basis, since the analytical patterns within the domains are very similar (possibily due to the similarity between analytical inquiries and datasets within the domains).}}. 
 \par \rchange{The transition probability represents the probability that an action from one class would be followed by one from the other.} For example, in material science, \rchange{60\% of events that started with }bottom-up exploration \rchange{lead} to context creation and to top-down pattern search the rest of the time. Self-directed edges indicate the probability that the participant
 would continue with the same type of sensemaking process. For example, when an astronomer performs top-down pattern search, \rchange{64\% of the transitions were }followed by another top-down process and \rchange{by} context creation the rest of the time,
 but never followed by a bottom-up process.
 This high self-directed transition probability
 reflects how astronomers often need to iteratively
 refine their top-down query through pattern
 or match specification when looking for a specific pattern. %when A1 looks for supernovae, he needs to iteratively refine his top-down query through pattern or match specification interfaces. %He could also chose to refine ----- , control --- to issue the desired query.
 %Each event sequence is separated by labeled session breaks signaling the beginning of a new line of inquiry. The
 % \begin{figure}[ht!]
 %   \centering
 %   \includegraphics[width=\linewidth]{figures/markov_transition.pdf}
 %   \caption{Markov models computed based on evaluation study event sequences, with edges denoting the probability that participant in the particular domain will go from one sensemaking process to the next. Nodes are scaled according to the eigenvector centrality, representing the percentage of time participants would spend in a particular sensemaking process in steady state.\label{fig:transition
 %   \vspace*{-5pt}
 % \end{figure}
 % Similar to the sensemaking model proposed by Pirolli and Card~\cite{Pirolli}, the ---- sensemaking loop representing iterative process
 % ---- highlights how the two newly discovered VQS sensemaking process in this paper are essential for `closing the loop' between the sensemaking acts in VQSs. %, equally important
 % both the browsing-act through recommendations and performing search via these results are
 %These examples show that both the browsing-act through recommendations and performing search via these results are
 %The three sensemaking ----- not mutually exclusive, participants can go from one to the next and there is no single progression (e.g. context --> bottom -up --> top-down). Iterative blah blah. three process are equally important.
 %The three paradigms of sensemaking described earlier are not mutually exclusive.
 %Different sensemaking processes can be useful for different problem contexts.
 \cut{
 \begin{table}[ht!]
   \centering
   \includegraphics[width=\linewidth]{figures/science_task_new.pdf}
   \caption{\rchange{Table of example usage scenarios from each domain for each sensemaking process.}\cut{Each VQS sensemaking process maps to scientific tasks and goals from each use case, from pattern search to comparing visualization collections to improving overall data understanding.} We find that our participants typically have one focused goal expressible through a single sensemaking process, but since their desired insights may not always be achievable with a single class of operation, \rchange{they make use of the two other sensemaking processes to support\cut{ (\textbf{Support})} them in accomplishing their main goal\cut{  (\textbf{Goal})}}.}%(shown with lighter background color)
   \label{science_task}
   \vspace{-10pt}
\end{table}
}
 \par To study how important each sensemaking process
 is for participant's overall analysis,
 we compute the eigenvector centrality of each graph,
 displayed as node labels in Figure~\ref{fig:transition}.
 These values represent the percentage of time the participants
 spend in each of the sensemaking processes
 when the transition model has evolved to a steady state~\cite{pierre2011}.
 Given that nodes in Figure~\ref{fig:transition}
 are scaled by this value, in all domains,
 we observe that there is always a prominent node
 connected to two less prominent ones---but it is also clear
 that all three nodes are essential to all domains.
 Our observation demonstrates how \emph{participants
 often construct a central workflow
 around a main sensemaking process based on their analytical \textbf{goals}
 and interleave variations with the two other \textbf{support} processes as they iterate on the analytic task}. \cut{This finding is further illustrated in Table~\ref{science_task}, where the usage scenarios exemplify how each sensemaking process supports essential subtasks and enables participants' scientific goals.}%\agp{move fig} 
 For example, the material scientists focus on context creation 56\% of the time, mainly through dynamic class creation, followed by bottom-up inquiries (such as drag-and-drop) and top-down pattern searches (such as sketch modification).
 %through dynamic classes than top-down pattern search. %astronomers focus largely on performing top-down pattern search, while filtering on the visualization space. %in Section~\ref{sec:pd_findings}}.
 The central process adopted by each domain
 is tightly coupled with the problem characteristics associated with each domain. For example, without an initial query in mind,
 geneticists relied heavily on bottom-up querying
 through recommendations to jumpstart their queries.
 %Despite the differing levels of usage from each subject area, we learn that \textit{each sensemaking process fulfills a central role in participants' analysis to address their high-level research objectives}.
 % \agp{maybe point to figure?}.
 % they brought to the study
 %pattern instance and visualized attributes
  \begin{figure}[ht!]
   \centering
   \includegraphics[width=\linewidth]{figures/markov_transition_new.pdf}
   \caption{Markov models computed based on evaluation study event sequences, with edges denoting the probability that participant in the particular domain will go from one sensemaking process to the next. Nodes are scaled according to their eigenvector centrality, representing the percentage of time participants would spend in a particular sensemaking process in steady state. The data consists of 206 event actions taken by participants during the study (80 for astronomy, 65 for genetics, and 61 for material science).}\label{fig:transition}
   \vspace*{-20pt}
 \end{figure}
 \par The Markov transition model exemplifies how participants
 adopted a diverse set of workflows
 based on their unique set of research questions. The bi-directional and cyclical nature
 of the transition graphs in Figure~\ref{fig:transition} highlight how the three sensemaking processes do not simply follow a linear progression towards finding a single pattern or attribute of interest. %, going from unknown to known in the Figure~\ref{2dmodel problem space}
 Instead, the high connectivity of the transition model illustrates how these three equally-important processes form a sensemaking loop, representing iterative acts of dynamic foraging and hypothesis generation. This finding reinforces the importance of each sensemaking process and indicates that future VQSs need to be \emph{integrative} in supporting all three sensemaking process to enable a diverse set of potential workflows for addressing a wide range of analytical inquiries. %This flexibility is enabled by the diverse set of potential workflows that could be constructed in an integrative VQS like \zvpp, for addressing a wide range of analytical inquiries.%single-directional%. The VQS sensemaking loop%full-fledged
 \subsection{Limitations}%suggests that direct sketch is inefficient
 \par Although evidence from our evaluation study points to the infrequent use of direct sketch, we have not performed controlled studies with a sketch-only system as a baseline to validate this hypothesis. \rchange{While we employed quantitative comparisons in various analysis throughout this section, our goal is to gain a formative understanding of VQS usage behavior across our small sample. Future studies with larger sample sizes and more representative samples are required to generalize these findings.} The goal of our study is to uncover qualitative insights that might reveal why VQSs are not widely used in practice; further validation of specific findings is out of the scope of this paper. While concerns regarding study results being focused on \zvpp must be acknowledged, we note that \zvpp is one of the most comprehensive VQSs to-date, covering many of the features from past systems and more (as evident from Table~\ref{table:relatedwork}). We believe that our integrative VQS, \zvpp, can serve as a baseline for future research in VQS to evaluate against and build upon. Given that this paper covered three design studies along with one evaluation study, we were unable to cover each domain to the level of detail typically found in a dedicated design study paper. Instead, our focus was to highlight the differences and similarities among these domains relevant to the capabilities required in VQS\cut{and we defer domain-specific participatory design details and artifacts to Appendix~\ref{apdx:pdartifact}}. Future longitudinal studies may also help alleviate the novelty effects that participants may have experienced during the evaluation study. While we have generalized our findings beyond existing work by employing three different and diverse domains\cut{(see Figure~\ref{fig:transition})},
 our case studies have so far
 been focused on scientific data analysis with domain-experts,
 as a first step towards greater adoption of VQSs.
 Other potential domains that could benefit from VQSs include:
 financial data for business intelligence,
 electronic medical records for healthcare,
 and personal data for quantified self.
 These different domains may each pose different sets of challenges (such as designing for novices) unaddressed by the findings in this paper,
  pointing to a promising direction for future work.
\input{08-conclusion}
\bibliographystyle{abbrv-doi}
\bibliography{reference}
\newpage
% \vspace*{-15pt}
 \appendix
 % \vspace*{-15pt}
 % \npar\textbf{{\huge Appendix}}
 \npar In Appendix A, we first describe additional details about the participatory design process, as well as domain-specific artifacts collected from contextual inquiry. Next, in Appendix B, we articulate the space of problems amenable to VQSs and describe how the sensemaking processes (introduced in Section~\ref{sec:sensemaking}) fit into different parts of the problem space. Finally, in Appendix C, we provide supplementary information regarding our analysis methods and results for the evaluation study.
 % \vspace*{-10pt}
 \section{Artifacts from Participatory Design\label{apdx:pdartifact}}
 % \vspace*{-10pt}
 \npar During the contextual inquiry, participants demonstrated the use of domain-specific tools for conducting analysis in their existing workflow, including:
   \begin{denselist}
     \item \href{http://descut.cosmology.illinois.edu}{Image Cutout Service (Astronomy)}
     \item \href{http://cs.cmu.edu/~jernst/stem/}{Short Time-series Expression Miner (Genetics)}
     \item \href{http://srdata.nist.gov/solubility/}{Solubility Database (Material Science)}
   \end{denselist}
 % \npar Our collaboration with participants is illustrated in Figure~\ref{timeline}, where we began with an existing VQS (\zv, as illustrated in Figure~\ref{oldZV}) and incrementally incorporated features, such as dynamic class creation (Figure~\ref{dcc}), throughout the PD process.
 % \begin{figure}[h!]
 % 	\centering
 % 	% \captionsetup{justification=centering,margin=2cm}
 % 	% \includegraphics[width=6in]{figures/timeline.pdf}
 %   \includegraphics[width=\linewidth]{figures/timeline.pdf}
 % 	\caption{Timeline for progress in participatory design studies.}
 % 	\label{timeline}
 % 	% \vspace{-10pt}
 % \end{figure}
 \begin{figure}[h!]
 	\centering
 	\includegraphics[width=0.9\linewidth]{figures/oldZV_nozql.pdf}
 	\caption{The existing \zv prototype allowed users to sketch a pattern in (a), which would then return (b) results that had the closest Euclidean distance from the sketched pattern. The system also displays (c) representative patterns obtained through K-Means clustering and (d) outlier patterns to help the users gain an overview of the dataset.}
 	\label{oldZV}
   % \vspace{-5pt}
 \end{figure}
 % \npar As discussed in Section~\ref{sec:feature_dsicovery}, not all of the features proposed by participants during PD were incorporated in the \zvpp prototype. Based on our meeting logs with participants, we found that reasons for not carrying a feature from the design to implementation stage included:
 % \begin{denselist} %he amount of nice-to-have features that one could envision for the tool is endless.
 % \item Nice-to-haves: One of the most common reasons for unincorporated features comes from participant's requests for nice-to-have features. To this end, we use two criteria to heuristically judge whether to implement a particular feature:
 % \begin{enumerate}[leftmargin=*]
 % \item \textit{Necessity:} Without this feature, can participants still work with this dataset using the tool and meet their information needs?
 % \item \textit{Generality:} Will this feature benefit only this specific use case or be potentially useful for other domains as well?
 % \end{enumerate}
 % \item ``One-shot'' operations: We decided not to include features that only needed to be performed once and remain fixed thereafter in the analysis workflow. For example, certain preprocessing operations such as filtering null values only needed to be performed once with an external tool.
 % \item Substantial research or engineering effort: Some proposed features did not make sense in the context of VQS or required a completely different set of research questions. For example, the question of how to properly compute similarity between time series with non-uniform number of datapoints arose in the astronomy and genetics use case, but requires the development of a novel distance metric and algorithm that is out of the scope of our design study objective. %. For example, M3 proposed functional fitting to obtain fitting coefficients. Other features
 % \item Underdeveloped ideas: Other feature requirements came from casual specification that were underspecified. For example, A1 wanted to look for objects that have deficiency in one band and high emission in another band, but the scientific definition of ``deficiency'' in terms of brightness levels was ambiguous.
 % \end{denselist}
  \begin{figure}[h!]
   \centering
   \includegraphics[width=0.9\linewidth]{figures/dcc.pdf}
   \caption{Example of dynamic classes. (a) Four different classes with different Lithium solvation energies (li) and boiling point (bp) attributes based on user-defined data ranges. (b) Users can hover over the visualizations for each dynamic class to see the corresponding attribute ranges for each class. The visualizations of dynamic classes are aggregate across all the visualizations that lie in that class based on the user-selected aggregation method.}
   \label{dcc}
   % \vspace{-10pt}
 \end{figure}
 % \newpage
 % \vspace*{-30pt}
 % \npar Table~\ref{science_task} illustrates how each of the subtasks in participant's workflow can be addressed by a sensemaking process.}
 % \begin{table}[h!]
 % 	\centering
 % 	\includegraphics[width=\linewidth]{figures/science_task.pdf}
 % 	\vspace{-6pt}\caption{Each VQS sensemaking process maps to scientific tasks and goals from each use case, from pattern search to comparing visualization collections to gaining overall data understanding. We find that our scientific participants typically have one focussed goal expressible through a single sensemaking process, but since their desired insights may not always be achievable with a single operation, they make use of the two other sensemaking processes to support them in accomplishing their main goal.}
 % 	\label{science_task}
 % 	\vspace{-10pt}
 % \end{table}
 \section{Characterizing the Problem Space for VQSs\label{appdx:problem_space}}
 We now characterize the space of problems addressable by VQSs and describe how each sensemaking process fits into different problem areas that VQSs are aimed to solve. Visual querying often consists of searching for a desired pattern instance (Z) across a visualization collection specified by some given attributes (X,Y). Correspondingly, we introduce two axes depicting the amount of information known about the visualized attribute and pattern instance as shown in Figure~\ref{2dmodel}.
 \npar Along the \textbf{pattern instance} axis, the visualization that contains the desired pattern may already be \texttt{known} to the analyst, exist as a pattern \texttt{in-the-head} of the analyst, or be completely \texttt{unknown} to the analyst. In the \texttt{known} pattern instance region (Figure~\ref{2dmodel} grey cell), systems such as Tableau, where analysts manually create and examine each visualization one at a time, is more well-suited than VQSs, since analysts can directly work with the selected instance without having to search for which visualization exhibits the desired pattern. We define \textit{top-down pattern search} as the process where analysts query a fixed collection of visualizations based on their in-the-head pattern (Figure~\ref{2dmodel} blue). On the other hand, \textit{bottom-up data-driven inquiries} (Figure~\ref{2dmodel} green) are driven by recommendations or queries that originate from the data (or equivalently, the visualization), since the pattern of interest is unknown and external to the user.
 %analysts often do not start with a known pattern instance. T
 \npar The second axis, \textbf{visualized attributes},
 depicts how much the analyst
 knows about which X and Y axes
 they are interested in visualizing.
 In both the astronomy and genetics use cases,
 as well as past work in this space, the attribute to be visualized is \texttt{known}, as data was in the form of a time series. In the case of our material science participants, they wanted to explore relationships between different
 X and Y variables. In this realm of \texttt{unknown} attributes, context creation (Figure~\ref{2dmodel} yellow) is
 essential for allowing users to pivot across different visualization collections.
 \begin{figure}[h!]
   \centering
   \includegraphics[width=0.9\linewidth]{figures/2dmodel.pdf}
   \caption{The problem space for VQSs is characterized by how much the analyst knows about the visualized attributes and the pattern instance. Colored areas highlight the three sensemaking processes in VQSs for addressing these characteristic problems. While prior work has focused solely on use cases in the blue region, we envision opportunities for VQSs beyond this to a larger space of use cases covered by the yellow and green regions.}
   \label{2dmodel}
   \vspace{-10pt}
 \end{figure}
 \rchange{
   \section{Evaluation Study Protocol\label{apdx:studyprotocol}}
   Here, we detail the procedures that were conducted during the evaluation study. At the beginning of the study, participants were asked a set of pre-study survey questions to collect basic information about participant's dataset, scientific questions, and existing workflows. While this information was similar to the ones collected through participatory design and contextual inquiry (Section~\ref{sec:participantdatasets}), the pre-study survey ensured that we have background information even for the ``blank-slate'' participants (who were not part of the earlier design study).
   \begin{denselist}
    \item What is your current role as a scientist? What are some examples of recent questions you’ve researched?
    \item Describe the workflow that you currently use to analyze and make sense of this type of data.
    \item Can you describe an interesting finding you found with your current workflow and the process you took to obtain this insight? 
   \end{denselist}
   \par After the tutorial and overview of the system, participant's selected dataset was loaded in. Participants were asked about their familiarity with the dataset and their analytical goals for the session.
   \begin{denselist}
    \item How familiar are you with this dataset? If you have worked with this dataset before, is there any insight that you already know from this dataset? 
    \item What is your goal for this dataset? How long have you been working with this dataset? What are you hoping to accomplish with this dataset?
   \end{denselist}
   \par During the main experiment, participants engaged in talk-aloud exercises as they explored their data. These two semi-structured interview questions were often posed when participants begin a new line of analytical inquiry.
   \begin{denselist}
    \item What is your current goal in this phase of the exploration? What type of insights are you hoping to obtain? 
    \item What actions are you planning to perform? How are you operationalize to achieve those goals?
   \end{denselist}
   In addition, we occasionally remind participants that they ask for help on something they want to accomplish on \zvpp, but were not sure about the sequence of interactions. They were also encouraged to use other tools in their existing workflow alongside \zvpp to perform their analysis. 
   \par At the end of the study, we interviewed participants with a set of open-ended questions regarding their experience with \zvpp, as listed below.
   \begin{denselist}
     \item How was \zvpp different from your existing workflow? 
     \item Can you describe how you would use \zvpp in your current workflow?
     \item On a scale of 1-10, how interested would you be in adopting this tool for your day-to-day workflow?
     \item What were some insights that you have gained from this session?
     \item Given the insights that you have obtained from \zvpp, what additional analysis will you run downstream before you publish these results? 
     \item What are the pros/cons for using \zvpp?
     \item Were there any queries that you were unable to address with \zvpp during today's session?
     \item What are additional features in \zvpp that would help with your scientific workflow or serve your scientific need?
   \end{denselist}
}
 \section{Evaluation Study Analysis Details\label{apdx:studydetails}}
 We analyzed the transcriptions of the evaluation study recordings through open-coding and
 categorized every event in the user study using the following coding labels:
 \begin{denselist}
     \item Insight (Science) \textbf{[IS]}: Insight that connected back to the science (e.g. ``This cluster resembles a repressed gene.'')
     \item Insight (Data) \textbf{[ID]}: Data-related insights (e.g. ``A bug in my data cleaning code generated this peak artifact.'')
     \item Provoke (Science) \textbf{[PS]}: Interactions or observations that provoked a scientific hypothesis to be generated.
     \item Provoke (Data) \textbf{[PD]}: Interactions or observations that provoked further data actions to continue the investigation.
     \item Confusion \textbf{[C]}: Participants were confused during this part of the analysis.
     \item Want \textbf{[W]}: Additional features that participant wants, which is not currently available on the system.
     \item External Tool \textbf{[E]}: The use of external tools outside of \zvpp to complement the analysis process.
     \item Feature Usage \textbf{[F]}: One of the features in \zvpp was used.
     \item Session Break \textbf{[BR]}: Transition to a new line of inquiry.
 \end{denselist}
 \begin{table}[h!]
   \begin{tabular}{lrrrrrrrrr}
   \hline
    Domain           &   IS &   ID &   PS &   PD &   C &   W &   E &   BR &   F \\
   \hline
    astro            &    4 &   12 &   13 &   57 &   2 &  18 &  20 &   22 &  67 \\
    genetics         &    8 &   12 &    7 &   35 &   4 &  13 &   1 &   21 &  52 \\
    mat sci          &   14 &    8 &    7 &   44 &   8 &  11 &   3 &   12 &  48 \\
   \hline
   \end{tabular}
   \caption{Count summary of thematic event code across all participants of the same subject area.}
 \end{table}
 \npar In addition, based on the usage of each feature during the user study, we categorized the features into one of the three usage types:
 \begin{denselist}
     \item Practical \textbf{[P]}: Features used in a sensible and meaningful way.
     \item Envisioned usage \textbf{[E]}: Features which could be used practically if the envisioned data was available or if they conducted downstream analysis, but was not performed due to the limited time during the user study.
     \item Not useful \textbf{[N]}: Features that are not useful or do not make sense for the participant's research question and dataset.
 \end{denselist}
 The feature usage labels for each user is summarized in Figure~\ref{feature_heatmap}. A feature is regarded as \emph{useful} if it has a \textbf{P} or \textbf{E} code label. Using the matrix from Figure~\ref{feature_heatmap}, we compute the percentage of useful features for each sensemaking process as: $\frac{\textrm{\# of useful features in process}}{\textrm{total \# of features in process} \times \textrm{total \# of users}}$.
 \begin{figure}[h!]
     \centering
     \includegraphics[width=0.7\columnwidth]{figures/PENcoding.pdf}
     \vspace{-6pt}\caption{Heatmap of features categorized as practical usage (P), envisioned usage (E), and not useful (N). Columns are arranged in the order of subject areas and the features are arranged in the order of the three foraging acts. Participants preferred to query using bottom-up methods such as drag-and-drop over top-down approaches such as sketching or input equations. Participants found that context creation via filter constraints and dynamic class creation were powerful ways to compare between subgroups or filtered subsets.}
     \label{feature_heatmap}
     \vspace{-5pt}
 \end{figure}
 % \vspace{-5pt}
 \begin{figure}[h!]
   \includegraphics[width=0.95\linewidth]{figures/the_origins_of_sketch.pdf}
   \vspace{-5pt}
   \caption{The number of times a pattern query originates from one of the workflows. Pattern queries are far more commonly generated via bottom-up than top-down processes.}\label{fig:origins_of_sketch}
   \vspace{-5pt}
 \end{figure}

\end{document}

