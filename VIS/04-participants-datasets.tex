%!TEX root = main.tex
 \section{Participants and Datasets\label{sec:participantdatasets}}
 In this section, we describe our study participants, their scientific goals, and their preferred analysis workflows\change{, based on the contextual inquiry that we conducted at the start of the design study}. %At the start of our design study, we conducted a contextual inquiry to learn about our participants' existing data analysis workflows. 
 While we collaborated with each application domain in depth, we focus on the key findings in each domain to highlight their commonalities and differences, in order to provide a backdrop for our generalized VQSs findings described later on.
 %use cases to highlight behaviors that participants have adopted for conducting certain analysis tasks.
 %At the start of our design study, we conducted contextual inquiry to learn about our participants' existing data analysis workflows. Next, we describe our study participants, their scientific goals,  and their preferred analysis workflows.
 \par\noindent\stitle{Astronomy:} The Dark Energy Survey (DES) is a multi-institution project that surveys 300 million galaxies over 525 nights to study dark energy~\cite{Drlica-Wagner2017}. The telescope used to survey these galaxies also focuses on smaller patches of the sky on a weekly interval to discover astronomical transients (objects whose brightness changes dramatically as a function of time), such as supernovae or quasars. Their dataset consists of a large collection of \emph{light curves: brightness observations over time, one associated with each astronomical object, plotted as time series. Over} five months, we worked closely with A1, an astronomer on the project's data management team at a supercomputing facility. Their scientific goal is to identify potential astronomical transients in order to study their properties. \techreport{These insights can help further constrain physical models regarding the formation of these objects.}
 \npar To identify transients, astronomers programmatically generate visualizations of candidate objects with \texttt{matplotlib} and visually examine each light curve. If an object of interest is identified through visual analysis, the astronomer may inspect the image of the object for verifying that the significant change in brightness is not due to an imaging artifact. While an experienced astronomer who has examined many transient light curves can often distinguish an interesting transient object from noise by sight, manual searching for transients is time-consuming and error prone, since the large majority of objects are false-positives. A1 was interested in VQSs as he recognized how specific pattern queries could help astronomers directly search for these rare transients.
 \par\noindent\stitle{Genetics:} Gene expression is a common measurement in genetics obtained via microarray experiments~\cite{Peng2016}. \techreport{In these experiments, a grid containing thousands of DNA fragments are exposed to stimuli and measurements for the level at which a gene is expressed are recorded as a function of time.} We worked with a graduate student (G1) and professor (G3) at a research university who were using gene expression data to understand how genes are related to phenotypes expressed during early embryonic development\techreport{\cite{Peng2016,Gloss2017}}. Their data consisted of a collection of gene expression profiles over time for mouse stem cells, aggregated over multiple experiments.\techreport{, downloaded from an online database\footnote{\url{ncbi.nlm.nih.gov/geo/}}.} %They were interested in using \zv to cluster gene expression data before conducting analysis with a downstream machine learning workflow.
 \npar Their typical workflow is as follows: G1 first loads the preprocessed gene expression data into custom desktop application to  visualize and cluster the profiles\techreport{\footnote{\url{www.cs.cmu.edu/~jernst/stem/}}. After setting several system parameters and executing the clustering algorithm, the overlaid time series for each cluster is displayed on the interface. G1 visually inspects that the patterns in each cluster looks ``clean'' and checks that the number of outlier genes (i.e., those that do not fall into any of the clusters) is low.  If the number of outliers is high or the clustered visualizations look ``unclean'', she reruns the analysis by increasing the number of clusters. Once the visualized clusters look ``good enough'', G1 exports the clusters to her downstream regression tasks.}
 \npar Prior to the study, G1 and G3 spent over a month attempting to determine the best number of clusters based on a series of static visualizations and statistics computed after clustering. While regenerating their results took no more than 15 minutes every time they made a change, the multi-step, segmented workflow meant that all changes had to be done offline.\techreport{, so that valuable meeting time was not wasted trying to regenerate results.} They were interested in VQSs, as interactively querying time series with clustering results had the potential to dramatically speed up their collaborative analysis process.
 %The team were interested in VQSs as they saw how interactively querying time series with clustering results could dramatically speed up their collaborative analysis process.
 %that can improve battery performance and stability
 \par\noindent\stitle{Material Science:} We collaborated with material scientists at a research university who are working to identify solvents for energy efficient and safe batteries. These scientists work on a large simulation dataset containing chemical properties for more than 280,000 solvents~\cite{Khetan2018}. Each row of their dataset represents a unique solvent with 25 different chemical attributes. We worked closely with a postdoctoral researcher (M1), professor (M2), and graduate student (M3) for over a year to design a sensible way of exploring their data. They wanted to use VQSs to identify solvents that not only have similar properties to known solvents, but are also more favorable (e.g., cheaper or safer to manufacture). To search for these desired solvents, they need to understand how changes in certain chemical attributes affect other properties under specific conditions.
 \begin{table*}[ht!]
  \begin{tabular}{|l|l|l|l|l|}
\hline
Component & Feature & Purpose & Task Example & \begin{tabular}[c]{@{}l@{}}Similar Features\\ in Past VQSs\end{tabular} \\ \hline
\rowcolor[HTML]{AADFFD}
\cellcolor[HTML]{AADFFD} & \begin{tabular}[c]{@{}l@{}}Query by Sketch\\ (Figure \ref{zvOverview}B1)\end{tabular} & \begin{tabular}[c]{@{}l@{}}Freehand sketching for \\ specifying pattern query.\end{tabular} & \begin{tabular}[c]{@{}l@{}}\A Find patterns with a peak \\ and long-tail decay that\\ may be supernovae candidates.\end{tabular} & \begin{tabular}[c]{@{}l@{}}All include sketch \\ canvas except~\cite{Hochheiser2004}.\end{tabular} \\ \cline{2-5}
\rowcolor[HTML]{AADFFD}
\cellcolor[HTML]{AADFFD} & \begin{tabular}[c]{@{}l@{}}Input Equation\\ (Figure \ref{zvOverview}A1)\end{tabular} & \begin{tabular}[c]{@{}l@{}}Specify a exact functional \\ form as a pattern query \\ (e.g., y=$x^2$).\end{tabular} & \begin{tabular}[c]{@{}l@{}}\M Find patterns exhibiting \\ inversely proportional \\ chemical relationship.\end{tabular} & ---- \\ \cline{2-5}
\rowcolor[HTML]{AADFFD}
\multirow{-3}{*}{\cellcolor[HTML]{AADFFD}\begin{tabular}[c]{@{}l@{}}Pattern\\ Specification\end{tabular}} & \begin{tabular}[c]{@{}l@{}}Pattern Upload\\ (Figure \ref{zvOverview}D2)\end{tabular} & \begin{tabular}[c]{@{}l@{}}Upload a pattern consisting\\ of a sequence of points as \\ a query.\end{tabular} & \begin{tabular}[c]{@{}l@{}}\A Find supernovae based on \\ previously discovered sources.\end{tabular} & \begin{tabular}[c]{@{}l@{}}Upload CSV\\ \cite{mohebbi2011google}\end{tabular} \\ \hline
\rowcolor[HTML]{AADFFD}
\cellcolor[HTML]{AADFFD} & \begin{tabular}[c]{@{}l@{}}Smoothing\\ (Figure \ref{zvOverview}D2)\end{tabular} & \begin{tabular}[c]{@{}l@{}}Interactively adjusting the level \\ of denoising on visualizations,\\ effectively changing the degree\\ of shape approximation when \\ performing pattern matching.\end{tabular} & \begin{tabular}[c]{@{}l@{}}\textbf{A, M:} Eliminate patterns \\ matched to spurious noise.\end{tabular} & \begin{tabular}[c]{@{}l@{}}Smoothing\\ ~\cite{Mannino2018}\\ Angular slope queries\\ ~\cite{Hochheiser2004}\\ Trend querylines\\ ~\cite{ryall2005querylines}\end{tabular} \\ \cline{2-5}
\rowcolor[HTML]{AADFFD}
\cellcolor[HTML]{AADFFD} & \begin{tabular}[c]{@{}l@{}}Range \\ Selection\\ (Figure \ref{zvOverview}B2, D4)\end{tabular} & \begin{tabular}[c]{@{}l@{}}Restrict to query only in \\ specific x/y ranges of interest \\ through brushing selected\\ x-range and filtering \\ selected y-range.\end{tabular} & \begin{tabular}[c]{@{}l@{}}\A Matching only around \\ shape exhibiting a peak.\\ \M Matching only around \\ shape region that exhibit linear\\ or exponential relationships\end{tabular} & \begin{tabular}[c]{@{}l@{}}Text Entry\\ ~\cite{wattenberg2001sketching,Mannino2018}\\ Min/max boundaries\\ ~\cite{ryall2005querylines}\\ Range Brushing\\ ~\cite{Hochheiser2001}\end{tabular} \\ \cline{2-5}
\rowcolor[HTML]{AADFFD}
\multirow{-3}{*}{\cellcolor[HTML]{AADFFD}\begin{tabular}[c]{@{}l@{}}Match\\ Specification\end{tabular}} & \begin{tabular}[c]{@{}l@{}}Range \\ Invariance\\ (Figure \ref{zvOverview}D1,4)\end{tabular} & \begin{tabular}[c]{@{}l@{}}Ignoring vertical or horizontal \\ differences in pattern matching \\ through option for x-range\\ normalization and y-invariant\\ similarity metrics .\end{tabular} & \begin{tabular}[c]{@{}l@{}}\A Searching for existence of a\\ peak above a certain amplitude.\\ \G Searching for a \\ ``generally-rising" pattern.\end{tabular} & \begin{tabular}[c]{@{}l@{}}Temporal invariants\\ ~\cite{correll2016semantics}\end{tabular} \\ \hline
\rowcolor[HTML]{FBE39C}
\cellcolor[HTML]{FBE39C} & \begin{tabular}[c]{@{}l@{}}Data selection\\ (Figure \ref{zvOverview}A)\end{tabular} & \begin{tabular}[c]{@{}l@{}}Changing the collection of \\ visualizations to iterate over.\end{tabular} & \begin{tabular}[c]{@{}l@{}}\M Explore tradeoffs and \\ relationships between \\ physical attributes.\end{tabular} & ---- \\ \cline{2-5}
\rowcolor[HTML]{FBE39C}
\multirow{-2}{*}{\cellcolor[HTML]{FBE39C}\begin{tabular}[c]{@{}l@{}}View\\ Specification\end{tabular}} & \begin{tabular}[c]{@{}l@{}}Display control\\ (Figure \ref{zvOverview}D4)\end{tabular} & \begin{tabular}[c]{@{}l@{}}Changing the details of \\ how visualizations should\\ be displayed.\end{tabular} & \begin{tabular}[c]{@{}l@{}}\M Non-time-series data should \\ be displayed as scatterplot.\end{tabular} & ---- \\ \hline
\rowcolor[HTML]{FBE39C}
\cellcolor[HTML]{FBE39C} & \begin{tabular}[c]{@{}l@{}}Filter\\ (Figure \ref{zvOverview}D3)\end{tabular} & \begin{tabular}[c]{@{}l@{}}Display and query only on data \\ that satisfies the composed \\ filter constraints.\end{tabular} & \begin{tabular}[c]{@{}l@{}}\A Eliminate unlikely \\ candidates by navigating to \\ more probable data regions.\\ \textbf{M, G:} Compare how overall\\ patterns change when filtered \\ to particular data subsets.\end{tabular} & ---- \\ \cline{2-5}
\rowcolor[HTML]{FBE39C}
\multirow{-2}{*}{\cellcolor[HTML]{FBE39C}Slice-and-Dice} & \begin{tabular}[c]{@{}l@{}}Dynamic Class \\ (Figure~\ref{dcc})\end{tabular} & \begin{tabular}[c]{@{}l@{}}Create custom classes of data \\ that satisfies one or more \\ specified range constraints. \\ Display aggregate \\ visualizations for separate\\ data classes.\end{tabular} & \begin{tabular}[c]{@{}l@{}}\textbf{A, M:} Examine aggregate \\ patterns of different data \\ classes.\end{tabular} & ---- \\ \hline
\rowcolor[HTML]{B5E1A4}
\begin{tabular}[c]{@{}l@{}}Result \\ Querying\end{tabular} & \begin{tabular}[c]{@{}l@{}}Drag-and-drop\\ (Figure \ref{zvOverview}C, E)\end{tabular} & \begin{tabular}[c]{@{}l@{}}Querying with any selected\\ result visualization as pattern\\ query (either from \\ recommendations or results).\end{tabular} & \begin{tabular}[c]{@{}l@{}}\textbf{A, G, M:} Find other objects that\\ are similar to X; Examine what \\ other objects similar to X look \\ like overall.\end{tabular} & \begin{tabular}[c]{@{}l@{}}Drag-and-drop\\ ~\cite{Hochheiser2001}\\ Double-Click\\ ~\cite{correll2016semantics}\end{tabular} \\ \hline
\rowcolor[HTML]{B5E1A4}
Recommendation & \begin{tabular}[c]{@{}l@{}}Representative \\ and Outliers\\ (Figure \ref{zvOverview}E)\end{tabular} & \begin{tabular}[c]{@{}l@{}}Displaying visualizations of \\ representative trends and outlier\\ instances based on clustering.\end{tabular} & \begin{tabular}[c]{@{}l@{}}\A Examine anomalies and debug \\ data errors through outliers.\\ \textbf{G, M:} Understand representative \\ trends common to this dataset \\ (or filtered subset).\end{tabular} & ---- \\ \hline
\end{tabular}
  \caption{\change{List of major features incorporated via participatory design. We organize each feature based on its functional component. Table cells are further colored according to the sensemaking process that each component corresponds to (Blue: Top-down, Yellow: Context creation, Green: Bottom-up). We list the functional purpose of each feature based on how it is implemented in \zvpp, example use cases from participatory design (\A astronomy, \M material science, \G genetics), and how similar features have been incorporated in past VQSs.}\label{bigfeaturetable}}
\end{table*}
%\dor{Trouble with vertically centering text in Component column}
\clearpage

 \npar M1 typically starts his data exploration process by iteratively applying filters to a list of potential battery solvents using SQL queries. Once the remaining solvent list is sufficiently small, he manually examines the properties of each solvent individually by examining the 3D chemical structure of the solvent in a custom software, as well as gathering information regarding the solvent by cross-referencing an external chemical database and existing uses of this solvent in literature. The collected information, including cost, availability, and other physical properties, enable researchers to select the final set of desirable solvents that they can feasibly experiment with in lab. They were interested in VQSs as it was impossible for them to uncover hidden relationships between different attributes across large numbers of solvents manually.%(such as how changing one attribute affects another attribute)
 \par Next, we describe the collaborative feature discovery process and the system prototype from participatory design.
 